\documentclass[a4paper,twoside,openright,12pt]{book}
\usepackage[utf8]{inputenc}  %Kodna stran za Windows okolje, za linux je kodna stran latin2
\usepackage[Slovene]{babel}    % pravila za slovensko deljenje besed
\usepackage[pdftex]{UNI-LJ-FE-Diploma} %Stil za diplome na Fakulteti za elektrotehniko (za pdfTeX v MkiTex)
%\usepackage[pctex]{UNI-LJ-FE-Diploma} %Stil za diplome na Fakulteti za elektrotehniko  (za pcTex)
\usepackage{xkeyval}
\usepackage{graphicx}
\usepackage{epstopdf}
\usepackage{tikz}
\usepackage{breqn}
\usetikzlibrary{calc}
\usepackage{subcaption}
\usepackage{hyperref} % za povezave do naslovov v pdf
\usetikzlibrary{decorations.pathreplacing}
\hypersetup{
    colorlinks,
    citecolor=black,
    filecolor=black,
    linkcolor=black,
    urlcolor=black
}
%*************************** PRILAGODITVE *****************************
% mapa s slikami
\potgrafike{./Slike/}
%prilagoditev levega roba sodih strani. �e se pri dvostranskem tisku robovi ne umemajo se lahko pove�a ali pomanj�a
\zamaknirobsodihstrani{0mm}
%*************************** NASLOVNA STRAN *****************************
\naslov{ Vpliv statične in dinamične ekscentričnosti magnetnega senzorja RM44 na napako v signalu kota}
\avtor{Mitja Alič} \univerza{Univerza v Ljubljani}
\fakulteta{Fakulteta za elektrotehniko}
\delo{Magistrsko delo}
\date{Ljubljana, 2018}
\mentor{doc. dr. Mitja Nemec}
\begin{document}
%------------------------ ZAČETNI DEL -----------------------------------
\frontmatter
%%------------------------------------------------------------------------
%%************************ NASLOVNA STRAN ********************************
\maketitle
%%*************************** ZAHVALA ************************************
\chapter*{Zahvala}
Zahvaljujem se mentorju doc. dr. Mitji Nemcu za pomoč pri izdelavi magistrskega dela. Prav tako se zahvaljujem sodelovcem laboratorija LRTME.
Zahvala gre tudi dr. Blažu Šmidu in drugim v podjetju RLS Merilna tehnika.
Zahvaljujem se družini in prijateljem, ki so me spodbujali in podpirali tekom celotnega študija.
%%*************************** VSEBINA *************************************
\tableofcontents
%%*************************** SEZNAM SLIK in TABEL  ***********************
%\seznamslik
%\seznamtabel
%%**************************  SEZNAM UPORABLJENIH SIMBOLOV  **************
\chapter*{Seznam simbolov}
V zaključnem delu so uporabljeni naslednje veličine in
simboli:
\begin{table}[h]
\centering
\begin{tabular}{l l l l}
 \hline \multicolumn{2}{c}{\bf{Veličina / oznaka}} & \multicolumn{2}{c}{\bf{Enota}}  \\
 \hline
Ime & Simbol & Ime & Simbol \\\hline
 
 napajalna napetost 					&$VDD$			&volt		&V\\
 ničelni potencial 						&$GND$			&volt		&V\\
 referenčni kot 						&$\Theta$  		&stopinja 	&$^\circ$ \\
 pomerjeni kot 							&$\varphi$  	&stopinja 	&$^\circ$ \\
 napaka									&$\varepsilon$	&stopinja 	&$^\circ$ \\
 Z-komponenta gostote magnetnega polja	&$B_z$			&militesla 	&mT \\
 statična ekscentričnost v x			&$\Delta x_s$	&milimetri 	&mm \\
 statična ekscentričnost v y			&$\Delta y_s$	&milimetri 	&mm \\
 dinamična ekscentričnost v x			&$\Delta x_d$	&milimetri 	&mm \\
 dinamična ekscentričnost v y			&$\Delta y_d$	&milimetri	&mm \\
  \hline
\end{tabular}
%%\end{footnotesize}
  \caption{Veličine in simboli}
  \label{prebojne_trdnosti}
\end{table}
%%------------------------ GLAVNI DEL ------------------------------------
\mainmatter
%%-------------------------------------------------------------------------
%%********************* POVZETEK V SLOVENŠČINI ****************************
\chapter*{Povzetek}


Za regulacijo pogonov se v industriji uporablja dajalnike zasuka.
Primer dajalnika zasuka je magnetni enkoder.
Magnetni enkoder je zaradi svoje robustnosti primeren za delovanje v takem okolju.

S Hallovimi sonadmi enkoder meri Z-komponento magnetnega polja magneta, ki se nahaja na rotirajočem delu pogona.
Iz signalov Hallovih sond enkoder pridobi signala v obliki sinusa in kosinusa. Preko matematičnega algoritma se izračuna kot zasuka.
Algoritem predstavlja izračun matematične funkcije $atan2()$.

Ob nepravilni montaži enkoderja ali magneta, se prostorski zajem signalov sinusne in kosinusne oblike spremeni. Posledično se v izračunanem kotu pojavi napaka.
Napaka kota se izrazi glede na nepravilno zajeto magnetno polje. V magistrskem delu je predstavljeno spreminjanje zajemanja magnetnega polja posamezne Hallove sonde v odvisnosti od nepravilne montaže.

Napako zaradi nepravilne montaže, se najlažje razbere iz signalov Hallovih sond. Izhod enkoderja je le podatek o zasuku.
Ker se za izračun kota potrebuje funkcijo $atan2()$, je v delu izpeljano, kako napaka v vhodnih signalih rezultira v napaki izhoda funkcije.

V programu MATLAB je bil sestavljen simulacijski model za merjenje napake enkoderja v odvisnosti od nepravilne montaže.
Simulacije so bile opravljene na linearno aprokismirani Z-komponenti gostote magnetnega polja in na numerično izračunanem polju realnega magneta.
Napaka ob napačni montaži je bila analizirana v frekvenčenem prostoru in je predstavljena kot potek posameznega harmonika v odvisnosti od nepravilne montaže.

Opravljene so bile meritve na enkoderju RM44, ki so bile primerjane s simulacijami. Meritve so potrdile pričakovane poteke napake predvidene s simulacijami. Na napravi je bila idealna lega določena empirično. Posledično se je to pokazalo tudi v potekih napake v odvisnosti od nepravilne montaže.

\kljucnebesede  dajalnik položaja, Hallova sonda, nepravilna montaža, predvidevanje napake, arcustangens
%%*************************** POVZETEK V ANGLE��INI ***********************
\chapter*{Abstract}

Position sensors are used for motor drive controls.
Usage of magnetic encoder is appropriate for sensing position in robust industrial environment.

Magnetic encoder senses the angular position of a permanent magnet placed above the sensor.
The permanent magnet must be diametrically polarized and of cylindrical shape.
Encoder uses Hall sensors to detect magnetic flux density.
Signals from Hall sensors are then converted to signals in shape of sine and cosine. From sine an cosine signal, an angle calculation algorithm is performed.
Algorithm could be presented by mathematical function of atan2().

Incorrect installation of sensor or magnet, causes error in output angle.
This work presents, how incorrect installation of sensor or magnet  impact to measured magnetic field by Hall sensor.

Incorrect installation can be recognized from field measured by Hall sensors. Encoders output signal is only angle.
In this thesis is demonstrated how to recognize changes of signals from Hall sensors, by knowing encoder output only.

The basic simulation model was built in MATLAB. Simulations were  made  by linear approximation of Z-component of magnetic field and numerical calculations by magnet model.
Encoders error was analyzed in frequency spectrum.

Simulations have been compared to the encoder RM44 measurements. 
Simulation results of error has been confirmed by measurements. Best installation of sensor was determined empirically. 
Empirical determination of installation, has been resulting in the output error.

\keywords position encoders, Hall effect sensor, superficial implementation, anticipating an error, arcustangens
%%%***************************** UVOD **************************************
\chapter{Uvod} \label{uvod}

Skozi celotno zgodovino so si ljudje želeli olajšati fizična dela na različne načine. Ponavljajoča dela je olajšala uporaba pogonov. Električni pogoni so delovne procese optimizirali. Za točnejše delovanje so se razvili različni načini krmiljenja. Z novimi načini krmiljenja, so se pojavile tudi potrebe po merjenju novih količin. V zadnjih desetletjih, je pri krmiljenuju, potrebna informacija o trenutnem položaju pogona.

Trenutni položaj merijo dajalniki pomika ali zasuka. Pri rotacijskih dajalnikih ločimo dajalnike, ki merijo zasuk na koncu osi (angl.: on axis) in dajalnike, ki merijo zasuk na osi (angl.: through hole). Možna delitev rotacijskih dajalnikov je tudi na eno-obratne (angl.: single-turn) in več-obratne (angl.: multi-turn). Eno-obratni rotacijski dajalniki podajo položaj znotraj enega obrata, medtem ko več-obratni štejejo tudi število polnih obratov. Dajalnike položaja delimo tudi glede na uporabljeni princip zaznavanja fizikalne
spremembe, torej glede na uporabljeno tehnologijo. Poznamo magnetne, optične,
induktivne in druge\cite{killer}.

Pri magnetnem principu senzor dajalnika zaznava spremembo jakosti in smeri
magnetnega polja. 
Magnetno polje se ustvari z aktuatorjem radialno polariziranega magneta. Meri se s Hallovimi sondami ali AMR senzorji. Iz zajetega polja sledi izračun dejanskega položaja. Dajalnik položaja, ki pretvarja merjeno magnetno polje v informacijo o položaju imenujemo enkoder.

Kot vsak merilni element, ima tudi magnetni enkoder napako. Napaka se lahko pojavi ob narobe merjenem magnetnem polju. Napako odpravimo s kalibracijo senzorjev zajema polja. Napako lahko povzroči tudi napačno pomerjeno polje. To se zgodi ob nepravilni montaži enkoderja ali magnetnega aktuatorja na pogon. S poznavanjem vplivov nepravilne montaže na napako pomerjenega položaja, se napako lahko predvivi in odstrani.
V tej magistrski nalogi je predstavljen vpliv napačno merenega magnetnega polja. Predstavljen je simulacijski model enkoderja, ter odvistnost napake na nepravilno montažo. Simulacije so primerjane z meritvami na enkoderju RM44.


\chapter{Senzor RM44}

Senzor RM44 je 13 bitni enkoder, primeren za merjenje zasuka in hitrosti elektromotorskega pogona.
Enkoder se nahaja v robustem ohišju, zato je primeren za delovanje v težkem industrijskem okolju.% , pritrjenega na konec rotirajoče osi pogonskega sklopa.
Obliko izhodnega podatka o zasuku, je prilagodljiva na sistem aplikacije v kateri bo uporabljen. Izhod senzorja je lahko analogni, v obliki sinusa in cosinusa, inkrementalni s signaloma A in B s katerih lahko izračunamo smer vrtenja ter signal Ri kateri določa referenčno točko. Izhod je lahko tudi digitalen preko komunikacijo SSI ali analogna napetost, ki se linearno spreminja med potencialom GND in Vdd v odvistnosti od kota zasuka. Senzor ima možnost nastavitev resolucije od 5 do 13 bitov \cite{RM44}\cite{AM8192}. Senzor na katerem so bile opravljene meritve je imel na voljo analogna signala sinus in kosinus. Točno ime senzorja je RM44AC0001S20F2E10, v delu bo poimenovan okrajšano na RM44.

Ključni element senzorja je čip AM512B. V čipu so Hallove sonde za meritev z-komponente gostote magnetnega pretoka. 

\begin{figure}[h]
	\centering
	\includegraphics[width=0.5\columnwidth]{./Slike/senzorRM44.jpg}
	\caption{Senzor RM44}
	\label{RM44}
\end{figure}



\chapter{Zastavljena naloga}

Senzor RM44 mora biti za pravilno delovnanje in točnost izhodnega podatka pravilno montiran.% V podatkovnih listih je podana toleranca $\pm100\mathrm{\mu m}$. 

Magistrsko delo predstavlja vpliv nepravilno montiranega senzorja ali nepravilno montiranega magneta na napako. Kako nepravilna montaža vpliva na izhodna signala sinus in kosinus ter neposredno iz tega tudi na napako. V delu je predstavljena tudi odvisnost napake pri spremembi idealnih signalov sinus in kosinus na izračunan kot.

V začetku je bila opravljena  izpeljava kako se giblje magnet ali senzor v sistemu z nepravilno montažo. Opravil sem simulacije na linearno aproksimiranem magnetnem polju, ter na numerično izračunanem polju simuliranega realnega magneta.
Tehnologija senzorja RM44 je poslovna skrivnost, zato je bil postavljen lasten simulacijski model senzorja, s pričakovanji, da bo rezultat  slabši od končnih meritev.

Na tej točki bi bilo primerno definirati pojme, kateri se bodo uporabljali tekom izdelave dela.
Izmik senzorja bo med spreminjanja kota zasuka postavljen fiksno in se njegova lokacija nebo spreminjala na os vrtenja. Ta izmik je poimenovan statična ekscentričnost.
V nalogi bo preverjeno kako vpliva izmik magneta na točnost izhodnega podatka. Ob izmiku magneta iz osi vrtenja se pojavi opletanje magneta. Lokacija središča magneta se spreminja glede na določen zasuk magneta. Opletanje magneta je poimenovano dinamična ekscentričnost.














%
%Senzorji se delijo resolverje in enkoderje. Resolver ima unikatno oblikovan rotorski aktuator, kjer se med vretenjem zaradi posebne oblike, zračna reža spreminja sinusno. To ima za posledico istovrstno spreminjanje upornosti magnetnih poti fluksa med primarnimi in sekundarnimi glavami navitij ter nato induciranih napetosti \cite{Ursic}.
%
%
%delijo na absolutne ali inkrementalne merilnike. Inkrementalni nam sporočijo relativo spremembo zasuka, ob prehodu referenčne točke se lahko senzor šele inicializira in od takrat dalje je možen izračun dejsanskega zasuka. Primer takega senzorja je optični senzor zasuka.
%
%Absolutni dajalniki zasuka lahko neglede na dan zasuk razbere dejansko vrednost zasuka rotorja. Primera sta 
%
%Enkoder oz. rotacijski enkoder je naprava 
\chapter{Analiti"cna izpeljava vplivov dinami"cne in stati"cne ekscentri"cnosti}

V tem poglavju bom analiti"cno prikazati vpliv napak omenjenih ekscentri"cnosti, ki se pojavita zaradi neprimerne vgradnje te vrste enkoderja. Napaki razli"cno vplivati na izhode senzorja, zato ju lahko obravnamvam posami"cno. Preko analiti"cne izpeljave bomo spoznali kako se spreminja lokacija Hall-ove sonde glede na magnet ob pravilni monta"zi. Z vpeljavo dodane ekscentri"cnosti v model bomo videli, kako se trajektorija gibanja Hall-ove sonde glede na magnet spremeni. S poznavanjem lokacije Hall-ove sonde nad magnetom bomo lahko od"citali vrednost \Bz.


\section{Definicija koordinatnih sistemov}

Definirajmo kartezi"cni koordinatni sistem, ki ima v izhodi"scu postavljen radialno magnetiziran magnet. Na poljubno to"cko $S_{h0}(x_0,y_0)$, vendar ne v izhodi"s"ce postavimo Hall-ovo sondo. Na sliki \ref{fig:def_kks} je prikazan tak sistem. Hall-ova sonda je postavljena na abcisno os za la"zje razumevanje. Vrednost $y_0$ je lahko poljubna in kon"cna re"sitev izpeljave bo splo"sna za poljubno lokacijo Hall-ove sonde v za"cetni legi.

\begin{figure}[h!]
	\centering
	\begin{tikzpicture}
	\magnet {0} {0} {0}{$S_r(0, 0)$}{1};
	\hall {2.3}{0} {0};
	\end{tikzpicture}
	\caption{Definicija koordinatnega sistema z magnetom in Hall-ovo sondo}
	\label{fig:def_kks}
\end{figure}

Z rotacijo magneta za kot $\theta$, se lokacija Hall-ove sonde glede na magnet spremeni. Nova lokacija Hall-ove sonde glede na magnet je enaka, "ce namesto magnet, zarotiramo Hall-ovo sondo za kot $-\theta$ . Novo lokacjo Hall-ove sonde glede na magnet lahko zapi"semo z rotacijsko matriko.

\begin{equation}
\label{equ:rotacija_hall}
\begin{bmatrix} x\\y \end{bmatrix}=
\begin{bmatrix} \cos(-\theta)&-\sin(-\theta)\\\sin(-\theta)&\cos(-\theta) \end{bmatrix}
\begin{bmatrix} x_0\\y_0 \end{bmatrix}
\end{equation}

Argument rotacijske matrike je $-\theta$, pri "cemer vemo, da smo namesto magneta zarotirali Hall-ovo sondo v nasprotno smer. Z upo"stevanjem lihosti funkcije sinus in sodosti funkcije kosinus, se ena"cba \ref{equ:rotacija_hall} poenostavi v:
\begin{equation}
\label{equ:rotacija_hall_simplify}
\begin{bmatrix} x\\y \end{bmatrix}=
\begin{bmatrix} \cos(\theta)&\sin(\theta)\\-\sin(\theta)&\cos(\theta) \end{bmatrix}
\begin{bmatrix} x_0\\y_0 \end{bmatrix}
\end{equation}




\begin{figure}[h!]
%	\centering


    \begin{subfigure}[b]{0.5\textwidth}
	\centering
	
		\begin{tikzpicture}
		[scale=1, every node/.style={scale=1}]
				\magnet {0} {0} {30}{}{1}
				\hall {2.3}{0} {0};
				\draw [dashed](0,0)--(2.3,0) arc (0:30:2.3)--(0,0);
				\node at(1,0.3){$\theta$};
                \draw [->] (1.5,0) arc (0:30:1.5);
		\end{tikzpicture}
	\caption{Zasukan magnet za kot $\mathrm{\theta}$}
	\label{subfig:zasuk_magnet}
\end{subfigure}
\begin{subfigure}[b]{0.5\textwidth}
	\centering
	
		\begin{tikzpicture}[scale=1, every node/.style={scale=1}]		
				\magnet {0} {0} {0}{}{1}
				\hall {1.99}{-1.15} {-30};
				\draw [dashed](0,0)--(2.3,0) arc (0:-30:2.3)--(0,0);
				\node at(1,-0.3){-$\theta$};
                \draw [->] (1.5,0) arc (0:-30:1.5);
		\end{tikzpicture}
	
	\caption{Zasukan senzor za kot $\mathrm{-\theta}$}
	\label{subfig:zasuk_hall}
\end{subfigure}

\caption{Sprememba lokacije glede na magnet ob rotaciji}
\label{fig:zasuk_magneta}

\end{figure}



\section{Izpeljava gibanja lokacije Hall-ove sonde na magnet pri dinami"cni ekscentri"cnosti}

Opazujmo sedaj sistem gibanja Hall-ove sonde glede na magnet ter dinami"cno ekscentri"cnost. Magnet je
 postavljen v izhodi"sce koordinatnega sistema $S_m(0,0)$. Sedaj magnet izmaknemo v novo lego $S_{m1}(\Delta
  x_d,\Delta y_d)$ (Slika \ref{fig:def_din_eks}). Os vrtenja je "se vedno postavljena v izhodi"s"ce
   koordinatnega sistema. Sredi"sce magneta $S_{m1}(\Delta x_d,\Delta y_d)$ tako tekom vrtenja okoli
    koordinatnega izhodi"sca opi"se kro"znico z radijem $\sqrt{\Delta x_d^2+\Delta y_d^2}$. V sistem sedaj
     dodajmo Hall-ovo sondo v njeno za"cetno lego glede na izhodi"sce $S_{h0}(x_0,y_0)$.




\begin{figure}[h!]
	\centering
	\begin{tikzpicture}
		\magnet {0.6} {0.3} {0}{$S_{m1}(\Delta x_d,\Delta y_d)$}{1};
		\draw [dashed]  (0,0) circle (0.67);
		\draw (0,0)--(0.6,0.3);
		\fill (0,0) circle [radius=1pt];
		\node at (0.35,-0.3){$S_0(0,0)$};
		\hall {2.3}{0} {0};
		\kks{3}
	\end{tikzpicture}
	\caption{Shema definicije dinami"cne ekscentri"cnosti vpliva na magnet}
	\label{fig:def_din_eks}
\end{figure}




Enako gibanje Hall-ove sonde na magnet lahko dose"zemo tudi z obrnjenim sistemom. Vrnimo magnet v izhodi"s"cno lego $S_m(0,0)$. Sedaj postavimo os vrtenja magneta v to"cko $(-\Delta x_d,-\Delta y_d)$. Hall-ovo sondo postavimo v to"cko $S_{h1}(x_0-\Delta x_d,y_0 - \Delta y_d).$



\begin{figure}[h!]
	\centering
	\begin{tikzpicture}
		\magnet {0} {0} {0}{$S_{m}(0,0)$}{1};
		\draw (0,0)--(-0.6,-0.3);
		\fill (-0.6,-0.3) circle [radius=1pt];
		\node at (0,-0.6){$S_0(-\Delta x_d,-\Delta y_d)$};
		\hall {1.7}{-0.3} {0};
		\kks{3}
	\end{tikzpicture}
	\caption{Shema definicije dinami"cne ekscentri"cnosti vpliva na Hall-ovo sondo}
	\label{fig:def_din_eks_na_stator}
\end{figure}

Sistema prikazana na slikah \ref{fig:def_din_eks} in \ref{fig:def_din_eks_na_stator}, se v za"cetnih legah ne razlikujeta. Sedaj zarotirajmo Hall-ovo sondo okoli osi vrtenja $S_0(-\Delta x_d,-\Delta y_d)$. Hall-ova sonda se giblje glede na magnet enako, kot "ce bi magnet zavrteli z dinami"cno ekscentri"cnostjo (Slika \ref{fig:def_din_eks}). Gibanje Hall-ove sonde na magnet je izra"zeno kot gibanje po kro"znici s sredi"s"cem v to"cki $(-\Delta x_d,-\Delta y_d)$.

\begin{figure}[h!]
	\centering
	\begin{tikzpicture}
		\magnet {0} {0} {0}{}{1};
		\draw [dotted]  (-0.6,-0.3) circle (2.3);
		\draw (0,0)--(-0.6,-0.3);
		\fill (-0.6,-0.3) circle [radius=1pt];
		\hall {1.39}{-1.45} {-30};
		\draw [dashed](-0.6,-0.3)--(1.7,-0.3) arc(0:-30:2.3)--(-0.6,-0.3);
		\node at(0.4,-0.6){$-\theta$};
        \draw [->] (0.9,-0.3) arc (0:-30:1.5);
		\kks{3}
	\end{tikzpicture}
	\caption{Potek Hall-ove sonde ob rotaciji glede na magnet ob dinami"cni ekscentri"cnosti}
	\label{fig:potek_sonde_din_eks}
\end{figure}

Potek Hall-ove sonde ob rotaciji z upo"stevanjem dinami"cne ekscentri"cnosti lahko zapi"semo kot (\ref{equ:rotacija_hall_simplify}) z dodatkom enosmerne komponente dinami"cne ekscentri"cnosti.

\begin{equation}
\label{equ:rotacija_hall_din}
\begin{bmatrix} x\\y \end{bmatrix}=
\begin{bmatrix} \cos(\theta)&\sin(\theta)\\-\sin(\theta)&\cos(\theta) \end{bmatrix}
\begin{bmatrix} x_0\\y_0 \end{bmatrix}
+
\begin{bmatrix} -\Delta x_d\\-\Delta y_d \end{bmatrix}
\end{equation}

V (\ref{equ:rotacija_hall_din}) lahko izrazimo - in izraz se poenostavi.

\begin{equation}
\label{equ:rotacija_hall_din_simplify}
\begin{bmatrix} x\\y \end{bmatrix}=
\begin{bmatrix} \cos(\theta)&-\sin(\theta)\\\sin(\theta)&\cos(\theta) \end{bmatrix}
\begin{bmatrix} x_0\\y_0 \end{bmatrix}
-
\begin{bmatrix} \Delta x_d\\\Delta y_d \end{bmatrix}
\end{equation}

\section{Izpeljava gibanja lokacije Hall-ove sonde na magnet pri stati"cni ekscentri"cnosti}


Postavimo sistem nazaj v izhodi"scno lego, brez ekscentri"cnosti. Tako sredisce magneta, kot os vrtenja postavimo v izhodi"sce. Hall-ova sonda je postavljena v to"cko $S_{h0}(x_0,y_0)$. Sedaj premaknimo Hall-ovo sondo za $(\Delta x_s, \Delta y_s)$, v novo to"cko $S_{h1}(x_0+\Delta x_s, y_0+\Delta y_s)$. Na sliki \ref{fig:def_sta_eks} je prikazana le stati"cna ekscentri"cnost v y-osi, vendar celotni razmislek velja za obe stati"cni ekscentri"cnosti enako.


\begin{figure}[h!]
	\centering
	\begin{tikzpicture}
	\magnet {0} {0} {0}{}{1};
	\hall {2.4}{-0.6} {0};
	\draw[<->,dotted] (0,0)--(0,-0.6);
	\draw[dotted] (2.3,-0.6)--(0,-0.6);
	\node at (0.4,-0.3){$\Delta y_s$};
	\end{tikzpicture}
	\caption{Shema definicije stati"cne ekscentri"cnosti}
	\label{fig:def_sta_eks}
\end{figure}

Po enakem razmi"sljanju kot v zgornjih poglavjih, sedaj zarotirajmo Hall-ovo sondo za kot \kol{-\theta} okoli izhodi"s"ca. Hall-ova sonda se giblje po kro"znici z radijem $\sqrt{(x_0+\Delta x_s)^2+(y_0+\Delta y_s)^2}$.


\begin{figure}[h!]
	\centering
	\begin{tikzpicture}
	\magnet {0} {0} {0}{}{1};
	\hall {1.69}{-1.67} {-30};
	\draw[<->,dotted] (0,0)--(-0.3,-0.52);
	\draw[dotted] (1.69,-1.67)--(-0.3,-0.52);
	\draw [dashed] (0,0) -- (2.377,0) arc(0:-30:2.377)--(0,0);
	\node at (0.8,-0.2) {\kol{-\theta}};
	\draw [dotted] (0,0)circle[radius=2.377];
%    \draw [dotted, <->] (0,0)--(2.06,1.19);
%    \node at (4.03,0.6){$\sqrt{(x_0+\Delta x_s)^2+(y_0+\Delta y_s)^2}$};
	\draw [dotted, <->] (0,0)--(1.69,-1.67);
    \draw [->] (1.5,0) arc (0:-30:1.5);
	\end{tikzpicture}
	\caption{Potek Hall-ove sonde ob rotaciji glede na magnet ob stati"cni ekscentri"cnosti}
	\label{fig:def_sta_eks_stat}
\end{figure}


To lahko zapi"semo v izraz (\ref{equ:rotacija_hall_simplify}) kot:

\begin{equation}
\label{equ:rotacija_hall_stat}
\begin{bmatrix} x\\y \end{bmatrix}=
\begin{bmatrix} \cos(\theta)&\sin(\theta)\\-\sin(\theta)&\cos(\theta) \end{bmatrix}
\begin{bmatrix} x_0+\Delta x_s\\y_0+\Delta y_s \end{bmatrix}
\end{equation}







\section{Kon"cna ena"cba za dolo"canje lokacije Hall-ove sonde}

Do sedaj smo postopoma izpeljali ena"cbe za:
\begin{itemize}
  \item sistem magneta in Hall-ove sonde ob pravilni monta"zi
  \item sistem magneta in Hall-ove sonde z dinami"cno ekscentri"cnostjo magneta
  \item sistem magneta in Hall-ove sonde z stati"cno ekscentri"cnostjo Hall-ove sonde
\end{itemize}

Ena"cbi sistema z ekscentri"cnostjo sti med seboj neodvisni zato lahko ena"cbe sistemov zdru"zimo. Uporabimo princip superpozicije in dobimo kon"cno ena"cbo za lociranje Hall-ove sonde glede na magnet v odvistnosti od zasuka magneta, z upo"stevanjem vpliva tako dinami"cne kot stati"cne ekscentri"cnosti. Kon"cna ena"cba se glasi:

\begin{equation}
\label{equ:rotacija_hall_koncna}
\begin{bmatrix} x\\y \end{bmatrix}=
\begin{bmatrix} \cos(\theta)&\sin(\theta)\\-\sin(\theta)&\cos(\theta) \end{bmatrix}
\begin{bmatrix} x_0+\Delta x_s\\y_0+\Delta y_s \end{bmatrix}-
\begin{bmatrix} \Delta x_d\\\Delta y_d \end{bmatrix}
\end{equation}


Ogledali smo si, kako je ob rotaciji locirana Hall-ova sonda glede na magnet. Ogledali smo si tudi, kako na lokacijo sonde vplivati dinami"cna in stati"cna ekscentri"cnost. S poznavanjem magnetnega polje $B_z=B_z(x , y)$, lahko dolo"cimo kak"sno vrendost polja $B_z$ pomeri Hall-ova sonda ob rotaciji ($B_z=B_z(\theta)$). Ob poznavanju polja $B_z$, lahko dolo"cimo zasuk magneta glede na postavitev Hallove sonde.


\chapter{Izpeljava poteka polja $B_z(\theta)$ in ocena napake zaradi ekscentri"cnosti}

V tem poglavju si bomo ogledali kak"sno polje pomeri Hall-ova sonda, z linearno aproksimiranim magnetnim poljem $B_z$. Preko pomirjenega polja, bomo izra"cunali kak"sna je napake pomerjenega kota od referen"cnega in kako se napaka spreminja z ekscentri"cnostjo.

\section{Definicija  gostote magnetnega polja $B_z$}


Dajalnik pozicije RM44 meri z komponento gostote magnetnega polja zato se lahko osredoto"cimo le nanjo. Potek komponente $B_z$ nad cilindri"cnim magnetom je prikazan na sliki \ref{fig:magnetno_polje}.

%Magsnetno polje  v prostoru lahko izra"cunamo z Biot-Savartovim zakonom. Poznati moramo specifikacije trajnega magneta in izra"cunati integral po prostoru. Tako dobimo v poljubni to"cki v prostoru vrednost B.  Hall-ove sonde v senzorju RM44 merijo le z-komponento magnetnega polja, zato se lahko osredoto"cimo le nanjo. Odvistnost z-komponente vektorja B na konstantni vi"sini od magneta je vidno na sliki \ref{fig:magnetno_polje}.



\begin{figure}[h]
	\centering
		\includegraphics[width=0.75\columnwidth]{./Slike/magnetno_polje.jpg}
	\caption{z-komponenta vektorja gostote magnetnega polja nad cilindri"cnim magnetom}
	\label{fig:magnetno_polje}
\end{figure}


Potek z-komponente lahko izra"cunamo po Biot-Savartovim zakonom oz. numeri"cno se"stejemo prispevke posameznih del"ckov magneta. Tako dobimo vrednost celotnega vektorja gostote magnetnga polja v posamezni to"cki. Magnetno polje z komponente v okolici osi vrtenja magneta lahko aproksimiramo z ravnino

\begin{equation}
\label{equ:poljeB}
B_z(x,y)=k\cdot x.
\end{equation}

Tak"sna aproksimacija zadostuje za ocenitev poteka napake. S poznavanjem lokacije Hall-ove sonde, kar smo si ogledali v prej"snjem poglavju, sedaj dobimo potek pomerjene komponente gostote magnetnega polja. Aprokisirano polje je linearno odvisno od x komponente. Za la"zje razumevanje definirajmo konstanto $k$ enako 1.

\section{Postavitev Hall-ovih sond za zajem polja in pomerjeno polje v odvistnosti od ekscentri"cnosti}
Sedaj si oglejmo, kako bi dolo"cili kot zasuka poljubne to"cke okoli izhodi"s"ca. Definirajmo kartezi"cni koordinatni sistem, in v njem poljubno to"cko $(x_0,y_0)$, ki ni v izhodi"s"cu(Slika \ref{fig:dolocitev_kota}).
Za dolo"canje kota $\varphi$ je potrebno poznati poznati polo"zaj to"cke. Kot $\varphi$ dolo"cimo preko trigonometri"cne funkcije $\arctan$: $$\varphi=\arctan\frac{y_0}{x_0}$$



\begin{figure}[h!]
	\centering
	\begin{tikzpicture}[scale=4]
	\CaCS{0.75}{0}{0}
	\draw [->,thick](0,0)--(-0.5,0.2855);
	\draw (0.15,0) arc (0:150:0.15);
	\node at (0.05,0.05){$\varphi$};
%	\node at (-0.5,0.285) {\textbullet};
	\node at (-0.32,0.35) {($x_0$,$y_0$)};
	\end{tikzpicture}
	\caption{Slika za pomo"c pri dolo"canju kota}
	\label{fig:dolocitev_kota}
\end{figure}


Za dolo"citev kota $\varphi$  je dovolj poznati "ze projekciji vektorja na koordinatni osi(slika \ref{fig:dolocitev_kota_2}),

\begin{figure}[h!]
	\centering
	\begin{tikzpicture}[scale=4]
	\CaCS{0.75}{0}{0}
	\draw [->,thick](0,0)--(-0.5,0.2855);
	\draw (0.15,0) arc (0:150:0.15);
	\node at (0.05,0.05){$\varphi$};
	\draw [dashed] (-0.5,0.2855)--(-0.5,0);
	\draw [dashed] (-0.5,0.2855)--(0,0.2855);
	%	\node at (-0.5,0.285) {\textbullet};
%	\node at (-0.32,0.35) {($x_0$,$y_0$)};
	\node at(-0.5,-0.1){$x_0$} ;
	\node at(0.1,0.2855){$y_0$};
	\end{tikzpicture}
	\caption{Slika za pomo"c pri dolo"canju kota}
	\label{fig:dolocitev_kota_2}
\end{figure}


"Ce poznamo projekciji to"cke na koordinatni osi, je to zadosten pogoj za dolo"citev kota $\varphi$.
Projekcijo lahko pridobimo "ce opazujemo projekciji polo"zaja to"cke v koordinatnih oseh.

Sedaj si predstavljajmo da ta poljubna to"cka predstavlja enega od polov magneta. Za poznavanje zasuka pola magneta, je dovolj od"citanje polja na koordinatnih oseh. Hall-ovi sondi ne smeti biti postavljeni na isto koordinatno os. Ni nujno da sta sondi postavljeni pravokotno druga na drugo, si pa s tem prihranimo korak v katerem bi bilo potrebno izra"cunati projekcijo na pravokotni koordinatni osi.

Iz zgornjega razmisleka lahko sedaj smiselno postavimo Hall-ovi sondi v koordinatni sistem. Najprimerneje ju je postaviti na koordinatni osi (Slika \ref{fig:zacetna_postavitev_sond}). Sondi postavimo na enako razdaljo od izhodi"s"ca $r_0$. Tako bo zajem poteka polja ob rotaciji magneta enak, le fazno zamaknjeno.



\begin{figure}[h!]
	\centering
	\begin{tikzpicture}[scale=1]
	\CaCS{3}{0}{0}
	\senzorja{0}{0}{0}{}
%	\magnet {0} {0} {10}{ }{0}
	\node at (2.0,-0.5){$\mathrm{H}_1(r_0,0)$};
	\node at (-1,2.3){$\mathrm{H}_2(0, r_0)$};
	\end{tikzpicture}
	\caption{Za"cetna postavitev Hallovih sond}
	\label{fig:zacetna_postavitev_sond}
\end{figure}

S poznavanjem lociranja sonde glede na magnet (\ref{equ:rotacija_hall_koncna}), funkcije polja (\ref{equ:poljeB}) ter za"cetne pozicije Hall-ovih sond lahko dolo"cimo potek polja sonde.

\begin{equation}\label{equ:Bx_splosna}
cos=B_{H_1}(\theta,r_0,\Delta x_s, \Delta y_s, \Delta x_d)= r_0 \cos\theta +\Delta x_s \cos\theta +\Delta y_s \sin\theta -\Delta x_d
\end{equation}
\begin{equation}\label{equ:By_splosna}
sin=B_{H_2}(\theta,r_0,\Delta x_s, \Delta y_s, \Delta x_d)= r_0 \sin\theta +\Delta x_s \cos\theta +\Delta y_s \sin\theta-\Delta x_d
\end{equation}

Zajeta signala bom od tu naprej imenoval sinus ($sin$) in cosinu ($cos$), ker to je njuna osnovna oblika.



\subsection{Sprememba magnetnega polja zaradi ekscentri"cnosti}

Oglejmo si primer kak"sno polje zajameti Hall-ovi sondi, ko ekscentri"cnosti ni. $sin$ in $cos$ izraza se poenostavita in dobimo poteka v obliki sinusa ter kosinusa z enako amplitudo $r_0$ (Slika \ref{sim_lin_polje_xd_000u_BxBy}).

\slikaeps{Poteka $sin$ in $cos$ brez ekscentri"cnosti}{sim_lin_polje_xd_000u_BxBy}

Upo"stevajmo sedaj le stati"cni ekscentri"cnosti $\Delta x_s$ in $\Delta y_s$. $\Delta x_d$ postavimo na 0.   Ena"cbi (\ref{equ:Bx_splosna}) in (\ref{equ:By_splosna}) lahko preuredimo v izraza:


\begin{equation}
\label{equ:Bx_stat}
cos(\theta,r_0,\Delta x_s, \Delta y_s)= \sqrt{(r_0+\Delta x_s)^2+\Delta y_s^2}\cos(\theta -\arctan \frac{\Delta y_s}{r_0+\Delta x_s})
\end{equation}
\begin{equation}\label{equ:By_stat}
sin(\theta,r_0,\Delta x_s, \Delta y_s)= \sqrt{\Delta x_s^2+(r_0+\Delta y_s)^2} \sin(\theta +\arctan \frac{\Delta x_s}{r_0+\Delta y_s})
\end{equation}

Iz njiju vidimo spremenjena poteka. Signaloma se je spremenila amplituda in fazni zamik (Slika \ref{sim_lin_polje_xs_200u_BxBy}).

\slikaeps{Poteka $sin$ in $cos$ z upo"stevanjem 0,2 mm stati"cni ekscentri"cnosti v x-osi }{sim_lin_polje_xs_200u_BxBy}

Postavimo sedaj vrednosti $\Delta x_s$ in $\Delta_ys$ na 0, $\Delta x_d$ predpostavimo da ni 0.
\begin{equation}
\label{equ:Bx_din}
cos(\theta,r_0,\Delta x_s, \Delta y_s, \Delta x_d)= r_0 \cos\theta-\Delta x_d
\end{equation}
\begin{equation}
\label{equ:By_din}
sin(\theta,r_0,\Delta x_s, \Delta y_s, \Delta x_d)= r_0 \sin\theta-\Delta x_d
\end{equation}
Polji obdr"zita enako amplitudo ter fazo, vendar dobita enosmerno komponento, ki je premo sorazmerna z izmikom magneta iz osi vrtenja (Slika \ref{sim_lin_polje_xd_200u_BxBy}).
\slikaeps{Poteka $sin$ in $cos$ z upo"stevanjem 0,2 mm dinami"cne ekscentri"cnosti v x-osi}{sim_lin_polje_xd_200u_BxBy}











\section{Izra"cun kota}

S poznavanjem potekov polja posamezne sonde sedaj s funkcijo $\arctan$ izra"cunamo kot.

\begin{equation}
\label{equ:izracun_kota_splosna}
\varphi(\theta,\Delta x_s, \Delta y_s, \Delta x_d)=\arctan\frac{B_y}{B_x}=\arctan\frac{r_0 \sin\theta +\Delta x_s \cos\theta +\Delta y_s \sin\theta-\Delta x_d}{r_0 \cos\theta +\Delta x_s \cos\theta +\Delta y_s \sin\theta -\Delta x_d}
\end{equation}





Iz podanega izraza (\ref{equ:izracun_kota_splosna}) je te"zko sklepati, kak"sen bo potek pomerjenega kota.
Na tem mestu definirajmo napako merjenega kota:
\begin{equation}
\varepsilon=\varphi-\theta
\end{equation}
Signala $sin$ in $cos$ imata periodo $360^\circ$ in sta zvezna. Iz tega sledi, da bo tudi $\varepsilon$ zvezen in imel periodo $360^\circ $.Pri"cakujem da bo potek napake $\varepsilon$ v obliki :
\begin{equation}
\label{equ:nastavek}
\varepsilon=A_0+A_1 \cos \theta +B_1 \sin \theta+A_2 \cos 2\theta +B_2 \sin 2\theta
\end{equation}

Za pribli"zek napake izraz  (\ref{equ:izracun_kota_splosna}) razvijmo v Taylorjevo vrsto po kotu $\theta$. V Taylorjevo vrsto razvjemo tudi nastavek pri"cakovanega poteka. Zaradi aproksimacije poteka merjenega kota se zadovoljimo z razvojem do petega reda. Za poenostavitev se bom ekscentri"cnosti lotil posamezno. V naslednjih podpoglavjih bom prikazal analiti"cne rezultate posameznega harmonika. Izpeljavo bom le teoreti"cno opisal.

Oba izraza i (\ref{equ:izracun_kota_splosna} in \ref{equ:nastavek})razvijem do petega reda Taylorjeve vrste. Z zdru"zitvijo posameznih potenc $\theta$, pridobimo sistem petih en"cb s petimi neznankami. S tem pridobim neznane faktorje $A_0$, $A_1$, $A_2$, $B_1$ in $B_2$. S poznavanjem teh faktorjev lahko ocenimo kas"seni bodo poteki posameznih harmonikov, ob posameznih ekscentri"cnostih.




 

\subsection{Aproksimacija pomerjenega kota ob stati"cni ekscentri"cnosti x}

V izrazu (\ref{equ:izracun_kota_splosna}) upo"stevamo le stati"cno ekscentri"cnost $\Delta x_s$.

\begin{equation}
\label{equ:izracun_kota_xs}
\varphi=\arctan \frac{r_0 \sin\theta +\Delta x_s \cos\theta}{r_0 \cos\theta +\Delta x_s \cos\theta}
\end{equation}


Z aproksimacijo po izrazu (\ref{equ:nastavek}) pridobimo koeficiente posameznega harmonika


\begin{eqnarray}
{A_0}&=\frac{-90 r_0^2  \Delta y_s (r_0^5+29 r_0^4 \Delta y_s+132 r_0^3  \Delta y_s^2+208 r_0 ^2  \Delta y_s^3+156 r_0  \Delta y_s^4+52 \Delta y_s^5)}{\pi (r_0^2+2 r_0 \Delta y_s+2 \Delta y_s^2)^4}+\arctan \frac{ \Delta y_s}{r_0+ \Delta y_s}\\
{A_1}&=\frac{2280 r_0^2  \Delta y_s^2(r_0^4+5 r_0^3  \Delta y_s+8 r_0^2  \Delta y_s^2+6 r_0  \Delta y_s^3+2 \Delta y_s^4)}{\pi(r_0^2+2 r_0 \Delta y_s+2 \Delta y_s^2)^4}\\
{B_1}&=-\frac{240 \Delta y_s^3(7r_0^3+18 r_0^2  \Delta y_s+18 r_0^2  \Delta y_s+18 r_0  \Delta y_s^2+8 \Delta y_s^3)}{\pi(r_0^2+2 r_0 \Delta y_s+2 \Delta y_s^2)^3}\\
{A_2}&=\frac{90 r_0^2  \Delta y_s (r_0^5-3 r_0^4 \Delta y_s-28 r_0^3  \Delta y_s^2-48r_0 ^2  \Delta y_s^3-36 r_0  \Delta y_s^4-12 \Delta y_s^5)}{\pi(r_0^2+2 r_0 \Delta y_s+2 \Delta y_s^2)^4}\\
{B_2}&=\frac{30 \Delta y_s (-3r_0^5-18 r_0^4 \Delta y_s-20 r_0^3  \Delta y_s^2 +12 r_0  \Delta y_s^4+8 \Delta y_s^5)}{\pi(r_0^2+2 r_0 \Delta y_s+2 \Delta y_s^2)^3}\\
\end{eqnarray}

Oglejmo si sliko potekov posameznih harmonikov.
\slikaeps{Poteki amplitud prvega in drugega harmonika ter enosmerne komponente ob spreminjanju stati"cne ekscentri"cnosti v x-osi}{potek_analitika_xs}


Te rezultate Taylorjeve vrste lahko upo"stevam le v okolici ni"cle. Iz rezultatov lahko pri"cakujemo nara"s"canje drugega harmonika ter nara"s"canje enosmerne komponete.






\subsection{Aproksimacija pomerjenega kota ob stati"cni ekscentri"cnosti y}

V izrazu (\ref{equ:izracun_kota_splosna}) upo"stevamo le stati"cno ekscentri"cnost $\Delta y_s$.

\begin{equation}
\label{equ:izracun_kota_ys}
\varphi=\arctan \frac{r_0 \sin\theta +\Delta y_s \sin\theta}{r_0 \cos\theta +\Delta y_s \sin\theta}
\end{equation}


Z aproksimacijo po izrazu (\ref{equ:nastavek}) pridobimo koeficiente posameznega harmonika


\begin{eqnarray}
A_0&=\frac{-90 y_s (r_0^2-23r_0 y_s-24y _s^2)}{\pi r_0^3}\\
A_1&=\frac{-1690 y_s^2(r_0+y_s)}{\pi r_0^3}\\
A_2&=\frac{90 y_s(r_0^2+9r_0y_s+8y_s^2)}{\pi r_0^3}\\
B_1&=\frac{240 y_s^3}{\pi r_0^3}\\
B_2&=\frac{30 (3 r_0^2y_s-4 y_s^3)}{\pi r_0^3}
\end{eqnarray}

Oglejmo si sliko potekov posameznih harmonikov.
\slikaeps{Poteki amplitud prvega in drugega harmonika ter enosmerne komponente ob spreminjanju stati"cne ekscentri"cnosti v y-osi}{potek_analitika_ys}


Iz grafa je razvidno, da do enosmerna komponenta upadala, drugi harmonik bo najbolj izrazit.




\subsection{Aproksimacija pomerjenega kota ob dinami"cni ekscentri"cnosti x}

Izraz  (\ref{equ:izracun_kota_splosna}) kjer upo"stevamo le dinami"cno ekscentri"cnost se poenostavi v:


\begin{eqnarray}
A_0&=\frac{180(r_0 \Delta x_d(r_0^6-7r_0^5 \Delta x_d+18r_0^4 \Delta x_d^2-36r_0^2 \Delta x_d^4 +28r_0 \Delta x_d^5-8 \Delta x_d^6))}{\pi(r_0^2-2r_0 \Delta x_d+2 \Delta x_d^2)^4}-\arctan \frac{\Delta x_d}{r_0- \Delta x_d}\\
A_1&=\frac{-180(r_0 \Delta x_d(r_0^6-8r_0^5 \Delta x_d+22r_0^4 \Delta x_d^2-44r_0^2 \Delta x_d^4 
	+32r_0 \Delta x_d^5-8 \Delta x_d^6)}{\pi (r_0^2-2r_0 \Delta x_d+2 \Delta x_d^2)^4)}\\
A_2&=\frac{-180(r_0^2 \Delta x_d^2 (r_0^4-4 r_0^3  \Delta x_d+8 r_0  \Delta x_d^3-4 
	\Delta x_d^4)}{\pi (r_0^2-2 r_0  \Delta x_d+2  \Delta x_d^2)^4)}\\
B_1&=\frac{60( \Delta x_d (3 r_0^5-18 r_0^4  \Delta x_d+64 r_0^3  \Delta x_d^2 -108 
	r_0^2  \Delta x_d^3+84 r_0  \Delta x_d^4-32  \Delta x_d^5))}{\pi (r_0^2-2 r_0 
	\Delta x_d+2  \Delta x_d^2)^3}\\
B_2&=\frac{60 (2  \Delta x_d^3 (-4 r_0^3+9 r_0^2  \Delta x_d-6 r_0  \Delta x_d^2+2  \Delta x_d
	^3))}{\pi(r_0^2-2 r_0  \Delta x_d+2  \Delta x_d^2)^3}
\end{eqnarray}


\slikaeps{Poteki amplitud prvega in drugega harmonika ter enosmerne komponente ob spreminjanju dinami"cne ekscentri"cnosti v x-osi}{potek_analitika_xd}



S slike \ref{potek_analitika_xd} vidimo da bo izrazit le prvi harmonik, ki pri majhnih odmikih nara"s"ca linearno.





















\chapter{Linearni model}

V prej"snjem poglavju smo magnetno polje magneta aproksimirali z ravnino ter napako izra"cunali z neskončno vrsto. V tem poglavju bom predstavil simulacije opravljene na magnetnem polju aproksimiranega z ravnino, izra"cnan kot $\varphi$ je rezultat numeri"cne funkicje atan2d(y,x) citeatan2d. Predstavil bom napako, jo razstavil na posamezne harmonike, ter prikazal spreminjanje amplitud glede na spremembo ekscentri"cnosti. Ravnina aproksimiranega magnetnega polja je: 
\begin{equation}
\label{equ:lin_polje}
B(x,y)= x
\end{equation}
Hall-ovi sondi sti postavljeni na kro"znico z radijem 2,4 mm.


\section{Brez napake}

Za za"cetek si poglejmo idealno montriran tako senzor kot magnet. Signala $sin$ in $cos$ imata enaki amplitudi in sta fazno zamaknjena za $90^{\circ}$. Napaka $\varepsilon$, ki se pojavi pri izra"cunu je tako le numeri"cna napaka funkcije atan2d (Slika \ref{./LIN/lin_00_napaka}).
\slikaeps{Potek signalov $sin$ in $cos$ ob idealni montaži}{./LIN/lin_00_sincos}
\slikaeps{Napaka $\varepsilon$ pri simulacijah z linearnim magnetnim poljem pri idealni monta"zi}{./LIN/lin_00_napaka}

Numeri"cno napako lahko na pri"cakovano napako zaradi ekscnetri"cnosti zanemarim.


\newpage
\section{Simulacija stati"cne ekscentri"cnosti v smeri x-osi}

Oglejmo si rezultate simulacij stati"cne ekscentri"cnosti v smeri x. Po pri"cakovanjih se bo povi"sala amplituda $sin$ in $cos$ signala ter zmanj"sal njun fazni zamik (izraza (\ref{equ:Bx_stat}) in \ref{equ:By_stat}).
 
\slikaeps{Signala $sin$ in $cos$ pri simulacijah z linearnim poljem pri 0,24 mm stati"cne ekscentri"cnosti v smeri x}{./LIN/lin_xs_sincos}
\newpage
Napaka $\varepsilon$  je prikazana na sliki \ref{./LIN/lin_xs_napaka}.

\slikaeps{Napaka $\varepsilon$ pri simulacijah z linearnim poljem pri 0,24 mm stati"cne ekscentri"cnosti v smeri x}{./LIN/lin_xs_napaka}

Napako razvijmo v Fourierovo vrsto in pridobimo amplitude posameznih harmonikov napake(Slika \ref{./LIN/lin_xs_fft}).
\slikaeps{Amplitude harmonikov napake $\varepsilon$ razvite v Fourierovo vrsto pri simulacijah z linearnim poljem pri 0,24 mm stati"cne ekscentri"cnosti v smeri x}{./LIN/lin_xs_fft}

Po pri"cakovanjih najbolj izstopata enosmerna komponenta (harmonik 0) in drugi harmonik. Na sliki \ref{./LIN/lin_xs_potek} vidimo odvisnost amplitud od spreminjanja ekscentri"cnosti.

\slikaeps{Potek amplitud posameznega harmonika napake $\varepsilon$ od stati"cne ekscentri"cnosti v smeri x}{./LIN/lin_xs_potek}

Poteke s slike \ref{./LIN/lin_xs_potek} predstavimo enako kot (\ref{vrsta:xs}).

\begin{eqnarray}
&C_0 =3,35\cdot 10^{-1}\Delta x_s^{3}-2,48\Delta x_s^{2}+1,19\cdot 10\Delta x_s+1,23\cdot 10^{-5} \\
&C_1 =5,56\cdot 10^{-4}\Delta x_s^{3}-2,00\cdot 10^{-3}\Delta x_s^{2}+4,34\cdot 10^{-3}\Delta x_s+7,67\cdot 10^{-8} \\
&C_2 =4,13\cdot 10^{-1}\Delta x_s^{3}-3,53\Delta x_s^{2}+1,69\cdot 10\Delta x_s-2,31\cdot 10^{-5} \\
&C_3 =-2,17\cdot 10^{-4}\Delta x_s^{3}+2,57\cdot 10^{-4}\Delta x_s^{2}+4,20\cdot 10^{-3}\Delta x_s+4,51\cdot 10^{-8} \\
&C_4 =-8,27\cdot 10^{-1}\Delta x_s^{3}+2,42\Delta x_s^{2}+8,08\cdot 10^{-3}\Delta x_s-1,60\cdot 10^{-4}
\end{eqnarray}




%
%\subsection{Sin\_cos}
%\subsection{napaka}
%\subsection{fft\_napake}
%\section{XS}
%\subsection{Sin\_cos}
%\subsection{napaka}
%\subsection{fft\_napake}
%\subsection{visanje\_napake}
%nastavek = 0.3352 xs^3-2.4826 xs^2+11.9361 xs+1.0156e-5
%H0	[1,01555901858575e-05;11,9361321852939;-2,48260332886115;0,335183727157872]
%H1	[2,34507105720499e-14;1,58051177708914e-13;-6,51993471506150e-13;9,46713162781764e-13]
%H2	[-1,86535135849702e-05;16,8819884980667;-3,52835647417836;0,412029175485044]
%H3	[1,30048463315339e-14;-2,42620430114880e-14;3,51682497683473e-13;-6,28770756738832e-13]
%
%


\section{Simulacija stati"cne ekscentri"cnosti v smeri y-osi}

Oglejmo si "se rezultate simulacij stati"cne ekscentri"cnosti v smeri y. Pričakujem podobne rezultate kot pri statični ekscentričnosti v smeri x, le da bo tu hitreje naračšala amplituda $sin$ signala.
 
\slikaeps{Signala $sin$ in $cos$ pri simulacijah z linearnim poljem pri 0,24 mm stati"cne ekscentri"cnosti v smeri y}{./LIN/lin_ys_sincos}

Napaka je prikazana na sliki \ref{./LIN/lin_ys_napaka}.
\slikaeps{Napaka $\varepsilon$ pri simulacijah z linearnim poljem pri 0,24 mm stati"cne ekscentri"cnosti v smeri y}{./LIN/lin_ys_napaka}

Razvijmo jo v Fourierovo vrsto in pridobimo amplitude posameznih harmonikov napake(Slika \ref{./LIN/lin_ys_fft}).
\slikaeps{Amplitude harmonikov napake $\varepsilon$ razvite v Fourierovo vrsto pri simulacijah z linearnim poljem pri 0,24 mm stati"cne ekscentri"cnosti v smeri y}{./LIN/lin_ys_fft}

Tudi tu najbolj izstopata enosmerna komponenta in drugi harmonik. Za razliko od stat"cne ekscentri"cnosti v smeri x je tu enosmerna komponenta negativna.



Na sliki \ref{./LIN/lin_ys_potek} vidimo odvisnost amplitud od spreminjanja stati"cne ekscentri"cnosti v smeri y.

\slikaeps{Potek amplitud posameznega harmonika napake $\varepsilon$ od stati"cne ekscentri"cnosti v smeri y}{./LIN/lin_ys_potek}

Poteke s slike \ref{./LIN/lin_ys_potek} predstavimo s polinomom tretje stopnje.

\begin{eqnarray}
&C_0 =3,35\cdot 10^{-1}\Delta y_s^{3}-2,48\Delta y_s^{2}+1,19\cdot 10\Delta y_s+1,22\cdot 10^{-5} \\                   
&C_1 =1,09\cdot 10^{-4}\Delta y_s^{3}-8,69\cdot 10^{-4}\Delta y_s^{2}+4,34\cdot 10^{-3}\Delta y_s+7,62\cdot 10^{-10} \\
&C_2 =4,12\cdot 10^{-1}\Delta y_s^{3}-3,53\Delta y_s^{2}+1,69\cdot 10\Delta y_s-2,31\cdot 10^{-5} \\                   
&C_3 =2,43\cdot 10^{-4}\Delta y_s^{3}-1,30\cdot 10^{-3}\Delta y_s^{2}+4,20\cdot 10^{-3}\Delta y_s+1,83\cdot 10^{-8} \\ 
&C_4 =-8,26\cdot 10^{-1}\Delta y_s^{3}+2,42\Delta y_s^{2}+6,13\cdot 10^{-3}\Delta y_s-1,60\cdot 10^{-4} \\               
\end{eqnarray}




%
%
%
%\section{YS}
%\subsection{Sin\_cos}
%\subsection{napaka}
%\subsection{fft\_napake}
%\subsection{visanje\_napake}
%
%H0	[-1,01555899896284e-05;-11,9361321852922;2,48260332885533;-0,335183727150766]
%H1	[2,04074110999664e-14;3,79449756434155e-13;-2,10317924735210e-12;2,95771540643260e-12]
%H2	[-1,86535135760308e-05;16,8819884980664;-3,52835647417692;0,412029175482732]
%H3	[1,26502083464184e-14;1,05725909044143e-14;-1,64647766137320e-14;1,89710785932321e-14]
%
%\section{ZS}
%ni nic ker je atan(k/k)
%\section{Xd}
%\subsection{Sin\_cos}
%\subsection{napaka}
%\subsection{fft\_napake}
%\subsection{visanje\_napake}    



\section{Dinami"cna ekscentri"cnost}

Oglejmo si sedaj rezultate simulacij dinami"cne ekscentri"cnosti. V signalih $sin$ in $cos$ se pojavi enosmerna komponenta (Slika \ref{./LIN/lin_xd_sincos}).
\slikaeps{Signala $sin$ in $cos$ pri simulacijah z linearnim poljem pri 0,24 mm dinami"cne ekscentri"cnosti v smeri x}{./LIN/lin_xd_sincos}

\slikaeps{Napaka $\varepsilon$ pri simulacijah z linearnim poljem pri 0,24 mm dinami"cne ekscentri"cnosti v smeri y}{./LIN/lin_xd_napaka}

V napaki prevladuje prvi harmonik kar je vidno tudi iz razvoja v Fourierovo vrsto (Slika \ref{./LIN/lin_xd_fft})

\slikaeps{Amplitude harmonikov napake $\varepsilon$ razvite v Fourierovo vrsto pri simulacijah z linearnim poljem pri 0,24 mm dinami"cne ekscentri"cnosti v smeri x}{./LIN/lin_xd_fft}



Na sliki \ref{./LIN/lin_xd_potek} vidimo odvisnost amplitud od spreminjanja ekscentri"cnosti.

\slikaeps{Potek amplitud posameznega harmonika napake $\varepsilon$ od dinami"cne ekscentri"cnosti v smeri x}{./LIN/lin_xd_potek}

Poteke harmonikov s slike \ref{./LIN/lin_xd_potek} aproksimiramo  s polinomi. 

\begin{eqnarray}
&C_0 =2,64\cdot 10^{-4}\Delta x_d^{3}+1,25\cdot 10^{-3}\Delta x_d^{2}+2,91\cdot 10^{-3}\Delta x_d+1,02\cdot 10^{-7} \\
&C_1 =1,58\cdot 10^{-4}\Delta x_d^{3}+2,37\cdot 10^{-3}\Delta x_d^{2}+3,38\cdot 10\Delta x_d+2,28\cdot 10^{-7} \\     
&C_2 =1,06\cdot 10^{-3}\Delta x_d^{3}+9,95\Delta x_d^{2}-1,95\cdot 10^{-3}\Delta x_d+7,96\cdot 10^{-7} \\             
&C_3 =3,91\Delta x_d^{3}-1,41\cdot 10^{-3}\Delta x_d^{2}+9,91\cdot 10^{-4}\Delta x_d+1,06\cdot 10^{-5} \\             
&C_4 =1,73\Delta x_d^{3}-5,52\cdot 10^{-1}\Delta x_d^{2}+6,15\cdot 10^{-2}\Delta x_d-1,36\cdot 10^{-3} \\           
\end{eqnarray}

Dinami"cna ekscentri"cnost v smeri y, pri simulacijah z linearnim poljem ni izra"zala napake, saj se polje ob tej ekscentri"cnosti zaradi aproksimacije z ravnino, za Hall-ove sonde ni spremenilo.

V tem poglavju sem predstavil numeri"cne rezultate simulacij z magnetnim poljem aproksimiranega z ravnino. Rezultati so potrdili prevladujo"ce harmomnike, pojavljajo se tudi vi"sji, kar sem upošteval pri napaki izrazeni z neskončno vrsto. V nadaljevanju pri"cakujem, z bolj"sim modelom magnetnega polja, manj"se napake po amplitudi.


%H0	[1,10753993340810e-13;1,24706314026430e-12;-6,88797159092575e-12;1,01065478093830e-11]
%H1	[3,33808204809378e-14;33,7618618558909;2,34008673890266e-12;-2,18397874745013e-12]
%H2	[1,02998847934023e-14;-1,41595026512391e-13;9,94718394324374;-7,34898016894629e-14]
%H3	[3,30877367536122e-15;-2,68649378966597e-13;9,00568019414565e-13;3,90762289998666]
%
%\section{YD}
%ni nic ker ni odvisno od njega


\chapter{Realni model magnetnega polja}

S poznavanjem točnejše funkcije polja, je točnejše predvidevanje potekov realnih $sin$ $cos$ in napake. Podjetje RLS,  je posredovalo rezultate z-komponente gostote magnetnega polja 2,55 mm nad simuliranim magnetom.  Definicijsko območje  je 20x20 mm s korakom 0,02 mm (Slika \ref{Realno_polje}).
\slikaeps{Model z- komponente gostote magnetnega polja uporabljen v simulacijah}{Realno_polje}
Polje, ki ga pomeri Hallova sonda v poljubni točki, je bilo aproksimirano s poljem, definiranim v geometrijsko najbližji točki definicijskega območja.%Polje  Geometrijsko sem poiskal najbližjo poznano točko v kateri imam simulirano magnetno polje in vzel vrednost polja v najbližnji točki za polje v moji točki. S tem sem se izognil linearni interpolaciji polja (funkciji interp2) in s tem skrajšal simulacijski čas za 93\% (iz 11min na 43.37s).
V tem poglavju so predstavljeni rezultati simulacij ekscentričnosti z uporabo realnega modela magnetnega polja.

\section{Brez napake}
%Kljub idealni montaži, se zaradi nepopolnega magneta, pojavi napaka. Magnet je lahko neenakomerno magnatiziran, kar nam že v začetku ustvari neko napako.
Kljub idealni montaži, Hallovi sondi ne zajameti idealnih signalov. Signala nimata popolnoma enakih amplitudi osnovnega harmonika, idealnega faznega zamika, vsebujeta tudi enosmerne komponente. Vsebujeta tudi višje harmonike (Slika \ref{./rea/00_sincos}). To se izrazi na napaki (Slika \ref{./rea/00_napaka}). V napaki se pojavijo skoki ($105^{\circ}$ ), ki so posledica nepopolnega numerično izračunanega modela magnetnega polja. Napaka razvita v Fourierova vrsto prikaže vrednosti amplitud posameznih harmonikov napake (Slika \ref{./rea/00_fft}). Izrazit je četrti harmonik, ki je pričakovan po podatkovnih listih \cite{AM8192}.
\slikaeps{$sin$ in $cos$ pri simulacijah z realnim magnetnim poljem brez ekscentričnosti}{./rea/00_sincos}
\slikaeps{Napaka $\varepsilon$ pri simulacijah z realnim magnetnim poljem brez ekscentričnosti}{./rea/00_napaka}
\slikaeps{Amplitude harmonikov napake $\varepsilon$ razvite v Fourierovo vrsto pri simulacijah z realnim poljem brez ekscentričnosti}{./rea/00_fft}
\newpage
\section{Simulacija statične ekscentričnosti v smeri x-osi}
Po pričakovanjih se bo spremenila amplituda $sin$ in $cos$ signala ter zmanjšal njun fazni zamik (izraza (\ref{equ:Bx_stat}) in \ref{equ:By_stat}). Na sliki \ref{./rea/xs_sincos} ni opaziti razlik, v primerjavi s $sin$ in $cos$ brez vpliva ekscentričnosti. Na sliki \ref{./rea/xs_napaka} je prikazana napaka $\varepsilon$. Oblika je bila pričakovana \cite{AM8192}. Razvoj napake v Fourierovo vrsto (slika \ref{./rea/xs_fft}) prikaže enako velikost enosmerne komponente in nižjo amplituda drugega harmonika, kot pri simulacijah z linearnim magnetnim poljem.
\slikaeps{$sin$ in $cos$ pri simulacijah z realnim poljem pri 0,24 mm statične ekscentričnosti v smeri x}{./rea/xs_sincos}
\slikaeps{Napaka $\varepsilon$ pri simulacijah z realnim poljem pri 0,24 mm statične ekscentričnosti v smeri x}{./rea/xs_napaka}
\slikaeps{Amplitude harmonikov napake $\varepsilon$ razvite v Fourierovo vrsto pri simulacijah z realnim poljem pri 0,24 mm statične ekscentričnosti v smeri x}{./rea/xs_fft}
\newpage
\subsection{Sprememba $sin$, $cos$ ter napake od $\Delta x_s$}
Na sliki \ref{./rea/xs_sincos_amp} je prikazana sprememba amplitude prvega harmonika signalov $sin$ in $cos$. Pričakovano je bilo hitrejše spreminjanje amplitude signala $cos$. Amplituda z višanjem ekscentričnosti pada, kar je razumljivo. Senzor je predviden za uporabo priporočenega magneta s premerom 4mm. S pravilno postavitvijo sond, je v najboljši legi pomerjeno polje z najvišjo amplitudo. Z ekscentričnostjo Hallova sonda pomeri polje z nižjo maksimalno vrednostjo. Na sliki \ref{./rea/xs_sincos_off} je prikazan potek enosmerne komponente $sin$ in $cos$. Po rezultatih simulacij ni bilo pričakovano spreminjanje enosmerne komponente v $cos$. Zanimivo je tudi, naraščanje enosmerne komponente $cos$ signala pri ekscentričnostih višjih od 0,45 mm. Slika \ref{./rea/xs_sincos_phase} prikazuje potek faznih signalov. Rezultat je bil pričakovan.%Sedaj poglejmo kako se spreminjata analogna signala $sin$ in $cos$ ob spreminjanju ekscentričnosti. Na sliki \ref{./rea/xs_sincos_amp} je prikazana sprememba amplitude prvega harmonika, na sliki \ref{./rea/xs_sincos_off} enosmerni komponenti in na sliki \ref{./rea/xs_sincos_phase} fazni zamik signalov glede na njuno idealno poravnavo. Pri simulacijah z linearnim poljem sta amplitudi prvega harmonika naraščali. Hallovi sondi sta na radij 2,4 mm postavljeni z razlogom, imeti maksimalno amplitudo signala. Z vsakim premikom se ampliuda lahko le zmanjša, pri čemer se amplituda $cos$ manjša hitreje. To lahko razumemo, saj Hallova sonda za zajem $cos$ signala zajame večji radij kot sonda za signal $sin$. Sonda ne zajame več najvišje vrednsoti magnetnega polja zato  se mu amplituda tudi zmanjša. Zanimivo je tudi, da imati v idealni legi ob signala $sin$ in $cos$ enako enosmerno komponento. Z višanjem ekscentričnosti se enosmerna komponenta $cos$ manjša, enosmerna komponenta $sin$ ostaja enaka. Fazni zamik signala $cos$ je dokaj konstanten, medtem ko fazni zamik $sin$ linearno narašča kot je bilo simulirano pri linearnem polju.%\ref{./rea/xs_sincos_amp}. Amplituda $cos$ se spreminja hitreje, kot amplituda $sin$ .
\slikaeps{Amplituda osnovnega harmonika  $sin$ in $cos$ pri simulacijah z realnim poljem statične ekscentričnosti v smeri x}{./rea/xs_sincos_amp}
\slikaeps{Enosmerna komponenta $sin$ in $cos$ pri simulacijah z realnim poljem statične ekscentričnosti v smeri x}{./rea/xs_sincos_off}
\slikaeps{Fazni zamik $sin$ in $cos$ pri simulacijah z realnim poljem statične ekscentričnosti v smeri x glede na idealna signala $sin$ in $cos$}{./rea/xs_sincos_phase}
Poteke se aproksimira s kubičnimi polinomi. Enačbe potrdijo konstantnost amplitude prvega harmonika in enosmerne komponente signala $sin$, ter linearno naraščanje faznega zamika.

Na sliki \ref{./rea/xs_potek} so prikazani poteki amplitud posameznega harmonika ob spreminjanja statične ekscentričnosti v smeri x. Enosmerna komponenta in amplituda drugega harmonika naraščata linearno, ostali harmoniki ohranjajo konstantno amplitudo (\ref{real_xs_C0})-(\ref{real_xs_C4}).
\begin{eqnarray}
&A_{sin} = +4,18\cdot 10^{-2}\Delta x_s^3-6,17\cdot 10^{-2}\Delta x_s^2-3,60\cdot 10^{-3}\Delta x_s+39,9\\     
&Off_{sin} = -0,545\Delta x_s^3+0,343\Delta x_s^2-5,33\cdot 10^{-2}\Delta x_s+0,125\\   
&\delta_{sin} = -2,29\Delta x_s^3+0,365\Delta x_s^2+23,80\Delta x_s-0,125\\
&A_{cos} = -2,39\Delta x_s^3-3,28\Delta x_s^2-0,966\Delta x_s+39,9\\     
&Off_{cos} = +0,868\Delta x_s^3-0,423\Delta x_s^2-0,316\Delta x_s+0,131\\   
&\delta_{cos} = -2,71\Delta x_s^3+3,54\Delta x_s^2-0,597\Delta x_s-0,146
\end{eqnarray}
\slikaeps{Potek amplitud posameznega harmonika napake $\varepsilon$ od statične ekscentričnosti v smeri x pri simulacijah z realnim poljem}{./rea/xs_potek}
\begin{eqnarray}\label{real_xs_C0}
&C_0 =-1,30\Delta x_s^{3}+1,66\Delta x_s^{2}+1,16\cdot 10\Delta x_s-1,37\cdot 10^{-1} \\                          
&C_1 =-5,99\Delta x_s^{3}+3,85\Delta x_s^{2}-6,20\cdot 10^{-1}\Delta x_s+2,05\cdot 10^{-1} \\                     
&C_2 =-3,28\cdot 10^{-1}\Delta x_s^{3}-5,20\cdot 10^{-2}\Delta x_s^{2}+12,0\Delta x_s+1,66\cdot 10^{-2} \\
&C_3 =-1,84\Delta x_s^{3}+1,50\Delta x_s^{2}-4,91\cdot 10^{-1}\Delta x_s+1,16\cdot 10^{-1} \\
\label{real_xs_C4}                     
&C_4 =8,53\Delta x_s^{3}-3,17\Delta x_s^{2}-4,72\cdot 10^{-1}\Delta x_s+3,20\cdot 10^{-1}
\end{eqnarray}

Enosmerna komponenta narašča enako kot pri simulacijah z linearnim poljem. Drugi harmonik narašča nekoliko počasneje, kot je naraščal pri simulacijah z linearnim poljem. 

\section{Simulacija statične ekscentričnosti v smeri y-osi}

Tako kot pri statični ekscentričnosti v smeri x, se tudi na $sin$ in $cos$ signalih ob povzročeni ekscentričnosti ne opazi vidne razlike (slika \ref{./rea/ys_sincos}). Napaka $\varepsilon$ (slika \ref{./rea/ys_napaka}) je enake oblike kot je bila pri simulacijah statične ekscentričnosti v smeri x (slika \ref{./rea/xs_napaka}). Napaka ima le negativno enosmerno komponento. Razvoj napake v Fourierovo vrsto (slika \ref{./rea/xs_fft}) potrdi pričakovanja.%Oglejmo si rezultate simulacij statične ekscentričnosti v smeri y. Pričakujem podobne rezultate kot pri statični ekscentričnosti v smeri x, le da bo tu hitreje padala amplituda $sin$ signala in spreminjal se bo fazni zamik $cos$. Pri izmiku za 10\% neizgleda, da bi se siganal kaj spremenila vendar predvidevam, da bo fft signalov nakazal na padanje. Na sliki \ref{./rea/ys_napaka} vidimo obliko napake, kot smo jo pričakovali.
\slikaeps{$sin$ in $cos$ pri simulacijah z realnim poljem pri 0,24 mm statične ekscentričnosti v smeri y}{./rea/ys_sincos}
\slikaeps{Napaka $\varepsilon$ pri simulacijah z realnim poljem pri 0,24 mm statične ekscentričnosti v smeri y}{./rea/ys_napaka}
\slikaeps{Amplitude harmonikov napake $\varepsilon$ razvite v Fourierovo vrsto pri simulacijah z realnim poljem pri 0,24 mm statične ekscentričnosti v smeri y}{./rea/ys_fft}
\newpage
\subsection{Sprememba $sin$, $cos$ ter napake od $\Delta y_s$}
Potek amplitude osnovnega harmonika $sin$ (slika \ref{./rea/ys_sincos_amp}) se spreminja kot se je spreminjala amplituda osnovnega hamronika $cos$ pri simulacijah statične ekscentričnosti v smeri x. Sprememba $cos$ od statične ekscentričnosti v smeri y je proti spremembi $sin$ zanemarljiva. Enosmerena komponenta (slika \ref{./rea/ys_sincos_off}) pri $sin$ se spreminja enako, kot enosmerna komponenta $cos$ pri statični ekscentričnosti v smeri x (slika \ref{./rea/xs_sincos_off}).  Fazni zamik signalov se spreminja po pričakovanjih (slika \ref{./rea/ys_sincos_phase}). Fazni zamik se manjša, pri čemer pada fazni zamik $cos$ signala.%Oglejmo si sedaj poteke amplitude, enosmerne komponente in faznega zamika pri statični ekscentričnosti v smeri y. Poteki so podobni kot pri ekscentričnosti v semeri x le kar je veljalo prej za $sin$ bo sedaj za $cos$ in obratno. Na sliki \ref{./rea/ys_sincos_amp} vidimo sedaj pričakovano padanje amplitude prvega harmonika $sin$. Kot je pri ekscentričnosti v smeri x padala enosmerna komponenta $cos$, sedaj pada enosmerna komponenta $sin$. Fazni zamik $cos$ je enak kot je bil pri simulacijah z linearnim poljem. POteke sedaj opišimo še s kubičnimi polinomi.
\slikaeps{Amplituda osnovnega harmonika  $sin$ in $cos$ pri simulacijah z realnim poljem statične ekscentričnosti v smeri y}{./rea/ys_sincos_amp}
\slikaeps{Enosmerna komponenta $sin$ in $cos$ pri simulacijah z realnim poljem statične ekscentričnosti v smeri y}{./rea/ys_sincos_off}
\slikaeps{Fazni zamik $sin$ in $cos$ pri simulacijah z realnim poljem statične ekscentričnosti v smeri y glede na idealna signala $sin$ in $cos$}{./rea/ys_sincos_phase}
Poteki zapisani s kubičnimi polinomi.
\begin{eqnarray}
&A_{sin} = -2,39\Delta y_s^3-3,28\Delta y_s^2-0,966\Delta y_s+39,9\\     
&Off_{sin} = +0,868\Delta y_s^3-0,423\Delta y_s^2-0,316\Delta y_s+0,131\\   
&\delta_{sin} = -2,71\Delta y_s^3+3,54\Delta y_s^2-5,97\cdot 10^{-1}\Delta y_s-1,46\cdot 10^{-1}\\
&A_{cos} = +3,76\cdot 10^{-2}\Delta y_s^3-5,99\cdot 10^{-2}\Delta y_s^2-3,87\cdot 10^{-3}\Delta y_s+39,9\\     
&Off_{cos} = -0,545\Delta y_s^3+0,342\Delta y_s^2-5,30\cdot 10^{-2}\Delta y_s+0,124\\   
&\delta_{cos} = +0,229\Delta y_s^3+0,473\Delta y_s^2-24,0\Delta y_s-0,124  
\end{eqnarray}
Enačbe prikazujejo podobne poteke kot poteki pri statični ekscentričnosti v smeri x. Poteki, ki so veljali za $sin$ tu veljajo za $cos$ in obratno. Razlikuje se le pri predzanku faznega zamika $\varphi_{cos}$. 

Posledično to vpliva na posamezne harmonike napake. Po pričakovanju je enosmerna komponenta negativana, drugi harmonik narašča počasneje kot je pri simulacijah z linearnim poljem, kar je pričakovano. Poteki aproksimirani s kubičnimi polinomi so podobni aprksimacijam amplitud posameznih harmonikov napake statične ekscentričnosti v smeri x.
\slikaeps{Potek amplitud posameznega harmonika napake $\varepsilon$ od statične ekscentričnosti v smeri y pri simulacijah z realnim poljem}{./rea/ys_potek}%Potek enosmerne komponente ob majhnih odmikih linearno narašča, enako kot pri pri ekscentričnosti v smeri x, le z negaticvnim predznakom. Drugi harmonik narašča z večanjem ekscentričnosti prav tako, kot narašča amplituda drugega harmonika ob večanju ekscentričnosti v smeri x.
\begin{eqnarray}
&C_0 =-2,56\Delta y_s^{3}+2,36\Delta y_s^{2}-1,24\cdot 10\Delta y_s-1,33\cdot 10^{-1} \\     
&C_1 =-2,46\Delta y_s^{3}+3,57\Delta y_s^{2}-1,19\Delta y_s+2,14\cdot 10^{-1} \\             
&C_2 =2,92\Delta y_s^{3}-1,53\Delta y_s^{2}+1,23\cdot 10\Delta y_s-2,78\cdot 10^{-2} \\      
&C_3 =-2,93\Delta y_s^{3}+2,15\Delta y_s^{2}-4,19\cdot 10^{-1}\Delta y_s+1,07\cdot 10^{-1} \\
&C_4 =8,63\Delta y_s^{3}-2,82\Delta y_s^{2}-7,73\cdot 10^{-1}\Delta y_s+3,33\cdot 10^{-1}          
\end{eqnarray}
\section{Dinamična ekscentričnost v smeri x}
Vpliv dinamične ekscentričnosti v $sin$ in $cos$ bo viden v enosmerni komponenti. Na sliki  \ref{./rea/xd_sincos} sprememba ni opazna, posledica enosmerne komponente v $sin$ in $cos$  je vidna v napaki (Slika \ref{./rea/xd_napaka}). Napaka se izrazi v obliki prvega harmonika, ki je posledica enosmerne komponente. V napaki je viden tudi tretji harmonik saj enosmerna komponenta vpliva tudi nanj (\ref{vrsta_sinoff}). Razvoj napake v Fourierovo vrsto potrdi pričakovanja (Slika \ref{./rea/xd_fft}). Poglejmo si še fft napake s slike \ref{./rea/xd_napaka}, prikazanega na sliki \ref{./rea/xd_fft}.
\slikaeps{$sin$ in $cos$ pri simulacijah z realnim poljem pri 0,24 mm dinamične ekscentričnosti v smeri x}{./rea/xd_sincos}
\slikaeps{Napaka $\varepsilon$ pri simulacijah z realnim poljem pri 0,24 mm dinamične ekscentričnosti v smeri x}{./rea/xd_napaka}
\slikaeps{Amplitude harmonikov napake $\varepsilon$ razvite v Fourierovo vrsto pri simulacijah z realnim poljem pri 0,24 mm dinamične ekscentričnosti v smeri x}{./rea/xd_fft}
\newpage
\subsection{Sprememba $sin$, $cos$ ter napake od $\Delta x_d$}
Spremembe amplitude osnovnega harmonika pri $sin$ in $cos$ po pričakovanjih iz rezultatov statične ekscentričnosti simulacij z realnim poljem pada. Zanimivo je, enako spreminjanje amplitude osnovnega harmonika (slika \ref{./rea/xd_sincos_amp}). Enako se spreminjata tudi enosmerni komponenti signalov (slika \ref{./rea/xd_sincos_off}). Fazna razlika signalov ostaja konstantna, vendar je opazno lezenje obeh signalov in posledično naraščanje enosmerne komponente napake.
\slikaeps{Amplituda osnovnega harmonika  $sin$ in $cos$ pri simulacijah z realnim poljem dinamične ekscentričnosti v smeri x}{./rea/xd_sincos_amp}
\slikaeps{Enosmerna komponenta $sin$ in $cos$ pri simulacijah z realnim poljem dinamične ekscentričnosti v smeri x}{./rea/xd_sincos_off}
\slikaeps{Fazni zamik $sin$ in $cos$ pri simulacijah z realnim poljem dinamične ekscentričnosti v smeri x glede na idealna signala $sin$ in $cos$}{./rea/xd_sincos_phase}
Poteki zapisani s kubičnimi polinomi predstavijo enako spreminjanje signala $sin$ in $cos$.
\begin{eqnarray}
&A_{sin} = -6,54\Delta x_d^3-1,78\Delta x_d^2-1,04\Delta x_d+39,9\\     
&Off_{sin} = 2,20\Delta x_d^3-1,11\Delta x_d^2-8,45\Delta x_d+1,28\cdot 10^{-1}\\   
&\delta_{sin} = -4,82\Delta x_d^3+4,73\Delta x_d^2-8,49\cdot 10^{-1}\Delta x_d-1,14\cdot 10^{-1}\\
&A_{cos} = -6,54\Delta x_d^3-1,78\Delta x_d^2-1,04\Delta x_d+39,9\\     
&Off_{cos} = 2,20\Delta x_d^3-1,11\Delta x_d^2-8,45\Delta x_d+1,28\cdot 10^{-1}\\   
&\delta_{cos} = -4,82\Delta x_d^3+4,73\Delta x_d^2-8,49\cdot 10^{-1}\Delta x_d-1,14\cdot 10^{-1}
\end{eqnarray}
Potek posameznih harmonikov napake je viden na sliki \ref{./rea/xd_potek}. Po pričakovanjih najhitreje narašča prvi harmonik napake, sledi mu treji. Ostali harmoniki so zanemarljivi. Poteki so aproksimirani tudi s kubičnimi polinomi.
\slikaeps{Potek amplitud posameznega harmonika napake $\varepsilon$ od dinamične ekscentričnosti v smeri x pri simulacijah z realnim poljem}{./rea/xd_potek}
\begin{eqnarray}
&C_0 =-5,61\Delta x_d^{3}+5,24\Delta x_d^{2}-9,00\cdot 10^{-1}\Delta x_d-1,14\cdot 10^{-1} \\
&C_1 =-2,27\Delta x_d^{3}+3,60\Delta x_d^{2}+2,44\cdot 10\Delta x_d-8,53\cdot 10^{-2} \\     
&C_2 =-1,71\Delta x_d^{3}+2,37\Delta x_d^{2}-3,36\cdot 10^{-1}\Delta x_d+9,84\cdot 10^{-3} \\
&C_3 =1,07\Delta x_d^{3}-1,37\Delta x_d^{2}+8,73\Delta x_d+8,63\cdot 10^{-2} \\              
&C_4 =6,38\Delta x_d^{3}+4,03\Delta x_d^{2}-2,02\Delta x_d+3,51\cdot 10^{-1}       
\end{eqnarray}

\section{Dinamična ekscentričnost v smeri y}

V simulacijah z linearnim poljem napaka ni bila odvisna od dinamične ekscentričnosti v smeri y. Kljub temu je bila opravljena simulacija. Rezultati so razlikujejo od pričakovanj. Spremembe v $sin$ in $cos$ ni opaziti (slika \ref{./rea/yd_sincos}), vendar v napaki se pojavi prvi in tretji harmonik (slika \ref{./rea/yd_sincos}). Razvoj v Fourierovo vrsto potrdi izstopanje omenjeinih harmonikov.
\slikaeps{$sin$ in $cos$ pri simulacijah z realnim poljem pri 0,24 mm dinamične ekscentričnosti v smeri y}{./rea/yd_sincos}
\slikaeps{Napaka $\varepsilon$ pri simulacijah z realnim poljem pri 0,24 mm dinamične ekscentričnosti v smeri y}{./rea/yd_napaka}
\slikaeps{Amplitude harmonikov napake $\varepsilon$ razvite v Fourierovo vrsto pri simulacijah z realnim poljem pri 0,24 mm dinamične ekscentričnosti v smeri y}{./rea/yd_fft}
\newpage
\subsection{Sprememba $sin$, $cos$ ter napake od $\Delta y_d$}
Sprememba amplitude osnovnega harmonika od naraščanja ekscentričnosti pada (slika \ref{./rea/yd_sincos_amp}). Razlika amplitud ostaja nespremenjena. Enosmerna komponenta (slika \ref{./rea/yd_sincos_off}) se spreminja minimalno, komponenti obeh signalov sta enaki. Vidno je tudi sofazno lezenje faznih zamikov obeh signalov (slika \ref{./rea/yd_sincos_phase}). Poteki so aproksimirani s kubičnimi polinomi in potrdijo enako spreminjnanje.
\slikaeps{Amplituda osnovnega harmonika  $sin$ in $cos$ pri simulacijah z realnim poljem dinamične ekscentričnosti v smeri y}{./rea/yd_sincos_amp}
\slikaeps{Enosmerna komponenta $sin$ in $cos$ pri simulacijah z realnim poljem dinamične ekscentričnosti v smeri y}{./rea/yd_sincos_off}
\slikaeps{Fazni zamik $sin$ in $cos$ pri simulacijah z realnim poljem dinamične ekscentričnosti v smeri y glede na idealna signala $sin$ in $cos$}{./rea/yd_sincos_phase}
\begin{eqnarray}
&A_{sin} = +1,15\Delta y_d^3-2,72\Delta y_d^2-3,47\cdot 10^{-1}\Delta y_d+3,99\cdot 10\\     
&Off_{sin} = -0,244\Delta y_d^3-0,292\Delta y_d^2+0,169\Delta y_d+0,131\\   
&\delta_{sin} = +2,39\Delta y_d^3-2,10\Delta y_d^2+9,01\cdot 10^{-1}\Delta y_d-1,47\cdot 10^{-1}\\
&A_{cos} = +1,15\Delta y_d^3-2,72\Delta y_d^2-3,47\cdot 10^{-1}\Delta y_d+3,99\cdot 10\\     
&Off_{cos} = -0,244\Delta y_d^3-0,292\Delta y_d^2+0,169\Delta y_d+0,131\\   
&\delta_{cos} = +2,39\Delta y_d^3-2,10\Delta y_d^2+9,01\cdot 10^{-1}\Delta y_d-1,47\cdot 10^{-1} 
\end{eqnarray}
Na sliki \ref{./rea/yd_potek} je prikazana odvisnost amplitud napake ob spreminjanju dinamične ekscentričnosti v smeri y. Napaka, se po pričakovanjih najbolj izrazi s prvim in tretjim harmonikom. Oblika napake ni posledica spremembe amplitude osnovnega harmonika, enosmerne komponente ali spremembe faznega zamika v $sin$ in $cos$. Naraščanje prvega in tretjega harmonika je posledica vpliva drugega harmonika, ki se pojavi v $sin$ in $cos$. Drugi harmonik v $sin$ in $cos$ se pojavi zaradi magnetnega polja, kar v tem delu ni raziskano zakaj.
\slikaeps{Potek amplitud posameznega harmonika napake $\varepsilon$ od dinamične ekscentričnosti v smeri y pri simulacijah z realnim poljem}{./rea/yd_potek}
\begin{eqnarray}
&C_0 =2,50\Delta y_d^{3}-2,14\Delta y_d^{2}+8,63\cdot 10^{-1}\Delta y_d-1,47\cdot 10^{-1}\\                           
&C_1 =-9,46\Delta y_d^{3}+7,85\Delta y_d^{2}+6,81\Delta y_d+8,35\cdot 10^{-2} \\                                       
&C_2 =-0,148\Delta y_d^{3}+0,762\Delta y_d^{2}-3,01\cdot 10^{-2}\Delta y_d+4,54\cdot 10^{-4} \\
&C_3 =-6,17\Delta y_d^{3}+4,40\Delta y_d^{2}+7,91\Delta y_d-3,84\cdot 10^{-2} \\                                       
&C_4 =5,60\Delta y_d^{3}-1,89\Delta y_d^{2}-2,84\cdot 10^{-1}\Delta y_d+3,13\cdot 10^{-1}       
\end{eqnarray}

V tem poglavju so bile prikazane simulacije z uporabo realnega polja, ki ga merijo Hall-ove sonde. Rezultati imajo manjšo napako kot pri simulacijah z aproksimiranim linearnim magnetnim poljem. Opaziti je bilo manjši fazni zamik obeh signalov $sin$ in $cos$ pri dinamični ekscentričnosti, kar bi bilo smiselno pri meritvah podbrobno opazovati. Na koncu, pri dinamični ekscentričnosti v smeri y je prikazano tudi, da se v zajetem polju pojavijo tudi višji harmoniki, ki še dodatno ustvarijo napako.
\chapter{Meritve}

Simulacije so prikazale okvirne poteke analognih signalov ter napake  ob posamezni ekscentri"cnosti.
Na merilni napravi so bile opravljene meritve ekscentričnosti. V tem poglavju je opisana sama merilna naprava, zajem podatkov ter izvedba meritev.

\section{Oprema in postavitev merilnega mesta}

Merilno mesto sestavljala krmilna plošča za regulacijo motorskega pogona in obdelavo signalov sestavljena v LRTME.
Vsebuje elektromotorski pogon z inkrementalnim, referen"cnim dajalnikom zasuka TONiC podjetja Renishaw in magnetnim aktuatorjem za RM44 podjetja RLS  d.o.o.
Magnetni aktuator je možno premikati le v eni prostorski osi (slika \ref{premikanjeMagneta.png}).
Senzor RM44 je pritrjen na konstrukcijo 6 osnega mikrometrskega nastavljalnika pozicije HTIMS601.
Celotno merilno mesto je prikazano na sliki \ref{postavitevmerilnegamesta.png}.

%\bitnaslika{Dinamično ekscentričnost se lahko izmeri le v eni smeri}{premikanjeMagneta.png}

%\bitnaslika{Postavitev testnega mesta}{postavitevmerilnegamesta.png}%{0.6\textwidth}{0.877\textwidth}% 1.46 je razmerje visina sirina

Za manevriranje s HTIMS601 je potrebno nastaviti 6 osi.
Vsako os se nastavlja s enim od vijakov (\ref{HTIMS601.png}).
S postavitvijo koordinatnega sistema, je vsak od vijakov definiral premik senzorja. 
S spremembo vrtenja vijakov translacijskih osi, se je lokacija senzorja pred magnetom spreminjala z enako spremembo. S spremembo vrtenja rotacijskih vijakov, se je zaradi ročice na katero je pritrjen senzor, senzor zarotiral in hkrati tudi premaknil iz dotedanje lege. S spremembo rotacije je potrebno popraviti tudi nastavitve vijakovm, ki senzor premikajo v translacijskih oseh.
Na sliki \ref{HTIMS601.png} je prikazano kateri vijak je za nastavljanje posamezne prostorske osi. Vijaki poimenovani x-os, y-os in z-os so za nastavljanje translaciijo merjenca, rot x-osi, rot y-osi in rot z-osi so za nastavljanje rotacije premikajoče plošče na vrhu HTIMS601.
Senzor se je posledično ob premiku vijakov rot x-osi, rot y-osi in rot z-osi, glede na magnet vrtel in premikal.
S potenciometrom se nastavlja hitrost vrtenja motorja.
Hitrost vrtenja je nastavljena na 60 RPM (slika \ref{hitrost}). Hitrost ni popolnoma konstantna. Vzrok je v vztrajnosti pogona. Mitja Nemec je problem skušal čimbolje odpraviti, z dodajanjem primernih uteži na primerna mesta na vztrajniku.

Za senzor RM4 sem postavil koordinati sistem prikazan na sliki \ref{koordinatnisistem.png}
 
%\bitnaslika{Naprava za nastavljanje stati"cne ekscentri"cnosti}{HTIMS601.png}{0.7\textwidth}{1.023\textwidth}
%\slikaeps{Potek hitrosti od zasuka}{hitrost}
%\bitnaslika{Postavitev koordinatnega sistema}{koordinatnisistem.png}

\newpage
\section{Zajem podatkov}

Mitja Nemec je pripravil tudi grafični uporabniški vmesnik za prikazovanje meritev (slika \ref{GUI.png}).
Vmesnik lahko prikazuje potek refernečnega kota, analognih signalov sinus in kosinus senzorja RM44, izračunanega kota iz sin in cos signala, napako med izračunanim kotom senzorja in refernčnim dajalnikom, hitrost vrtenja ter tok prve faze motorskega pogona. Signaloma sin in cos se v programu prištejeti enosmerni komponenti, ki bi popravili izhodna signala.
Krmilna plošča zajema podatke s pogona s frekvenco 1kHz.
Referenčni inkrementalni dajalnik, se ob zagonu inicializira. V programu se podatek o kotu deli s 12595200. Definicijsko območje referenčnega kota se giblje med 0 in 1.
Signala sin in cos se na krmilni plošči ojačata in pretvorita z 12 bitnim AD pretvornikom. Izhodu AD pretvornika se odšteje 2048 in deli s 4096. Definicijsko območje sin in cos signala se gibljeta med $\pm0,5$.
Hitrost in napaka sta izračunana iz zajetih signalov.
Podatki so v obliki enega paketa poslani s krmilne plošče na 1 sekundo. Pri frekvenci vrtenja 1 Hz, grafični vmesnik prikaže en obrat.
Podatke se lahko izvozi v obliki .csv datoteke in nato poljubno obdela.

Na sliki  \ref{GUI.png} je prikazan sinusni signal prikazan kot da je zamaknjen za 180$\mathrm{^\circ}$. To je posledica pozitvne smeri vrtenja senzorja \cite{RM44}. Senzorju se lahko nastavi v katero smer narašča izhod. To sem rešil tako, da sem obrnil podatke. Popraviti je bilo potrebno tudi potek referenčnega dajalnika.
%\bitnaslika{Grafični vmesnik s potiki signalov}{GUI.png}
\section{Senzor v izhodiščni legi}
Senzor in magnet se lahko gibljeta, najprimernejša, izhodiščna lega, ni definirana. Z merilno urico Mitutoyo 543-391B se je dinamično ekscentričnost magneta nastavilo na najmanjšo. Oplet z merilno urico je bil pomerjen $\pm 3 qmathrm{\mu m_{pp}}$.
S prilagajanjem vijakov HTIMS601, opazovanjem signalov $sin$ in $cos$ ter napake sem nastavil senzor v lego kjer je bila statična ekscentričnost najmanjša. Najprimernejšo lego sem iskal glede na največjo vrednost amplitud in ortogonalnost  $sin$ in $cos$ signalov. Signala $sin$ in $cos$ sta morala ustrezati definicijskem območju zajema AD pretvornika.


NA sliki \ref{./MER/00_sincos} sta predstavljena signala $sin$ in $cos$. Enosmerni komponenti sta prisotni v obeh signaloih, posledično se izrazi v napaki prvi harmonik (slika \ref{./MER/00_napaka}).
V napaki se pojavi med $\mathrm{95^\circ}$ in $\mathrm{140^\circ}$ preskok napake. V signalih $sin$ in $cos$ se v tem območju pojavi fazni zamik, $sin$ pri $\mathrm{95^\circ}$ in nato $cos$ pri $\mathrm{160^\circ}$
Z razvojem napake v Fourierovo vrsto (slika \ref{./MER/00_fft}) vidimo velikosti posameznih amplitud napake.
Enosmerna komponenta je posledica sofaznih zamikov obeh signalov $sin$ in $cos$. 
Prvi harmonik je posledica ensmermih komponent $sin$ in $cos$. Z matematično obdelavo siganlov $sin$ in $cos$ sem enosmerni komponenti odstranil,vendar se prvi harmonik napake še vedno izrazi. Prvi harmonik napake je odvisen tudi od drugega harmonika v signalih $sin$ in $cos$.
Z odstranitvijo tudi drugega harmnonika iz signalov $sin$ in $cos$ je bil  prvi harmonik v napaki odstranjen.
Signala  $sin$ in $cos$ med obdelavo ne bosta popravljena. Spreminjanje signalov  $sin$ in $cos$ in napake se bo opazovalo glede na potek, ki je bil pomerjen v izhodiščni legi.
\slikaeps{Signala $sin$ in $cos$ pomerjena v izhodi"s"cni legi}{./MER/00_sincos}
\slikaeps{Napaka $\varepsilon$ pomerjena v izhodi"s"cni legi}{./MER/00_napaka}
\slikaeps{Amplitude harmonikov napake $\varepsilon$ razvite v Fourierovo vrsto pri meritvah v izhodiščni legi}{./MER/00_fft}

\subsection{Meritve v izhodišni legi}
V izhodiščni legi sem opravil več meritev. Osredotočil sem se na enosmerni komponenti  in amplitudi osnovnega harmonika signalov $sin$ in $cos$. 
Porazdelitev enosmerne komponente signala $sin$ in $cos$ je prikazana na sliki \ref{./MER/00_off}.
Srednja vrednost enosmerne komponente $sin$ je $-8,85 \cdot 10^{-4}$, standardna deviacija je $1,08\cdot 10^{-4}$.
Srednja vrednost enosmerne komponente $cos$ je $4,20 \cdot 10^{-3}$, standardna deviacija je $5,20\cdot 10^{-5}$.
\slikaeps{Porazdelitev meritev enosmerne komponente signalov $sin$ in $cos$}{./MER/00_off}
Porazdelitev amplitude osnovnega harmonika signala $sin$ in $cos$ je prikazana na sliki \ref{./MER/00_off}.
Srednja vrednost amplitude osnovnega harmonika $sin$ je $0,451$, standardna deviacija je $2,20\cdot 10^{-4}$.
Srednja vrednost amplitude osnovnega harmonika $cos$ je $0,449$, standardna deviacija je $1,95\cdot 10^{-4}$.
\slikaeps{Porazdelitev meritev amplitude osnovnega harmonika signalov $sin$ in $cos$}{./MER/00_amp}
\newpage
\section{Meritve statične ekscentričnosti v smeri x-osi}
Pri meritvi je pričakovati spremembo amplitud in faznih zamikov signalov $sin$ in $cos$. Na sliki \ref{./MER/xs_sincos} sta prikazana $sin$ in $cos$ signala. Na signalih, med 95 in 175$^\circ$ se pojavijo visoko harmonske motnje.
Napaka je prikazana na sliki \ref{./MER/xs_napaka}. Med 95 in 175$^\circ$ se signala $sin$ in $cos$ hkrati fazno zamakneta. Vzrok tega pojava nisem raziskoval. Napaka razvita v Fourierovo vrsto prikaže pričakovano povišanje drugega harmonika. Povišala sta se tudi amplitudi tretjega in četrtega harmonika.
\slikaeps{Signala $sin$ in $cos$ merjena pri 0,2 mm statične ekscentičnosti v smeri x}{./MER/xs_sincos}
\slikaeps{Napaka $\varepsilon$ merjena pri 0,2 mm statične ekscentičnosti v smeri x}{./MER/xs_napaka}
\slikaeps{Amplitude harmonikov napake $\varepsilon$ razvite v Fourierovo vrsto merjeno pri 0,2 mm statične ekscentičnosti v smeri x}{./MER/xs_fft}
\subsection{Sprememba $sin$, $cos$ ter napake od $\Delta x_s$}
Iz simulacij se pričakuje nižanje amplitude osnovnega harmonika $cos$ signala- Sprememba amplitud osnovnih harmonikov je prikazana na sliki \ref{./MER/xs_sincos_amp}. Amplituda signala $cos$ pada pričakovano, pada tudi ampituda signala $sin$. Signala nimata enake amplitude v izhodišču, kar je posledica nepopolne izhodiščne lege. Potek enosmerne komponente je prikazan na sliki \ref{./MER/xs_sincos_off}. Komopnenta $cos$ signala pada po kot je bilo predvideno v simulacijah. Fazni zamik signalov je prikazan na sliki \ref{./MER/xs_sincos_phase}. Pričakovano po simulacijah se fazni zamik med signaloma povečuje. Rezultati simulacij so prikazovali konstanten fazni kot signala $cos$ in spreminjanje le faznega kota $sin$. Pri meritvah se je fazni kot $cos$ zmanjševal, fazni kot $sin$ naraščal. Razlika med njima je majša, kot je bila posimulirana. Vsota faznih zamikov je konstantna, zato se enosmerna komponenta v napaki ne spreminja.
Poteke signalov $sin$ in $cos$, ter poteke amplitud posameznih harmonikov napake, se aproksimira s kubičnimi polinomi.
\slikaeps{Potek amplitude osnovnega harmonika $sin$ in $cos$ pri meritvah stati"cne ekscentri"cnosti v smeri x}{./MER/xs_sincos_amp}
\slikaeps{Potek enosmerne komponente $sin$ in $cos$ pri meritvah stati"cne ekscentri"cnosti v smeri x}{./MER/xs_sincos_off}
\slikaeps{Fazni zamik osnovnega harmonika  $sin$ in $cos$ pri meritvah stati"cne ekscentri"cnosti v smeri x glede na izhodiščno lego}{./MER/xs_sincos_phase}
\begin{eqnarray}
&A_{sin} = -6,14\cdot 10^{-2}\Delta x_s^3-1,71\cdot 10^{-2}\Delta x_s^2-1,17\cdot 10^{-2}\Delta x_s^1+4,84\cdot 10^{-1}            \\  
&Off_{sin} = -3,88\cdot 10^{-2}\Delta x_s^3+2,37\cdot 10^{-2}\Delta x_s^2-2,25\cdot 10^{-3}\Delta x_s^1-9,53\cdot 10^{-4}            \\
&\delta_{sin} = +5,32            \Delta x_s^3-3,55\cdot 10^{0}\Delta x_s^2+3,49\cdot 10^{0}\Delta x_s^1+1,08\cdot 10^{0}            \\ 
&A_{cos} = +7,46\cdot 10^{-3}\Delta x_s^3-1,85\cdot 10^{-1}\Delta x_s^2+1,01\cdot 10^{-2}\Delta x_s^1+4,88\cdot 10^{-1}            \\  
&Off_{cos} = +2,21\cdot 10^{-2}\Delta x_s^3-1,91\cdot 10^{-2}\Delta x_s^2-3,91\cdot 10^{-3}\Delta x_s^1+6,48\cdot 10^{-3}            \\
&\delta_{cos} = +1,15\cdot 10    \Delta x_s^3-1,06\cdot 10^{1}\Delta x_s^2-1,14            \Delta x_s^1+1,19
\end{eqnarray}
Slika prikazuje poteke amplitud posameznih harmonikov napake v odvisnosti od statične ekscentričnosti v smeri x. Kvadratično narašča amplituda drugega harmonika, medtem ko so enosmerna komponenta in ostali harmoniki konstantni. Poteke se lahko aproksimira s kubičnimi polinomi.
\slikaeps{Potek amplitud posameznega harmonika napake $\varepsilon$ pri meritvah stati"cne ekscentri"cnosti v smeri x}{./MER/xs_potek}
\begin{eqnarray}
&C_0 =2,42\Delta x_s^{3}-1,71\Delta x_s^{2}+2,40\cdot 10^{-1}\Delta x_s+1,15 \\              
&C_1 =-3,01\Delta x_s^{3}+2,35\Delta x_s^{2}-6,35\cdot 10^{-1}\Delta x_s+1,21 \\             
&C_2 =-1,11\Delta x_s^{3}+5,06\Delta x_s^{2}+9,95\cdot 10^{-1}\Delta x_s+1,18\cdot 10^{-1} \\
&C_3 =-2,10\Delta x_s^{3}+1,61\Delta x_s^{2}-5,25\cdot 10^{-1}\Delta x_s+3,57\cdot 10^{-1} \\
&C_4 =-3,24\Delta x_s^{3}+2,29\Delta x_s^{2}-4,73\cdot 10^{-1}\Delta x_s+4,17\cdot 10^{-1}
\end{eqnarray}






\newpage
\section{Meritve statične ekscentričnosti v smeri y-osi}
Senzor sem vrnil v izhodiščno lego in opravil meritve statične ekscentričnosti v smeri y-osi. Slika \ref{./MER/ys_sincos} prikazuje zajeta signala $sin$ in $cos$. Amplituda $sin$ se je zmanjšala, kot je bilo pričakovati po rezultatih simulacij. POsledično se izraazi v napaki višji drugi harmonik (slika \ref{./MER/ys_napaka}). Z razvojem napake v Fourierovo vrsto se amplitude posameznih harmonikov lažje razberejo. Očitno je narasla amplituda drugega harmonika, enosmerna komponenta in ostali harmoniki so ohranili enake amplitude.
\slikaeps{Signala $sin$ in $cos$ merjena pri 0,2 mm statične ekscentičnosti v smeri y}{./MER/ys_sincos}
\slikaeps{Napaka $\varepsilon$ merjena pri 0,2 mm statične ekscentičnosti v smeri y}{./MER/ys_napaka}
\slikaeps{Amplitude harmonikov napake $\varepsilon$ razvite v Fourierovo vrsto merjeno pri 0,2 mm statične ekscentičnosti v smeri y}{./MER/ys_fft}
\subsection{Sprememba $sin$, $cos$ ter napake od $\Delta y_s$}
Slika \ref{./MER/ys_sincos_amp} prikazuje potek amplitud osnovnega hamonika napake v odvisnosti statične ekscentričnosti v smeri y. S potekov se opazi padanje amplitud. Razlika med signaloma linearno narašča. 
Slika \ref{./MER/ys_sincos_off} prikazuje potek enosmernih komponent. Pri meritvi 0,15 mm se pojavi skok enosmernih komponent, vendar razlika ostaja enaka.
Fazni kot se v pri manjših odmikih ne spreminja kot je bilo pričakovati po rezultatih simulacij. Pri večjih odmikih se sprememba faznih kotov izrazi kot se je pričakovalo. Vsota faznih zamikov $sin$ in $cos$ je kolikortoliko konstantna, zato ni sprememb v enosmerni komponenti.
\slikaeps{Potek amplitude osnovnega harmonika $sin$ in $cos$ pri meritvah stati"cne ekscentri"cnosti v smeri y}{./MER/ys_sincos_amp}
\slikaeps{Potek enosmerne komponente $sin$ in $cos$ pri meritvah stati"cne ekscentri"cnosti v smeri y}{./MER/ys_sincos_off}
\slikaeps{Fazni zamik osnovnega harmonika  $sin$ in $cos$ pri meritvah stati"cne ekscentri"cnosti v smeri y glede na izhodiščno lego}{./MER/ys_sincos_phase}
\begin{eqnarray}
&A_{sin} = +1,73            \Delta y_s^3-6,47\cdot 10^{-1}\Delta y_s^2-2,60\cdot 10^{-1}\Delta y_s^1+4,94\cdot 10^{-1}            \\    
&Off_{sin} = +4,69\cdot 10^{-2}\Delta y_s^3+1,32\cdot 10^{-2}\Delta y_s^2-2,29\cdot 10^{-2}\Delta y_s^1+7,08\cdot 10^{-4}            \\ 
&\delta_{sin} = +3,37\cdot 10    \Delta y_s^3-2,12\cdot 10^{1}\Delta y_s^2+3,81            \Delta y_s^1+8,82\cdot 10^{-1}            \\ 
&A_{cos} = +1,99            \Delta y_s^3-9,29\cdot 10^{-1}\Delta y_s^2-7,82\cdot 10^{-2}\Delta y_s^1+4,91\cdot 10^{-1}            \\    
&Off_{cos} = +1,87\cdot 10^{-1}\Delta y_s^3-7,92\cdot 10^{-2}\Delta y_s^2-9,87\cdot 10^{-3}\Delta y_s^1+9,94\cdot 10^{-3}            \\ 
&\delta_{cos} = -1,76\cdot 10    \Delta y_s^3-9,77\cdot 10^{-1}\Delta y_s^2+4,42            \Delta y_s^1+3,59\cdot 10^{-1}
\end{eqnarray}

Slika \ref{./MER/ys_potek} prikazuje poteke amplitude posameznih harmonikov napake. Kot je bilo pričakovano se spreminja le amplituda drugega harmonika. 
\slikaeps{Potek amplitud posameznega harmonika napake $\varepsilon$ pri meritvah stati"cne ekscentri"cnosti v smeri y}{./MER/ys_potek}
\begin{eqnarray}
&C_0 =-3,26\Delta y_s^{3}-3,10\Delta y_s^{2}+2,68\Delta y_s+6,00\cdot 10^{-1} \\                                      
&C_1 =-2,38\Delta y_s^{3}+3,50\Delta y_s^{2}-1,54\Delta y_s+1,12 \\                                                   
&C_2 =-4,15\Delta y_s^{3}-1,64\Delta y_s^{2}+8,22\Delta y_s+8,04\cdot 10^{-2} \\                                      
&C_3 =2,30\Delta y_s^{3}-1,37\cdot 10^{-1}\Delta y_s^{2}-3,13\cdot 10^{-1}\Delta y_s+1,78\cdot 10^{-1} \\             
&C_4 =7,11\cdot 10^{-1}\Delta y_s^{3}-7,96\cdot 10^{-2}\Delta y_s^{2}-3,14\cdot 10^{-1}\Delta y_s+3,21\cdot 10^{-1}
\end{eqnarray}







\section{Meritve dinamične ekscentričnosti v smeri x-osi}

Meritve dinamične ekscentričnosti sem lahko delal le v eni koordinatni osi. Os po kateri se na napravi spreminja ekscentričnost magneta je zasukana za 32,4$^\circ$ na abciso. Meritev zato ni bilo možno opraviti neposredno v eni koordinatni osi. Enačbe spodaj aproksimirajo poteke amplitude osnovnega harmonika, faze osnovnega harmonika in enosmerne komponete $sin$ in $cos$, ter poteke amplitud posameznega harmonika napke pri dinamični ekscentričnosti v smeri x in zarotiranem magnetu za 32,4$^\circ$ pri simulacijah z realnim magnetnim poljem. V enačbah je ekscentričnost označena z $\Delta_{xd}$, z zavedanjem, da je mišljen potek z zasukanim magnetom. 
\begin{eqnarray}
&A_{sin}=+1,03\Delta x_d^3-5,83\cdot10^{0}\Delta x_d^2-3,77\cdot10^{-1}\Delta x_d^1+3,99\cdot10\\
&Off_{sin}=+4,24\cdot10^{-1}\Delta x_d^3+6,40\cdot10^{-1}\Delta x_d^2-7,57\Delta x_d^1+1,29\cdot10^{-1}\\
&\delta_{sin}=-4,59\Delta x_d^3+7,50\cdot10^{0}\Delta x_d^2-4,12\cdot10^{-1}\Delta x_d^1-1,38\cdot10^{-1}\\
&A_{cos}=+1,03\Delta x_d^3-5,83\cdot10^{0}\Delta x_d^2-3,77\cdot10^{-1}\Delta x_d^1+3,99\cdot10\\
&Off_{cos}=+4,24\cdot10^{-1}\Delta x_d^3+6,40\cdot10^{-1}\Delta x_d^2-7,57\Delta x_d^1+1,29\cdot10^{-1}\\
&\delta_{cos}=-4,59\Delta x_d^3+7,50\cdot10^{0}\Delta x_d^2-4,12\cdot10^{-1}\Delta x_d^1-1,38\cdot10^{-1}
\end{eqnarray}
\begin{eqnarray}
&C_0 =-4,95\Delta x_d^{3}+6,52\Delta x_d^{2}-4,16\cdot 10^{-1}\Delta x_d-1,37\cdot 10^{-1} \\                         
&C_1 =-2,62\Delta x_d^{3}+2,78\Delta x_d^{2}+2,11\cdot 10\Delta x_d-2,89\cdot 10^{-2} \\                              
&C_2 =4,68\cdot 10^{-1}\Delta x_d^{3}+5,21\cdot 10^{-1}\Delta x_d^{2}-2,03\cdot 10^{-2}\Delta x_d+2,54\cdot 10^{-3} \\
&C_3 =-4,90\cdot 10^{-2}\Delta x_d^{3}+8,11\cdot 10^{-1}\Delta x_d^{2}+7,89\Delta x_d+1,25\cdot 10^{-1} \\            
&C_4 =1,04\cdot 10\Delta x_d^{3}-2,55\Delta x_d^{2}-2,82\cdot 10^{-1}\Delta x_d+2,96\cdot 10^{-1}
\end{eqnarray}
Poteki so podobni simulacijam brez zasukanega magneta, amplitude nampake so nižje.

Pri dinamični ekscentričnosti v povzročeni smeri se signala $sin$ in $cos$ nista opazno spremenila (\ref{./MER/xd_sincos}). Sprememb ni niti na napaki (\ref{./MER/xd_napaka}). Spremembe niso opazne niti na  napaki razviti v Fourierovo vrsto (\ref{./MER/xd_fft}).
\slikaeps{Signala $sin$ in $cos$ merjena pri 0,19 mm dinamične ekscentičnosti v smeri x}{./MER/xd_sincos}
\slikaeps{Napaka $\varepsilon$ merjena pri 0,19 mm dinamične ekscentičnosti v smeri x}{./MER/xd_napaka}
\slikaeps{Amplitude harmonikov napake $\varepsilon$ razvite v Fourierovo vrsto merjeno pri 0,19 mm dinamične ekscentičnosti v smeri x}{./MER/xs_fft}
\subsection{Sprememba $sin$, $cos$ ter napake od $\Delta x_d$}

Potek spreminjanja amplitude osnovnega harmonika $sin$ in $cos$ glede na dinamično ekscentričnost pričakovano pada (Slika \ref{./MER/xd_sincos_amp}). Razlika med amplitudama ostaja tekom spreminjanja ekscentričnosti enaka. Enosmerni komponenti $sin$ in $cos$ padati, vendar vsaka s svojim gradientom. Sprememba enosmerne komponente je manjša, kot je bila predvidena v simulacijah. sprememba faznega zamika osnovnega harmonika $sin$ in $cos$ je prikazana na sliki \ref{./MER/xd_sincos_phase}. Fazna zamika s spremebo dinamične ekscentričnosti naraščata. V simulacijah tako velika sprememba ni bila pričakovana.
\slikaeps{Potek amplitude osnovnega harmonika $sin$ in $cos$ pri meritvah dinami"cne ekscentri"cnosti v smeri x}{./MER/xd_sincos_amp}
\slikaeps{Potek enosmerne komponente $sin$ in $cos$ pri meritvah dinami"cne ekscentri"cnosti v smeri x}{./MER/xd_sincos_off}
\slikaeps{Fazni zamik osnovnega harmonika  $sin$ in $cos$ pri meritvah dinami"cne ekscentri"cnosti v smeri x glede na izhodiščno lego}{./MER/xd_sincos_phase}

Poteke prikazane na slikah \ref{./MER/xd_sincos_amp} \ref{./MER/xd_sincos_off} in \ref{./MER/xd_sincos_phase}, se lahko aproksimira s kubičnimi polinomi.
\begin{eqnarray}
&A_{sin}=-7,95\cdot10^{-1}\Delta x_d^3+4,05\cdot10^{-1}\Delta x_d^2-1,06\cdot10^{-1}\Delta x_d^1+4,68\cdot10^{-1}\\
&Off_{sin}=-3,42\cdot10^{-3}\Delta x_d^3+7,48\cdot10^{-3}\Delta x_d^2-5,61\cdot10^{-3}\Delta x_d^1-2,53\cdot10^{-3}\\
&\delta_{sin}=+1,76\cdot10\Delta x_d^3-2,62\Delta x_d^2+1,18\cdot10^{0}\Delta x_d^1+8,78\cdot10^{-1}\\
&A_{cos}=-7,90\cdot10^{-1}\Delta x_d^3+4,00\cdot10^{-1}\Delta x_d^2-1,04\cdot10^{-1}\Delta x_d^1+4,70\cdot10^{-1}\\
&Off_{cos}=-1,58\cdot10^{-2}\Delta x_d^3+1,22\cdot10^{-2}\Delta x_d^2-1,63\cdot10^{-2}\Delta x_d^1+5,05\cdot10^{-3}\\
&\delta_{cos}=+2,58\cdot10\Delta x_d^3-8,64\Delta x_d^2+2,01\cdot10^{0}\Delta x_d^1+1,00\cdot10^{0}
\end{eqnarray}
Iz potekov $sin$ in $cos$ signala je bila najbolj opazna sprememba faznega zamika obeh signalov. To se izrazi tudi v napaki izhodnega kota, ki pridobi enosmerno komponento. Ostali harmoniki napake so konstantni.
\slikaeps{Potek amplitud posameznega harmonika napake $\varepsilon$ pri meritvah dinami"cne ekscentri"cnosti v smeri x}{./MER/xd_potek}
\begin{eqnarray}
&C_0 =2,03\cdot 10\Delta x_d^{3}-5,17\Delta x_d^{2}+1,57\Delta x_d+9,20\cdot 10^{-1} \\                                
&C_1 =-4,97\cdot 10^{-1}\Delta x_d^{3}+6,69\cdot 10^{-1}\Delta x_d^{2}-2,36\cdot 10^{-1}\Delta x_d+1,07 \\             
&C_2 =3,14\cdot 10^{-2}\Delta x_d^{3}-1,88\cdot 10^{-1}\Delta x_d^{2}+7,10\cdot 10^{-2}\Delta x_d+1,29\cdot 10^{-1} \\ 
&C_3 =-5,91\cdot 10^{-1}\Delta x_d^{3}+4,51\cdot 10^{-1}\Delta x_d^{2}-6,48\cdot 10^{-2}\Delta x_d+1,35\cdot 10^{-1} \\
&C_4 =-9,91\cdot 10^{-1}\Delta x_d^{3}+9,61\cdot 10^{-1}\Delta x_d^{2}-8,81\cdot 10^{-2}\Delta x_d+2,91\cdot 10^{-1} 
\end{eqnarray}




%\section{Meritve statične ekscentričnosti v smeri x-osi}
%
%Napako meritve povzročene zaradi statični ekscentričnosti v smeri x se pričakuje nižjo kot je bila v simulacijah. Na sliki \ref{./MER/xs_sincos} sta prikazana analogna signala $sin$ in $cos$. 
%
%\slikaeps{Signala $sin$ in $cos$ merjena pri 0,2 mm statične ekscentičnosti v smeri x}{./MER/xs_sincos}
%
%Amplituda signala $cos$ je manjša od amplitude $sin$ kot je bilo pričakovano. Napaka med merjenim kotom in referenčnim dajalnikom je prikazana na sliki \ref{./MER/xs_napaka}. Amplituda napake je manjša, manjša je tudi enosmerna komponenta. Z razvojem napake v Fourierovo vrsto, lažje primerjamo amplitude posameznih harmonikov napake (slika \ref{./MER/xs_fft}). Amplituda drugega harmonika je štirikrat nižja kot pri simulacijah z realnim poljem, enosmerna komponenta je manjša za faktor 10.
%
%\slikaeps{Napaka $\varepsilon$ merjena pri 0,2 mm statične ekscentičnosti v smeri x}{./MER/xs_napaka}
%
%\slikaeps{Amplitude harmonikov napake $\varepsilon$ razvite v Fourierovo vrsto merjeno pri 0,2 mm statične ekscentičnosti v smeri x}{./MER/xs_fft}
%
%
%\subsection{Sprememba $sin$, $cos$ ter napake od $\Delta x_s$}
%
%Poglejmo kako se spreminjata analogna signala od statične ekscentričnosti v smeri x. Amplituda signala $cos$ pada hitreje kot amplituda signala $sin$ (slika \ref{./MER/xs_sincos_amp}). Po rezultatih simulacij z realnim poljem, ni bilo predvideno tako hitro padanje amplitude signala $sin$. Potek enosmernih komponent je podoben kot pri simulacijah (slika \ref{./MER/xs_sincos_off}). Fazni zamik se od simulacij razlikujeta (slika \ref{./MER/xs_sincos_phase}). Medtem ko je potek fazni zamik signala $cos$ dokaj podoben po velikosti rezultatom iz simulacij, je fazni zamik signala $sin$ povsem drug. Fazni zamik signala sinus z večanjem ekscentričnosti pada zelo počasi.
%\slikaeps{Amplituda osnovnega harmonika signalov $sin$ in $cos$ pri meritvah stati"cne ekscentri"cnosti v smeri x}{./MER/xs_sincos_amp}
%\slikaeps{Amplituda osnovnega harmonika signalov $sin$ in $cos$ pri meritvah stati"cne ekscentri"cnosti v smeri x}{./MER/xs_sincos_off}
%\slikaeps{Amplituda osnovnega harmonika signalov $sin$ in $cos$ pri meritvah stati"cne ekscentri"cnosti v smeri x}{./MER/xs_sincos_phase}
%Poteke s slik \ref{./MER/xs_sincos_amp} \ref{./MER/xs_sincos_off} in \ref{./MER/xs_sincos_phase} lahko aproksimiramo z enačbami. 
%\begin{eqnarray}
%&A_{sin} = +1,52\cdot 10^{-2}\Delta x_s^3-4,97\cdot 10^{-2}\Delta x_s^2-1,03\cdot 10^{-2}\Delta x_s+3,55\cdot 10^{-1}\\     
%&Off_{sin} = +2,71\cdot 10^{-3}\Delta x_s^3-2,18\cdot 10^{-3}\Delta x_s^2+4,13\cdot 10^{-4}\Delta x_s-7,52\cdot 10^{-4}\\   
%&\delta_{sin} = +2,60\cdot 10^{-1}\Delta x_s^3-3,98\cdot 10^{-2}\Delta x_s^2-1,19\cdot 10^{-1}\Delta x_s+1,46\cdot 10^{-1}\\
%&A_{cos} = +6,24\cdot 10^{-3}\Delta x_s^3-1,15\cdot 10^{-1}\Delta x_s^2-3,45\cdot 10^{-2}\Delta x_s+3,56\cdot 10^{-1}\\     
%&Off_{cos} = -5,65\cdot 10^{-3}\Delta x_s^3+4,40\cdot 10^{-3}\Delta x_s^2-4,97\cdot 10^{-3}\Delta x_s+1,64\cdot 10^{-3}\\   
%&\delta_{cos} = -1,37\cdot 10^{-1}\Delta x_s^3+1,46\cdot 10^{-1}\Delta x_s^2+2,88\cdot 10^{-1}\Delta x_s+1,55\cdot 10^{-1}  
%\end{eqnarray}
%
%Signaloma $sin$ in $cos$ se najbolj spreminja amplituda, kar se izraža v poteku drugega harmonika napake. Ostali harmoniki so konstantni.
%\slikaeps{Potek amplitud posameznega harmonika napake $\varepsilon$ pri meritvah stati"cne ekscentri"cnosti v smeri x}{./MER/xs_potek}
%Poteke amplitud posameznih harmonikov napake aproksimiramo s kubičnimi polinmi.
%\begin{eqnarray}
%&C_0 =-5,36\cdot 10^{-1}\Delta x_s^{3}+5,10\cdot 10^{-1}\Delta x_s^{2}-1,32\cdot 10^{-2}\Delta x_s+1,75\cdot 10^{-1} \\
%&C_1 =-4,19\cdot 10^{-1}\Delta x_s^{3}+1,29\Delta x_s^{2}-5,47\cdot 10^{-1}\Delta x_s+2,67\cdot 10^{-1} \\             
%&C_2 =-4,24\Delta x_s^{3}+1,12\cdot 10\Delta x_s^{2}+3,35\cdot 10^{-1}\Delta x_s+3,93\cdot 10^{-2} \\                  
%&C_3 =2,14\cdot 10^{-1}\Delta x_s^{3}-1,10\cdot 10^{-1}\Delta x_s^{2}+7,53\cdot 10^{-3}\Delta x_s+1,86\cdot 10^{-1} \\ 
%&C_4 =-8,16\cdot 10^{-1}\Delta x_s^{3}+1,30\cdot 10^{-1}\Delta x_s^{2}-4,06\cdot 10^{-2}\Delta x_s+2,30\cdot 10^{-1}
%\end{eqnarray}
%
%
%
%
%
%\section{Meritve statične ekscentričnosti v smeri y-osi}
%Oglejmo si meritve statične eskscentričnosti v smeri y.
%Na sliki \ref{./MER/ys_sincos} sta prikazana analogna signala. Med signaloma ni posebne razlike. 
%V napaki (slika \ref{./MER/ys_napaka}) vidimo le negativno enosmerno komponento, potek napake je podben napaki brez ekscentričnosti (slika \ref{./MER/00_napaka}). 
%Razvoj napake v Fourierovo vrsto (slika \ref{./MER/ys_fft}) prikaže negativno enosmerno komponento in rahlo povečanje ostalih amplitud.
%
%\slikaeps{Signala $sin$ in $cos$ merjena pri 0,2 mm statične ekscentičnosti v smeri y}{./MER/ys_sincos}
%\slikaeps{Napaka $\varepsilon$ merjena pri 0,2 mm statične ekscentičnosti v smeri y}{./MER/ys_napaka}
%\slikaeps{Amplitude harmonikov napake $\varepsilon$ razvite v Fourierovo vrsto merjeno pri 0,2 mm statične ekscentičnosti v smeri y}{./MER/ys_fft}
%\subsection{Sprememba $sin$, $cos$ ter napake od $\Delta y_s$}
%Po nepričakovanih rezulatih si oglejmo potek analognih signalov $sin$ in $cos$ v odvisnosti od ekscentričnosti. Slika \ref{./MER/ys_sincos_amp} prikazuje odvistnost amplitude osnovnega harrmonika od ekscentrilnosti. 
%S slike je opaziti povečevanje razlike med amplitudama $sin$ in $cos$. Enosmerni komponenti signalov imati pričakovan potek. Fazni zamik signalov je tudi nepričakovan. 
%Od večanju statične ekscentričnosti v smeri y je v simulacijah razlika med faznima zamikoma naraščala zaradi spremembe faze signala $cos$.
%Pri meritvah je razlika nižja in narašča z nasprotnim predznakom.
%\slikaeps{Amplituda osnovnega harmonika signalov $sin$ in $cos$ pri meritvah stati"cne ekscentri"cnosti v smeri y}{./MER/ys_sincos_amp}
%\slikaeps{Amplituda osnovnega harmonika signalov $sin$ in $cos$ pri meritvah stati"cne ekscentri"cnosti v smeri y}{./MER/ys_sincos_off}
%\slikaeps{Amplituda osnovnega harmonika signalov $sin$ in $cos$ pri meritvah stati"cne ekscentri"cnosti v smeri y}{./MER/ys_sincos_phase}
%Poteke s slik \ref{./MER/ys_sincos_amp} \ref{./MER/ys_sincos_off} in \ref{./MER/ys_sincos_phase} lahko aproksimiramo z enačbami. 
%\begin{eqnarray}   
%&A_{sin} = -4,18\Delta y_s^3+3,10\Delta y_s^2-5,58\cdot 10^{-1} \Delta y_s+4,22\cdot 10^{-1} \\     
%&Off_{sin} = +7,31\cdot 10^{-3} \Delta y_s^3-8,00\cdot 10^{-3} \Delta y_s^2-9,23\cdot 10^{-4} \Delta y_s-1,18\cdot 10^{-4} \\   
%&\delta_{sin} = +2,36\cdot 10\Delta y_s^3-2,05\cdot 10\Delta y_s^2+2,18\Delta y_s-3,97\cdot 10^{-1} \\
%&A_{cos} = -4,11\Delta y_s^3+3,14\Delta y_s^2-5,68\cdot 10^{-1} \Delta y_s+4,24\cdot 10^{-1} \\     
%&Off_{cos} = +2,56\cdot 10^{-2}\Delta y_s^3-2,06\cdot 10^{-2}\Delta y_s^2+4,03\cdot 10^{-3}\Delta y_s+2,08\cdot 10^{-3} \\   
%&\delta_{cos} = -3,04\cdot 10\Delta y_s^3+2,45\cdot 10\Delta y_s^2-3,13\Delta y_s-5,46\cdot 10^{-1} 
%\end{eqnarray}
%
%
%
%
%
%\section{Meritve dinamične ekscentričnosti}
%
%
%\slikaeps{Signala $sin$ in $cos$ merjena pri 0,19 mm dinamične ekscentričnosti}{./MER/xd_sincos}
%\slikaeps{Napaka $\varepsilon$ merjena pri 0,19 mm statične ekscentičnosti v smeri y}{./MER/xd_napaka}
%\slikaeps{Amplitude harmonikov napake $\varepsilon$ razvite v Fourierovo vrsto merjeno pri 0,19 mm statične ekscentičnosti v smeri y}{./MER/xd_fft}
%\subsection{Sprememba $sin$, $cos$ ter napake od $\Delta y_s$}
%
%
%\slikaeps{Potek amplitud osnovnega harmonika signalov $sin$ in $cos$ pri meritvah dinamične ekscentričnosti}{./MER/xd_sincos_amp}
%\slikaeps{Potek enosmerne komponente signalov $sin$ in $cos$ pri meritvah  dinamične ekscentričnosti}{./MER/xd_sincos_off}
%\slikaeps{Potek faznega zamika osnovnega harmonika signalov $sin$ in $cos$ pri meritvah dinamične ekscentričnosti}{./MER/xd_sincos_phase}
%
%\begin{eqnarray}   
%aa
%\end{eqnarray}
%
%
%\slikaeps{Potek amplitud posameznega harmonika napake $\varepsilon$ ob dinamične ekscentričnosti}{./MER/xd_potek}
%
%\begin{eqnarray}   
%aa
%\end{eqnarray}
%
%
%
%
%
















































\chapter{Primerjava simulacij in meritev}
\chapter{Sklep}
\section{Merjenje kota in vpliv na napako}
Aplikacije, ki za delovanje potrebujejo informacijo o zasuku, kot lahko merijo na različne načine. Senzor RM44 za merjenje kota uporablja Hallove sonde. Senzor je robusten in zato primeren tudi za aplikacije v bolj obremenjenih okoljih. Izhod senzorja kljub robustonsti lahko vsebuje napako. Napaka je lahko posledica nepravilne montaže. S poznavanjem vplivov na napako senzorja zaradi nepravilne montaže, se napako lahko predvidi in odstrani.
\section{Nepravilna montaža}
Merilni sistem za merjenje kota je sestavljen iz dveh delov, magnetnega aktuatorja in senzorja za merjenje magnetnega polja. Nepravilno je lahko montiran aktuator ali senzor.
Nepravilno montriran senzor, se izrazi kot statična eksecentričnost. Hallova sonda v senzorju zajame magnetno polje s spremenjeno amplitudo in s spremenjenim faznim zamikom.
 Nepravilno montiran aktuator se izrazi kot dinamična ekscentričnost. Hallova sonda v senzorju pomeri dodatno enosmerno komponento magnetnega polja. Z diferencialnim odčitavanjem polja se odstrani enosmerno komponento.
\section{Oblika napake}
Zaradi nepravilnega zajema magnetnega polja, podatek o kotu vsebuje napako. Napaka zaradi dinamične ekscentričnosti se izrazi kot sinusni signal. Z diferenicialnim odčitavanjem napaka ne vsebuje prvega harmonika temveč le enosmerno komponeto. Napaka zaradi statične ekscentričnosti se izrazi s sinusnim signalom dvojne frekvence in dodano enosmerno komponento.
\section{Izvedba meritev}
Delovnje senzorja nam je nepoznano (black-box). Senzor vrne izmerjena signala $B_{sin}$ in $B_{cos}$ iz katerih se z alogritmom, ki je v osnovi  funkcija $atan2$ izračuna kot. Na napravi izdelani v LRTME na fakulteti, so bile opravljene meritve. Meritve so potrdile pričlakovanja.  Senzor ni bil postavljen v pravilno izhodiščno lego, zato se poteki statične ekscentričnosti v smeri x in y razlikujejo. Pravilno izhodiščno lego bi našel  z uporabo razvojne plošče RMK2. RMK2 vsebuje enka čip AM256, uporabniku so navoljo vsi pini čipa. Čip ima analogni signal Error, s katerim se lahko preveri poravnavo med osjo vrtenja, magnetom in čipom.
\section{Komentar rezultatov}
Zgradba in podrobnejše delovanje senzorja je poslovna skrinost, zato sem predvideval, da bo napaka pri meritvah manjša kot so pokazali rezultati simulacij. Meritve so pokazale, višje amplitude napake, kot so bile posimulirane, kar  Simulacijski model je bil sestavljen iz štirih Hallovih sond, neposredno iz zajetega polja je bil izračunan kot. 
Podobni senzorji \cite{iCMHM}, omogočajo popravljanje enosmernih komponent in razmerija amplitud osnovega harmonika signalov  $B_{sin}$ in $B_{cos}$. S kalibracijo senzorja bi bila napaka manjša.

Pri projektu sem se osredotočil le na parametre enosmerne komponente, amplitude in faze osnovnega harmonika  $B_{sin}$ in $B_{cos}$ Signala vsebujeta tudi višje harmonike, kateri tu niso bili obravnavani.




\bibliographystyle{ieeetrslo}
\bibliography{literatura}

\end{document}