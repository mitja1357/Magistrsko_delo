\chapter{Meritve}
%\section{Brez napake}
%\subsection{Sin\_cos}
%\subsection{napaka}
%\subsection{fft\_napake}
%\section{XS}
%\subsection{Sin\_cos}
%\subsection{napaka}
%\subsection{fft\_napake}
%\subsection{visanje\_napake}
%
%nastavek = 0.3352 xs^3-2.4826 xs^2+11.9361 xs+1.0156e-5
%H0	[-0,0788300698956381;-0,244060204305370;-0,687148831322084;0,0103846445852245]
%H1	[0,0676397954063620;0,341023878959022;2,17530863209977;0,380689577026790]
%H2	[0,0321995432854790;2,97431457433791;-3,25440542919127;19,9501165528466]
%H3	[0,161639792449458;-0,0674032561371874;0,396378799075362;-0,767078050899793]
%
%
%
%\section{YS}
%\subsection{Sin\_cos}
%\subsection{napaka}
%\subsection{fft\_napake}
%\subsection{visanje\_napake}
%H0	[-0,125225617182917;5,45211948990653;-7,01270447995828;5,96652583236181]
%H1	[0,0441520939449605;-0,212896911948413;12,7503402690231;-19,8802291697083]
%H2	[0,0232360469649201;-0,425126668548418;24,9444995014035;-34,9921278370895]
%H3	[0,152839960231160;-0,403917105230653;2,00190861822707;-2,53837100668412]
%\section{ZS}
%\subsection{Sin\_cos}
%\subsection{napaka}
%\subsection{fft\_napake}
%\subsection{visanje\_napake}
%
%H0	[-0,133667645230547;0,0552962954246356;-0,943817911449795;0,698305490182285]
%H1	[0,0403141624551963;0,180185960213708;0,129741148793180;0,0611532545109655]
%H2	[0,0379918588235422;0,0463681332197860;-0,0603146120527139;0,0684982804818869]
%H3	[0,139485505473835;0,0451213265486862;0,193417839376187;-0,147858653757135]
%\section{Xd}
%\subsection{Sin\_cos}
%\subsection{napaka}
%\subsection{fft\_napake}
%\subsection{visanje\_napake}
%
%H0	[-0,0235614911086793;0,321071338576502;-1,84834137393735;1,46394619238393]
%H1	[0,136347931291116;1,31166100194763;0,668435274468364;0,567598464373045]
%H2	[0,126562221569952;0,0367924696515395;-0,138846754407765;0,0924108962295517]
%H3	[0,152741191050619;0,0862732030589709;-0,153919077032749;0,217469596711140]
%
%\section{Rezultati med meritvami in simulacijami}


Simulacije so prikazale okvirne poteke analognih signalov ter napake  ob posamezni ekscentri"cnosti.
Na merilni napravi so bile opravljene meritve ekscentričnosti. V tem poglavju je opisana sama merilna naprava, zajem podatkov ter izvedba meritev.

\section{Oprema in postavitev merilnega mesta}

Merilno mesto sestavljala krmilna plošča za regulacijo motorskega pogona in obdelavo signalov sestavljena v LRTME.
Vsebuje elektromotorski pogon z inkrementalnim, referen"cnim dajalnikom zasuka TONiC podjetja Renishaw in magnetnim aktuatorjem za RM44 podjetja RLS  d.o.o.
Magnetni aktuator je možno premikati le v eni prostorski osi (slika \ref{premikanjeMagneta.png}).
Senzor RM44 je pritrjen na konstrukcijo 6 osnega mikrometrskega nastavljalnika pozicije HTIMS601.
Celotno merilno mesto je prikazano na sliki \ref{postavitevmerilnegamesta.png}.

%\bitnaslika{Dinamično ekscentričnost se lahko izmeri le v eni smeri}{premikanjeMagneta.png}

%\bitnaslika{Postavitev testnega mesta}{postavitevmerilnegamesta.png}%{0.6\textwidth}{0.877\textwidth}% 1.46 je razmerje visina sirina

Za manevriranje s HTIMS601 je potrebno nastaviti 6 osi.
Vsako os se nastavlja s enim od vijakov (\ref{HTIMS601.png}).
S postavitvijo koordinatnega sistema, je vsak od vijakov definiral premik senzorja. 
S spremembo vrtenja vijakov translacijskih osi, se je lokacija senzorja pred magnetom spreminjala z enako spremembo. S spremembo vrtenja rotacijskih vijakov, se je zaradi ročice na katero je pritrjen senzor, senzor zarotiral in hkrati tudi premaknil iz dotedanje lege. S spremembo rotacije je potrebno popraviti tudi nastavitve vijakovm, ki senzor premikajo v translacijskih oseh.
Na sliki \ref{HTIMS601.png} je prikazano kateri vijak je za nastavljanje posamezne prostorske osi. Vijaki poimenovani x-os, y-os in z-os so za nastavljanje translaciijo merjenca, rot x-osi, rot y-osi in rot z-osi so za nastavljanje rotacije premikajoče plošče na vrhu HTIMS601.
Senzor se je posledično ob premiku vijakov rot x-osi, rot y-osi in rot z-osi, glede na magnet vrtel in premikal.
S potenciometrom se nastavlja hitrost vrtenja motorja.
Hitrost vrtenja je nastavljena na 60 RPM (slika \ref{hitrost}). Hitrost ni popolnoma konstantna. Vzrok je v vztrajnosti pogona. Mitja Nemec je problem skušal čimbolje odpraviti, z dodajanjem primernih uteži na primerna mesta na vztrajniku.

Za senzor RM4 sem postavil koordinati sistem prikazan na sliki \ref{koordinatnisistem.png}
 
%\bitnaslika{Naprava za nastavljanje stati"cne ekscentri"cnosti}{HTIMS601.png}{0.7\textwidth}{1.023\textwidth}
%\slikaeps{Potek hitrosti od zasuka}{hitrost}
%\bitnaslika{Postavitev koordinatnega sistema}{koordinatnisistem.png}

\newpage
\section{Zajem podatkov}

Mitja Nemec je pripravil tudi grafični uporabniški vmesnik za prikazovanje meritev (slika \ref{GUI.png}).
Vmesnik lahko prikazuje potek refernečnega kota, analognih signalov sinus in kosinus senzorja RM44, izračunanega kota iz sin in cos signala, napako med izračunanim kotom senzorja in refernčnim dajalnikom, hitrost vrtenja ter tok prve faze motorskega pogona. Signaloma sin in cos se v programu prištejeti enosmerni komponenti, ki bi popravili izhodna signala.
Krmilna plošča zajema podatke s pogona s frekvenco 1kHz.
Referenčni inkrementalni dajalnik, se ob zagonu inicializira. V programu se podatek o kotu deli s 12595200. Definicijsko območje referenčnega kota se giblje med 0 in 1.
Signala sin in cos se na krmilni plošči ojačata in pretvorita z 12 bitnim AD pretvornikom. Izhodu AD pretvornika se odšteje 2048 in deli s 4096. Definicijsko območje sin in cos signala se gibljeta med $\pm0,5$.
Hitrost in napaka sta izračunana iz zajetih signalov.
Podatki so v obliki enega paketa poslani s krmilne plošče na 1 sekundo. Pri frekvenci vrtenja 1 Hz, grafični vmesnik prikaže en obrat.
Podatke se lahko izvozi v obliki .csv datoteke in nato poljubno obdela.

Na sliki  \ref{GUI.png} je prikazan sinusni signal prikazan kot da je zamaknjen za 180$\mathrm{^\circ}$. To je posledica pozitvne smeri vrtenja senzorja \cite{RM44}. Senzorju se lahko nastavi v katero smer narašča izhod. To sem rešil tako, da sem obrnil podatke. Popraviti je bilo potrebno tudi potek referenčnega dajalnika.
%\bitnaslika{Grafični vmesnik s potiki signalov}{GUI.png}
\section{Senzor v izhodiščni legi}
Senzor in magnet se lahko gibljeta, najprimernejša, izhodiščna lega, ni definirana. Z merilno urico Mitutoyo 543-391B se je dinamično ekscentričnost magneta nastavilo na najmanjšo. Oplet z merilno urico je bil pomerjen $\pm 6 \mu m_{pp}$ .
S prilagajanjem vijakov HTIMS601 in opazovanjem signalov sin in cos ter napake sem nastavil senzor v lego kjer je bila statična ekscentričnost najmanjša. Najprimernejšo lego sem iskal glede na največjo vrednost amplitud  $sin$ in $cos$ signalov ter fazni zamik. Opazovati je bilo potrebno  da se je signal nahajal v definicijskem območju zajema z AD pretvornikom.

Signala sin in cos sta prikazana na sliki \ref{./MER/00_sincos}.
S slike \ref{./MER/00_sincos} vidimo prisotnost enosmernih komponent. Posledično se izrazi v napaki prvi harmonik (slika \ref{./MER/00_napaka}).
Z razvojem napake v Fourierovo vrsto (slika \ref{./MER/00_fft}) vidimo velikosti posameznih amplitud napake. 
Enosmerna komponenta je posledica sofaznih zamikov obeh signalov $sin$ in $cos$. 
Prvi harmonik je posledica ensmermih komponent $sin$ in $cos$. Z matematično obdelavo siganlov $sin$ in $cos$ sem enosmerni komponenti odstranil,vendar se prvi harmonik napake še vedno izrazi. Prvi harmonik napake je odvisen tudi od drugega harmonika v signalih $sin$ in $cos$.
Z odstranitvijo tudi drugega harmnonika iz signalov $sin$ in $cos$ je bil  prvi harmonik v napaki odstranjen.
Signala  $sin$ in $cos$ med obdelavo ne bosta popravljena. Spreminjanje signalov  $sin$ in $cos$ in napake se bo opazovalo glede na potek, ki je bil pomerjen v izhodiščni legi.
\slikaeps{Signala $sin$ in $cos$ pomerjena v izhodi"s"cni legi}{./MER/00_sincos}
\slikaeps{Napaka $\varepsilon$ pomerjena v izhodi"s"cni legi}{./MER/00_napaka}
\slikaeps{Amplitude harmonikov napake $\varepsilon$ razvite v Fourierovo vrsto pri meritvah v izhodiščni legi}{./MER/00_fft}

\subsection{Meritve v izhodišni legi}
V izhodiščni legi sem opravil več meritev. Osredotočil sem se na enosmerni komponenti  in amplitudi osnovnega harmonika signalov $sin$ in $cos$. 
Porazdelitev enosmerne komponente signala $sin$ in $cos$ je prikazana na sliki \ref{./MER/00_off}.
Srednja vrednost enosmerne komponente $sin$ je $-8,85 \cdot 10^{-4}$, standardna deviacija je $1,08\cdot 10^{-4}$.
Srednja vrednost enosmerne komponente $cos$ je $4,20 \cdot 10^{-3}$, standardna deviacija je $5,20\cdot 10^{-5}$.
\slikaeps{Porazdelitev meritev enosmerne komponente signalov $sin$ in $cos$}{./MER/00_off}
Porazdelitev amplitude osnovnega harmonika signala $sin$ in $cos$ je prikazana na sliki \ref{./MER/00_off}.
Srednja vrednost amplitude osnovnega harmonika $sin$ je $0,451$, standardna deviacija je $2,20\cdot 10^{-4}$.
Srednja vrednost amplitude osnovnega harmonika $cos$ je $0,449$, standardna deviacija je $1,95\cdot 10^{-4}$.
\slikaeps{Porazdelitev meritev amplitude osnovnega harmonika signalov $sin$ in $cos$}{./MER/00_amp}

\section{Meritve statične ekscentričnosti v smeri x-osi}

Napako meritve povzročene zaradi statični ekscentričnosti v smeri x se pričakuje nižjo kot je bila v simulacijah. Na sliki \ref{./MER/xs_sincos} sta prikazana analogna signala $sin$ in $cos$. 

\slikaeps{Signala $sin$ in $cos$ merjena pri 0,2 mm statične ekscentičnosti v smeri x}{./MER/xs_sincos}

Amplituda signala $cos$ je manjša od amplitude $sin$ kot je bilo pričakovano. Napaka med merjenim kotom in referenčnim dajalnikom je prikazana na sliki \ref{./MER/xs_napaka}. Amplituda napake je manjša, manjša je tudi enosmerna komponenta. Z razvojem napake v Fourierovo vrsto, lažje primerjamo amplitude posameznih harmonikov napake (slika \ref{./MER/xs_fft}). Amplituda drugega harmonika je štirikrat nižja kot pri simulacijah z realnim poljem, enosmerna komponenta je manjša za faktor 10.

\slikaeps{Napaka $\varepsilon$ merjena pri 0,2 mm statične ekscentičnosti v smeri x}{./MER/xs_napaka}

\slikaeps{Amplitude harmonikov napake $\varepsilon$ razvite v Fourierovo vrsto merjeno pri 0,2 mm statične ekscentičnosti v smeri x}{./MER/xs_fft}


\subsection{Sprememba $sin$, $cos$ ter napake od $\Delta x_s$}

Poglejmo kako se spreminjata analogna signala od statične ekscentričnosti v smeri x. Amplituda signala $cos$ pada hitreje kot amplituda signala $sin$ (slika \ref{./MER/xs_sincos_amp}). Po rezultatih simulacij z realnim poljem, ni bilo predvideno tako hitro padanje amplitude signala $sin$. Potek enosmernih komponent je podoben kot pri simulacijah (slika \ref{./MER/xs_sincos_off}). Fazni zamik se od simulacij razlikujeta (slika \ref{./MER/xs_sincos_phase}). Medtem ko je potek fazni zamik signala $cos$ dokaj podoben po velikosti rezultatom iz simulacij, je fazni zamik signala $sin$ povsem drug. Fazni zamik signala sinus z večanjem ekscentričnosti pada zelo počasi.
\slikaeps{Amplituda osnovnega harmonika signalov $sin$ in $cos$ pri meritvah stati"cne ekscentri"cnosti v smeri x}{./MER/xs_sincos_amp}
\slikaeps{Amplituda osnovnega harmonika signalov $sin$ in $cos$ pri meritvah stati"cne ekscentri"cnosti v smeri x}{./MER/xs_sincos_off}
\slikaeps{Amplituda osnovnega harmonika signalov $sin$ in $cos$ pri meritvah stati"cne ekscentri"cnosti v smeri x}{./MER/xs_sincos_phase}
Poteke s slik \ref{./MER/xs_sincos_amp} \ref{./MER/xs_sincos_off} in \ref{./MER/xs_sincos_phase} lahko aproksimiramo z enačbami. 
\begin{eqnarray}
&A_{sin} = +1,52\cdot 10^{-2}\Delta x_s^3-4,97\cdot 10^{-2}\Delta x_s^2-1,03\cdot 10^{-2}\Delta x_s+3,55\cdot 10^{-1}\\     
&Off_{sin} = +2,71\cdot 10^{-3}\Delta x_s^3-2,18\cdot 10^{-3}\Delta x_s^2+4,13\cdot 10^{-4}\Delta x_s-7,52\cdot 10^{-4}\\   
&\delta_{sin} = +2,60\cdot 10^{-1}\Delta x_s^3-3,98\cdot 10^{-2}\Delta x_s^2-1,19\cdot 10^{-1}\Delta x_s+1,46\cdot 10^{-1}\\
&A_{cos} = +6,24\cdot 10^{-3}\Delta x_s^3-1,15\cdot 10^{-1}\Delta x_s^2-3,45\cdot 10^{-2}\Delta x_s+3,56\cdot 10^{-1}\\     
&Off_{cos} = -5,65\cdot 10^{-3}\Delta x_s^3+4,40\cdot 10^{-3}\Delta x_s^2-4,97\cdot 10^{-3}\Delta x_s+1,64\cdot 10^{-3}\\   
&\delta_{cos} = -1,37\cdot 10^{-1}\Delta x_s^3+1,46\cdot 10^{-1}\Delta x_s^2+2,88\cdot 10^{-1}\Delta x_s+1,55\cdot 10^{-1}  
\end{eqnarray}

Signaloma $sin$ in $cos$ se najbolj spreminja amplituda, kar se izraža v poteku drugega harmonika napake. Ostali harmoniki so konstantni.
\slikaeps{Potek amplitud posameznega harmonika napake $\varepsilon$ pri meritvah stati"cne ekscentri"cnosti v smeri x}{./MER/xs_potek}
Poteke amplitud posameznih harmonikov napake aproksimiramo s kubičnimi polinmi.
\begin{eqnarray}
&C_0 =-5,36\cdot 10^{-1}\Delta x_s^{3}+5,10\cdot 10^{-1}\Delta x_s^{2}-1,32\cdot 10^{-2}\Delta x_s+1,75\cdot 10^{-1} \\
&C_1 =-4,19\cdot 10^{-1}\Delta x_s^{3}+1,29\Delta x_s^{2}-5,47\cdot 10^{-1}\Delta x_s+2,67\cdot 10^{-1} \\             
&C_2 =-4,24\Delta x_s^{3}+1,12\cdot 10\Delta x_s^{2}+3,35\cdot 10^{-1}\Delta x_s+3,93\cdot 10^{-2} \\                  
&C_3 =2,14\cdot 10^{-1}\Delta x_s^{3}-1,10\cdot 10^{-1}\Delta x_s^{2}+7,53\cdot 10^{-3}\Delta x_s+1,86\cdot 10^{-1} \\ 
&C_4 =-8,16\cdot 10^{-1}\Delta x_s^{3}+1,30\cdot 10^{-1}\Delta x_s^{2}-4,06\cdot 10^{-2}\Delta x_s+2,30\cdot 10^{-1}
\end{eqnarray}





\section{Meritve statične ekscentričnosti v smeri y-osi}
Oglejmo si meritve statične eskscentričnosti v smeri y.
Na sliki \ref{./MER/ys_sincos} sta prikazana analogna signala. Med signaloma ni posebne razlike. 
V napaki (slika \ref{./MER/ys_napaka}) vidimo le negativno enosmerno komponento, potek napake je podben napaki brez ekscentričnosti (slika \ref{./MER/00_napaka}). 
Razvoj napake v Fourierovo vrsto (slika \ref{./MER/ys_fft}) prikaže negativno enosmerno komponento in rahlo povečanje ostalih amplitud.

\slikaeps{Signala $sin$ in $cos$ merjena pri 0,2 mm statične ekscentičnosti v smeri y}{./MER/ys_sincos}
\slikaeps{Napaka $\varepsilon$ merjena pri 0,2 mm statične ekscentičnosti v smeri y}{./MER/ys_napaka}
\slikaeps{Amplitude harmonikov napake $\varepsilon$ razvite v Fourierovo vrsto merjeno pri 0,2 mm statične ekscentičnosti v smeri y}{./MER/ys_fft}
\subsection{Sprememba $sin$, $cos$ ter napake od $\Delta y_s$}
Po nepričakovanih rezulatih si oglejmo potek analognih signalov $sin$ in $cos$ v odvisnosti od ekscentričnosti. Slika \ref{./MER/ys_sincos_amp} prikazuje odvistnost amplitude osnovnega harrmonika od ekscentrilnosti. 
S slike je opaziti povečevanje razlike med amplitudama $sin$ in $cos$. Enosmerni komponenti signalov imati pričakovan potek. Fazni zamik signalov je tudi nepričakovan. 
Od večanju statične ekscentričnosti v smeri y je v simulacijah razlika med faznima zamikoma naraščala zaradi spremembe faze signala $cos$.
Pri meritvah je razlika nižja in narašča z nasprotnim predznakom.
\slikaeps{Amplituda osnovnega harmonika signalov $sin$ in $cos$ pri meritvah stati"cne ekscentri"cnosti v smeri y}{./MER/ys_sincos_amp}
\slikaeps{Amplituda osnovnega harmonika signalov $sin$ in $cos$ pri meritvah stati"cne ekscentri"cnosti v smeri y}{./MER/ys_sincos_off}
\slikaeps{Amplituda osnovnega harmonika signalov $sin$ in $cos$ pri meritvah stati"cne ekscentri"cnosti v smeri y}{./MER/ys_sincos_phase}
Poteke s slik \ref{./MER/ys_sincos_amp} \ref{./MER/ys_sincos_off} in \ref{./MER/ys_sincos_phase} lahko aproksimiramo z enačbami. 
\begin{eqnarray}   
&A_{sin} = -4,18\Delta y_s^3+3,10\Delta y_s^2-5,58\cdot 10^{-1} \Delta y_s+4,22\cdot 10^{-1} \\     
&Off_{sin} = +7,31\cdot 10^{-3} \Delta y_s^3-8,00\cdot 10^{-3} \Delta y_s^2-9,23\cdot 10^{-4} \Delta y_s-1,18\cdot 10^{-4} \\   
&\delta_{sin} = +2,36\cdot 10\Delta y_s^3-2,05\cdot 10\Delta y_s^2+2,18\Delta y_s-3,97\cdot 10^{-1} \\
&A_{cos} = -4,11\Delta y_s^3+3,14\Delta y_s^2-5,68\cdot 10^{-1} \Delta y_s+4,24\cdot 10^{-1} \\     
&Off_{cos} = +2,56\cdot 10^{-2}\Delta y_s^3-2,06\cdot 10^{-2}\Delta y_s^2+4,03\cdot 10^{-3}\Delta y_s+2,08\cdot 10^{-3} \\   
&\delta_{cos} = -3,04\cdot 10\Delta y_s^3+2,45\cdot 10\Delta y_s^2-3,13\Delta y_s-5,46\cdot 10^{-1} 
\end{eqnarray}





\section{Meritve dinamične ekscentričnosti}


\slikaeps{Signala $sin$ in $cos$ merjena pri 0,19 mm dinamične ekscentričnosti}{./MER/xd_sincos}
\slikaeps{Napaka $\varepsilon$ merjena pri 0,19 mm statične ekscentičnosti v smeri y}{./MER/xd_napaka}
\slikaeps{Amplitude harmonikov napake $\varepsilon$ razvite v Fourierovo vrsto merjeno pri 0,19 mm statične ekscentičnosti v smeri y}{./MER/xd_fft}
\subsection{Sprememba $sin$, $cos$ ter napake od $\Delta y_s$}


\slikaeps{Potek amplitud osnovnega harmonika signalov $sin$ in $cos$ pri meritvah dinamične ekscentričnosti}{./MER/xd_sincos_amp}
\slikaeps{Potek enosmerne komponente signalov $sin$ in $cos$ pri meritvah  dinamične ekscentričnosti}{./MER/xd_sincos_off}
\slikaeps{Potek faznega zamika osnovnega harmonika signalov $sin$ in $cos$ pri meritvah dinamične ekscentričnosti}{./MER/xd_sincos_phase}

\begin{eqnarray}   
aa
\end{eqnarray}


\slikaeps{Potek amplitud posameznega harmonika napake $\varepsilon$ ob dinamične ekscentričnosti}{./MER/xd_potek}

\begin{eqnarray}   
aa
\end{eqnarray}




















































