\chapter{Meritve}
%\section{Brez napake}
%\subsection{Sin\_cos}
%\subsection{napaka}
%\subsection{fft\_napake}
%\section{XS}
%\subsection{Sin\_cos}
%\subsection{napaka}
%\subsection{fft\_napake}
%\subsection{visanje\_napake}
%
%nastavek = 0.3352 xs^3-2.4826 xs^2+11.9361 xs+1.0156e-5
%H0	[-0,0788300698956381;-0,244060204305370;-0,687148831322084;0,0103846445852245]
%H1	[0,0676397954063620;0,341023878959022;2,17530863209977;0,380689577026790]
%H2	[0,0321995432854790;2,97431457433791;-3,25440542919127;19,9501165528466]
%H3	[0,161639792449458;-0,0674032561371874;0,396378799075362;-0,767078050899793]
%
%
%
%\section{YS}
%\subsection{Sin\_cos}
%\subsection{napaka}
%\subsection{fft\_napake}
%\subsection{visanje\_napake}
%H0	[-0,125225617182917;5,45211948990653;-7,01270447995828;5,96652583236181]
%H1	[0,0441520939449605;-0,212896911948413;12,7503402690231;-19,8802291697083]
%H2	[0,0232360469649201;-0,425126668548418;24,9444995014035;-34,9921278370895]
%H3	[0,152839960231160;-0,403917105230653;2,00190861822707;-2,53837100668412]
%\section{ZS}
%\subsection{Sin\_cos}
%\subsection{napaka}
%\subsection{fft\_napake}
%\subsection{visanje\_napake}
%
%H0	[-0,133667645230547;0,0552962954246356;-0,943817911449795;0,698305490182285]
%H1	[0,0403141624551963;0,180185960213708;0,129741148793180;0,0611532545109655]
%H2	[0,0379918588235422;0,0463681332197860;-0,0603146120527139;0,0684982804818869]
%H3	[0,139485505473835;0,0451213265486862;0,193417839376187;-0,147858653757135]
%\section{Xd}
%\subsection{Sin\_cos}
%\subsection{napaka}
%\subsection{fft\_napake}
%\subsection{visanje\_napake}
%
%H0	[-0,0235614911086793;0,321071338576502;-1,84834137393735;1,46394619238393]
%H1	[0,136347931291116;1,31166100194763;0,668435274468364;0,567598464373045]
%H2	[0,126562221569952;0,0367924696515395;-0,138846754407765;0,0924108962295517]
%H3	[0,152741191050619;0,0862732030589709;-0,153919077032749;0,217469596711140]
%
%\section{Rezultati med meritvami in simulacijami}


Simulacije so prikazale okvirne poteke analognih signalov ter napake  ob posamezni ekscentri"cnosti.
Na merilni napravi so bile opravljene meritve ekscentričnosti. V tem poglavju je opisana sama merilna naprava, zajem podatkov ter izvedba meritev.

\section{Oprema in postavitev merilnega mesta}

Merilno mesto sestavljala krmilna plošča za regulacijo motorskega pogona in obdelavo signalov sestavljena v LRTME.
Vsebuje elektromotorski pogon z inkrementalnim, referen"cnim dajalnikom zasuka TONiC podjetja Renishaw in magnetnim aktuatorjem za RM44 podjetja RLS  d.o.o.
Magnetni aktuator je možno premikati le v eni prostorski osi (slika \ref{premikanjeMagneta.png}).
Senzor RM44 je pritrjen na konstrukcijo 6 osnega mikrometrskega nastavljalnika pozicije HTIMS601.
Celotno merilno mesto je prikazano na sliki \ref{postavitevmerilnegamesta.png}.

\bitnaslika{Dinamično ekscentričnost se lahko izmeri le v eni smeri}{premikanjeMagneta.png}

\bitnaslika{Postavitev testnega mesta}{postavitevmerilnegamesta.png}%{0.6\textwidth}{0.877\textwidth}% 1.46 je razmerje visina sirina

Za manevriranje s HTIMS601 je potrebno nastaviti 6 osi.
Vsako os se nastavlja s enim od vijakov (\ref{HTIMS601.png}).
S postavitvijo koordinatnega sistema, je vsak od vijakov definiral premik senzorja. 
S spremembo vrtenja vijakov translacijskih osi, se je lokacija senzorja pred magnetom spreminjala z enako spremembo. S spremembo vrtenja rotacijskih vijakov, se je zaradi ročice na katero je pritrjen senzor, senzor zarotiral in hkrati tudi premaknil iz dotedanje lege. S spremembo rotacije je potrebno popraviti tudi nastavitve vijakovm, ki senzor premikajo v translacijskih oseh.
Na sliki \ref{HTIMS601.png} je prikazano kateri vijak je za nastavljanje posamezne prostorske osi. Vijaki poimenovani x-os, y-os in z-os so za nastavljanje translaciijo merjenca, rot x-osi, rot y-osi in rot z-osi so za nastavljanje rotacije premikajoče plošče na vrhu HTIMS601.
Senzor se je posledično ob premiku vijakov rot x-osi, rot y-osi in rot z-osi, glede na magnet vrtel in premikal.
S potenciometrom se nastavlja hitrost vrtenja motorja.
Hitrost vrtenja je nastavljena na 60 RPM (slika \ref{hitrost}). Hitrost ni popolnoma konstantna. Vzrok je v vztrajnosti pogona. Mitja Nemec je problem skušal čimbolje odpraviti, z dodajanjem primernih uteži na primerna mesta na vztrajniku.

Za senzor RM4 sem postavil koordinati sistem prikazan na sliki \ref{koordinatnisistem.png}
 
\bitnaslika{Naprava za nastavljanje stati"cne ekscentri"cnosti}{HTIMS601.png}{0.7\textwidth}{1.023\textwidth}
\slikaeps{Potek hitrosti od zasuka}{hitrost}
\bitnaslika{Postavitev koordinatnega sistema}{koordinatnisistem.png}


\section{Zajem podatkov}

Mitja Nemec je pripravil tudi grafični uporabniški vmesnik za prikazovanje meritev (slika \ref{GUI.png}).
Vmesnik lahko prikazuje potek refernečnega kota, merjenega kota senzorja RM44, analognih signalov sinus in kosinus senzorja RM44, napako med senzorjem in refernčnim dajalnikom, hitrost vrtenja ter tok prve faze motorskega pogona.
Krmilna plošča zajema podatke s pogona s frekvenco 1kHz. Podatki so bili v obliki enega paketa poslani s krmilne plošče na 1 sekundo. Pri frekvenci vrtenja 1 Hz, grafišni vmesnik prikaže en obrat.
Podatke se lahko izvozi v obliki .csv datoteke in nato poljubno obdela.

Na sliki  \ref{GUI.png} je prikazan sinusni signal prikazan kot da je zamaknjen za 180$\mathrm{^\circ}$. To je posledica pozitvne smeri vrtenja senzorja \cite{RM44}. Senzorju se lahko nastavi v katero smer narašča izhod. To sem rešil tako, da sem obrnil podatke. Popraviti je bilo potrebno tudi potek referenčnega dajalnika.


\bitnaslika{Grafični vmesnik s potiki signalov}{GUI.png}



\section{Senzor v izhodi"s"cni poziciji}
Za meritve ekscentri"cnosti je bilo potrebno senzor, kot magnetni aktuator nastaviti v "cim bolj idealno pozicijo. Signala sin in cos sta prikazana na sliki \ref{./meritve/00_sincos}. 

\slikaeps{Signala $sin$ in $cos$ pomerjena v izhodi"s"cni legi}{./Meritve/meritev_ys_000u_BxBy}
Napaka $\varepsilon$, ki je bila pomerjena  je prikazana na sliki \ref{./Meritve/meritev_ys_000u_napaka}.
\slikaeps{Napaka $\varepsilon$ pomerjena v izhodi"s"cni legi}{./Meritve/meritev_ys_000u_napaka}
Napaka je po pri"cakovanjih manj"sa kot je bila pomerjena pri simulacijah. Z razvojem v Fourierovo vrsto (Slika \ref{./Meritve/meritev_ys_000u_fft}) se potrdi izstopanje "cetrtega harmonika, ki je bil izrazit tudi v simulacijah z realnim modelom polja.
Napako razvijmo v Fourierovo vrsto in pridobimo amplitude posameznih harmonikov napake(Slika \ref{./Meritve/meritev_ys_000u_fft}).
\slikaeps{Amplitude harmonikov napake $\varepsilon$ razvite v Fourierovo vrsto pri simulacijah z linearnim poljem pri 0,2 mm stati"cne ekscentri"cnosti v smeri x}{./Meritve/meritev_ys_000u_fft}
































\slikaeps{Signala $sin$ in $cos$ pri simulacijah z linearnim poljem pri 0,2 mm stati"cne ekscentri"cnosti v smeri x}{./Meritve/meritev_xs_200u_BxBy}

\slikaeps{Napaka $\varepsilon$ pri simulacijah z linearnim poljem pri 0,2 mm stati"cne ekscentri"cnosti v smeri x}{./Meritve/meritev_xs_200u_napaka}

\slikaeps{Amplitude harmonikov napake $\varepsilon$ razvite v Fourierovo vrsto pri simulacijah z linearnim poljem pri 0,2 mm stati"cne ekscentri"cnosti v smeri x}{./Meritve/meritev_xs_200u_fft}


\slikaeps{Potek amplitud posameznega harmonika napake $\varepsilon$ od stati"cne ekscentri"cnosti v smeri x}{potek_meritev_xs}

\begin{eqnarray}
&A_0=11,94 \Delta x_s^2-2,48\Delta x_s+ 0,34\\
&C_1=\sqrt{A_1^2+B_1^2}=1,58 10^{-13} \Delta x_s^2-6,52 10^{-13} \Delta x_s+ 9,47 10^{-13}\\
&C_2=16,88 \Delta x_s^2-3,53  \Delta x_s+0,41\\
&C_3=-2,43 10^{-14} \Delta x_s^2+3,52 10^{-13} \Delta x_s- 6,29 10^{-13}
\end{eqnarray}






























































