

\chapter{Realni model magnetnega polja}

V tem poglavju bom predstavil rezultate simulacij, v katerih sem uporabil realen model magnetnega polja. Podjetje RLS merilna tehnika, mi je posredovalo rezultate simulacije z komponente gostote magnetnega polja 2,55 mm nad magnetom.  Simulirano polje pokriva obmo"cje 20x20 mm nad magnetom s korakom 0,02 mm (Slika \ref{Realno_polje}).

\slikaeps{Model z- komponente gostote magnetnega polja uporabljen v simulacijah}{Realno_polje}



\section{Brez napake}
Kljub idealni monta"zi, se bo  v realnosti pojavila napaka. Magnet je lahko neenakomerno magnatiziran, kar nam "ze v za"cetku ustvari neko napako.
Poglejmo si zato najprej zajeta signala $sin$ in $cos$ (Slika \ref{./Real/sim_real_polje_xd_000u_BxBy}).

\slikaeps{Poteka signala $sin$ in $cos$ pri simulacijah z realnim magnetnim poljem brez vplivov ekscentri"cnosti}{./Real/sim_real_polje_xd_000u_BxBy}
Signala $sin$ in $cos$ nimata popolne oblike kot bi ju pri"cakovali. Napaka je tudi posledica numeri"cnega izra"cuna magnetnega polja. Ta nepopolnost analognih signalov se izra"za tudi v napaki (Slika \ref{./Real/sim_real_polje_xd_000u_napaka}).
\slikaeps{Napaka pri simulacijah z realnim magnetnim poljem brez vplivov ekscentri"cnosti}{./Real/sim_real_polje_xd_000u_napaka}
Napaka je bolj izrazita kot pri linearnem modelu polja. V napaki se pojavljajo tudi skoki (npr pri$105^{\circ}$ ), ki so posledica nepopolnega numeri"cno izra"cunanega modela magnetnega polja.
Razvijmo napako v Fourierovo vrsto(Slika \ref{./Real/sim_real_polje_xd_000u_fft}).
\slikaeps{Amplitude harmonikov napake  simulacijah z realnim magnetnim poljem brez ekscentri"cnosti}{./Real/sim_real_polje_xd_000u_fft}
Prevladujeta prvi in "cetrti harmonik kar je pri"cakovano po podatkovnih listih citeAM8192.

\section{Simulacija stati"cne eksentri"cnosti}

Oglejmo si sedaj rezultate ob upo"stevanju stati"cne ekscentri"cnosti. Po pr"cakovanjih se bo razlika pokazala v amplitudah in faznem zamiku analognih signalov. Slika \ref{./Real/sim_real_polje_xs_200u_BxBy} prikazuje poteka simuliranih signalov $sin$ in $cos$. Amlituda se ni opazno spremenila, spremenil pa se je fazni zamik.
\slikaeps{Poteka signala $sin$ in $cos$ pri simulacijah z realnim magnetnim poljem s 0,2 mm stati"cne ekscentri"cnost v smeri xi}{./Real/sim_real_polje_xs_200u_BxBy}
S prikazom napake (Slika \ref{./Real/sim_real_polje_xs_200u_napaka}) opazimo povi"sanje drugega harmonika. Z razvojem v Fourierovo vrsto lahko to tudi potrdimo (Slika \ref{./Real/sim_real_polje_xs_200u_fft}). 
\slikaeps{Poteka napake pri simulacijah z realnim magnetnim poljem s 0,2 mm stati"cne ekscentri"cnost v smeri xi}{./Real/sim_real_polje_xs_200u_napaka}

\slikaeps{Amplitude harmonikov napake  simulacijah z realnim magnetnim poljem s 0,2 mm stati"cne ekscentri"cnosti v smeri x}{./Real/sim_real_polje_xs_200u_fft}

Po pri"cakovanjih iz simulacij z linearnim modela magnetnim poljem se napaka izra"za enako. Enosmerna komponenta in drugi harmonik imata manj"so amplitudo, kot simulacije z uporabo magnetnega polja aproksimiranega z ravnino, je pa ve"sji dele"z drugih harmonikov.


Oglejmo si "se napako simulacije pri stati"cni ekscentri"cnosti v smeri y (Slika \ref{./Real/sim_real_polje_ys_200u_napaka}).Opazimo negativno enosmerno komponento ter izrazit drugi harmonik, kar je pri"cakovano.  Na sliki \ref{./Real/sim_real_polje_ys_200u_fft} vidimo napako razvito v Fourierovo vrsto, ki nam poka"ze pri"cakovane velikosti amplitud.
\slikaeps{Napaka $\varepsilon$ pri simulacijah z realnim magnetnim poljem s 0,2 mm stati"cne ekscentri"cnosti v smeri y}{./Real/sim_real_polje_ys_200u_napaka}

\slikaeps{Amplitude harmonikov napake  simulacijah z realnim magnetnim poljem s 0,2 mm stati"cne ekscentri"cnosti  v smeri y}{./Real/sim_real_polje_ys_200u_fft}

\newpage
Oglejmo si sedaj "se poteka spreminjanja velikosti amplitud od vi"sanja ekscentri"cnosti.

\slikaeps{Potek amplitud posameznega harmonika napake $\varepsilon$ od stati"cne ekscentri"cnosti v smeri x}{potek_sim_real_polje_xs}

\begin{eqnarray}
&A_0=-1,55 \Delta x_s^3+1,83 \Delta x_s^2+11,60 \Delta x_s\\
&C_1=-6,55\Delta x_s^3+4,37 \Delta x_s^2- 0,76\Delta x_s+0,2\\
&C_2=-0,15\Delta x_s^3+0,20\Delta x_s^2+12,14\Delta x_s-0,01\\
&C_3=-1,43\Delta x_s^3+1,07\Delta x_s^2-0,36\Delta x_s +0,10
\end{eqnarray}
Aproksimacije potekov s polinomom nam opi"sejo linearno nara"s"canje enesmerne komponente ter amplitude drugega harmonika, pri manj"sih odmikih.  Opazimo da  je amplituda prvega in tretjega harmonika dokaj konstantna.



%\subsection{Sin\_cos}
%\subsection{napaka}
%\subsection{fft\_napake}
%\subsection{visanje\_napake}
%
%nastavek = 0.3352 xs^3-2.4826 xs^2+11.9361 xs+1.0156e-5
%H0	[-0,00699209073722017;11,6029221097184;1,83279480170159;-1,55042221235733]
%H1	[0,213427020213799;-0,763231162909124;4,37386076888708;-6,54664493553176]
%H2	[0,0160541277521412;12,0524850069422;-0,197016637153101;-0,147612052035732]
%H3	[0,102950730580531;-0,353882914777200;1,07051373889150;-1,42532588217098]
%
%\section{YS}
%\subsection{Sin\_cos}
%\subsection{napaka}
%\subsection{fft\_napake}
%\subsection{visanje\_napake}
%
%

Oglejmo si "se poteke spreminjnanja velikosti amplitud pri ekscentri"cnsoti v smeri y.

\slikaeps{Potek amplitud posameznega harmonika napake $\varepsilon$ od stati"cne ekscentri"cnosti v smeri y}{potek_sim_real_polje_ys}

\begin{eqnarray}
&A_0=-3,06 \Delta x_s^3+2,74 \Delta x_s^2-12,43 \Delta x_s\\
&C_1=3,26\Delta x_s^3+4,21 \Delta x_s^2- 1,33\Delta x_s+0,22\\
&C_2=2,05\Delta x_s^3-0,74\Delta x_s^2+12,14\Delta x_s-0,01\\
&C_3=-2,61\Delta x_s^3+1,96\Delta x_s^2-0,38\Delta x_s +0,10
\end{eqnarray}


Pri manj"sih odmikih stati"cne ekscentri"cnosti v smeri y, pridobiva negativno vrednost enosmerna komponenta in vi"sa se amplituda drugega harmonika. Poteki so pri"cakovani. Podrobnej"so primerjavo potekov obeh simulacij bom opravil na koncu poglavja.


%H0	[-0,000774350577012469;-12,4304703563319;2,74013123468618;-3,06353835179313]
%H1	[0,223167923323101;-1,33342694043456;4,20808447040165;-3,26357540021727]
%H2	[-0,0137732220186476;12,1366926708397;-0,744557116980244;2,04824841975282]
%H3	[0,101621076582716;-0,382265446928020;1,96243881009797;-2,61456623855890]
%
%\section{ZS}
%imam polje le v ravnini
\section{Simulacije dinam"cne ekscentri"cnosti }
%\subsection{Sin\_cos}
%\subsection{napaka}
%\subsection{fft\_napake}
%\subsection{visanje\_napake      }  
%
Poglejmo si sedaj "se dinami"cno ekscentir"cnost.Po simulacijah dinami"cne ekscentri"cnosti v smeri x pri linearnem polju, pri"cakujemo ob enosmerno komponento. Primer dinami"cne ekscentri"cnosti je na sliki \ref{./Real/sim_real_polje_xd_200u_BxBy}.
\slikaeps{Poteka signala $sin$ in $cos$ pri simulacijah z realnim magnetnim poljem s 0,2 mm dinami"cne ekscentri"cnost v smeri xi}{./Real/sim_real_polje_xd_200u_BxBy}
Velikost enosmerne komponente $sin$ je $-1,60\mathrm{mT}$ in $cos$ $-1,59\mathrm{mT}$. Enosmerna komponenta ni tako izrazita, kot pri simulacijah z linearnim poljem, vendar kljub temu povzro"ci napako (Slika \ref{./Real/sim_real_polje_xd_200u_napaka}).

\slikaeps{Poteka napake pri simulacijah z realnim magnetnim poljem s 0,2 mm dinami"cne ekscentri"cnost v smeri x}{./Real/sim_real_polje_xd_200u_napaka}
V napaki se izrazite poka"ze prvi harmonik kar je bilo pri"cakovati. Razvijmo napako "se v Fourierovo vrsto.
\slikaeps{Amplitude harmonikov napake  simulacijah z realnim magnetnim poljem s 0,2 mm dinami"cne ekscentri"cnosti v smeri x}{./Real/sim_real_polje_xd_200u_fft}
po pri"cakovanjih je najvi"sja amplituda prvega harmonika, narasel je tudi tretji harmonik.

Poglejmo si sedaj "se spreminjanje velikosti amplitud s spreminjanjem velikosti dinami"cne ekscentri"cnosti v smeri x.

\slikaeps{Potek amplitud posameznega harmonika napake $\varepsilon$ od dinamii"cne ekscentri"cnosti v smeri x}{potek_sim_real_polje_xd}

Aproksimirajmo poteke s polinomo tretje stopnje. po pri"cakovanjih je izrazito nara"s"canje prvega harmonika.
\begin{eqnarray}
&A_0=-5,70 \Delta x_d^3+5,27 \Delta x_d^2-0,89 \Delta x_d+0,01\\
&C_1=-8,36\Delta x_d^3+9,16 \Delta x_d^2-+22,94 \Delta x_d+0,01\\
&C_2=-1,75\Delta x_d^3+2,46\Delta x_d^2-0,40\Delta x_d+0,02\\
&C_3=1,04\Delta x_d^3-1,31\Delta x_d^2+8,71\Delta x_d+0,09
\end{eqnarray}


Simulacije dinami"cne ekscentri"cnosti v y smeri z aproksimacijo magnetnega polja z ravnino ni bilo mogo"ce. Z realnim modelom sem lahko opravil simulacije. Pri"cakovanja so, da se bo tudi tu izrazil prvi harmonik. Kot rezultat prilagam le poteke amplitud posameznega harmonika napake glede na ekscentri"cnost v smeri y.

\slikaeps{Potek amplitud posameznega harmonika napake $\varepsilon$ od dinamii"cne ekscentri"cnosti v smeri y}{potek_sim_real_polje_yd}

Aproksimirajmo poteke s polinomo tretje stopnje. Po pri"cakovanjih je izrazito nara"s"canje prvega harmonika. Amplituda prvega harmonika je pri dinami"cni ekscentri"cnosti v smeri y za faktor 3 ni"zja kot pri ekscentri"cnosti v x  smeri.
\begin{eqnarray}
&A_0=2,14 \Delta x_d^3-1,86\Delta x_d^2+0,80 \Delta x_d+0,02\\
&C_1=-11,96\Delta x_d^3+10,16 \Delta x_d^2+6,19 \Delta x_d+0,12\\
&C_2=-0,02\Delta x_d^3+0,75\Delta x_d^2-0,07\Delta x_d+0,01\\
&C_3=-8,33\Delta x_d^3+6,32\Delta x_d^2+7,41\Delta x_d
\end{eqnarray}

%\slikaeps{Poteka signala $sin$ in $cos$ pri simulacijah z realnim magnetnim poljem s 0,2 mm dinami"cne ekscentri"cnost v smeri y}{./Real/sim_real_polje_yd_200u_BxBy}
%
%
%\slikaeps{Poteka napake pri simulacijah z realnim magnetnim poljem s 0,2 mm dinami"cne ekscentri"cnost v smeri y}{./Real/sim_real_polje_yd_200u_napaka}
%
%\slikaeps{Amplitude harmonikov napake  simulacijah z realnim magnetnim poljem s 0,2 mm dinami"cne ekscentri"cnosti v smeri y}{./Real/sim_real_polje_yd_200u_fft}
%
%H0	[0,0111510309383977;-0,894923054154696;5,27197808464260;-5,69819797482716]
%H1	[0,0143170544619670;22,9403542088781;9,15989005045333;-8,36409138223516]
%H2	[0,0189230245440221;-0,395754684442712;2,46036707596027;-1,75121765929075]
%H3	[0,0862709237403795;8,70777569715877;-1,31040329058477;1,03796682705842]
%
%\section{YD}
%\subsection{Sin\_cos}
%\subsection{napaka}
%\subsection{fft\_napake}
%\subsection{visanje\_napake}
%
%H0	[-0,0154347120752856;0,797779839901433;-1,86015520249516;2,13823367319206]
%H1	[0,120051243593169;6,18566630103253;10,1603842022267;-11,9616860661348]
%H2	[0,00846974696955741;-0,0715292486907690;0,746150539220961;-0,0208954190768071]
%H3	[0,00115885710505121;7,41365534917132;6,31804550306713;-8,33363124295160]

V tem poglavju smo spoznali realno polje, ki ga merijo Hall-ove sonde. Rezultati imajo manj"so napako kot pri simulacijah z aproksimiranim linearnim magnetnim poljem.  Simulacijski model senzorja je bil zelo poenostavljen. Senzor RM44 vsebuje tudi interpolator ki z numeri"cnimipostopki izbolj"sa lo"cljivost, ter zmanj"sa napako. Iz tega lahko predvidimo, da bo napaka senzorja manj"sa, kot so bili rezultati simulacij.
