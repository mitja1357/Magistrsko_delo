\chapter{Realni model magnetnega polja}
S poznavanjem točnejše funkcije polja, je točnejše predvidevanje potekov realnih $B_{sin}$, $B_{cos}$ in napake. Podjetje RLS,  je posredovalo rezultate Z-komponente gostote magnetnega polja 1,80 mm nad simuliranim magnetom. Simuliran magnet je imel 4 mm premer in 4 mm višine. Definirano je imel remanenco 1050~mT in relativno permeabilnost 1.  Definicijsko območje  je 20x20 mm s korakom 0,02 mm (Slika \ref{Realno_polje}).
\slikaeps{Model Z- komponente gostote magnetnega polja uporabljen v simulacijah}{Realno_polje}
Simulacijski model je definiran s 4 sondami, kot je opisano v začetku poglavja \ref{linearnimodel}.
Polje, ki ga pomeri posamezna Hallova sonda ob zasuku ni bilo definirano oz. izpeljano analitično kot pri linearni aproksimaciji (\ref{equ:lin_potek_sin}) - (\ref{equ:lin_potek_zadnja}).  Prvo je bila izračunana relativna lokacija sonde glede na magnet. Nato se je vrednost Z-komponente gostote magnetnega polja  aproksimiralo s poljem, definiranim v geometrijsko najbližji točki definicijskega območja.
Pomerjeno polje nasproti ležečih sond se je nato odštelo in vstavilo v funkcijo atan2d();.
V tem poglavju so predstavljeni rezultati simulacij ekscentričnosti z uporabo realnega modela magnetnega polja.
\section{Napaka brez vplivov ekscentričnosti}
Kljub idealni montaži, $B_{sin}$ in $B_{cos}$ nista idealna signala.  Signala vsebujeta višje harmonike (Slika \ref{./rea/00_sincos}), ki niso opazno izraziti. Višji harmoniki se izrazijo v napaki (slika \ref{./rea/00_napaka}). Opazi se grob potek četrtega harmonika. V napaki so špice (primer $220^{\circ}$), ki so posledica načina izračuna Z-komponente gostote  magnetnega polja simuliranega magneta. Napaka razvita v Fourierovo vrsto prikaže vrednosti amplitud prvih štirih harmonikov napake (slika \ref{./rea/00_fft}). Izrazit je četrti harmonik, ki je pričakovan po podatkovnih listih \cite{AM8192}. Vzrok četrtega harmonika je tretji harmonik v signalih  $B_{sin}$ in $B_{cos}$, ki se pojavi zaradi magnetenja.
\slikaeps{$B_{sin}$ in $B_{cos}$ pri simulacijah z realnim magnetnim poljem brez ekscentričnosti}{./rea/00_sincos}
\slikaeps{Napaka $\varepsilon$ pri simulacijah z realnim magnetnim poljem brez ekscentričnosti}{./rea/00_napaka}
\slikaeps{Amplitude harmonikov napake $\varepsilon$ razvite v Fourierovo vrsto pri simulacijah z realnim poljem brez ekscentričnosti}{./rea/00_fft}
\newpage
\section{Simulacija statične ekscentričnosti v smeri x-osi}
Po pričakovanjih se bo spremenila amplituda $B_{sin}$ in $B_{cos}$ signala ter zmanjšal njun fazni zamik (izraza (\ref{equ:lin_potek_sin}) in \ref{equ:lin_potek_cos}). Na sliki \ref{./rea/xs_sincos} se opazi zmanjšano amplitudo signala $B_{cos}$. Na sliki \ref{./rea/xs_napaka} je prikazana napaka $\varepsilon$.  Razvoj napake v Fourierovo vrsto (slika \ref{./rea/xs_fft}) prikaže predvideno povišanje amplitudo drugega harmonika, kar je predvideno tudi v podatkovnih listih \cite{AM8192}.
\slikaeps{$B_{sin}$ in $B_{cos}$ pri simulacijah z realnim poljem pri 0,20 mm statične ekscentričnosti v smeri x}{./rea/xs_sincos}
\slikaeps{Napaka $\varepsilon$ pri simulacijah z realnim poljem pri 0,20 mm statične ekscentričnosti v smeri x}{./rea/xs_napaka}
\slikaeps{Amplitude harmonikov napake $\varepsilon$ razvite v Fourierovo vrsto pri simulacijah z realnim poljem pri 0,20mm statične ekscentričnosti v smeri x}{./rea/xs_fft}
\newpage
\subsection{Sprememba signalov Hallovih sond ter napake v odvisnosti od statične ekscentričnosti v smeri x}
Na sliki \ref{./rea/xs_sincos_amp} je prikazana sprememba amplitude prvega harmonika signalov $B_{sin}$ in $B_{cos}$. Pričakovano je hitrejše spreminjanje amplitude signala $B_{cos}$. Na sliki \ref{./rea/xs_sincos_off} je prikazan potek enosmerne komponente $B_{sin}$ in $B_{cos}$. Po rezultatih simulacij ni bilo pričakovano spreminjanje enosmerne komponente. Enosmerna komponenta se spreminja, vendar je njen vpliv praktično zanemarljiv. Slika \ref{./rea/xs_sincos_phase} prikazuje potek faznih signalov. Fazi osajati praktično konstantni.
\slikaeps{Amplituda osnovnega harmonika  $B_{sin}$ in $B_{cos}$ pri simulacijah z realnim poljem statične ekscentričnosti v smeri x}{./rea/xs_sincos_amp}
\slikaeps{Enosmerna komponenta $B_{sin}$ in $B_{cos}$ pri simulacijah z realnim poljem statične ekscentričnosti v smeri x}{./rea/xs_sincos_off}
\slikaeps{Fazni zamik $B_{sin}$ in $B_{cos}$ pri simulacijah z realnim poljem statične ekscentričnosti v smeri x glede na idealna signala $B_{sin}$ in $B_{cos}$}{./rea/xs_sincos_phase}
\newpage
Poteke se aproksimira s kubičnimi polinomi.
Na sliki \ref{./rea/xs_potek} so prikazani poteki amplitud posameznega harmonika ob spreminjanja statične ekscentričnosti v smeri x. Amplituda drugega harmonika narašča eksponentno, ensmerna komponenta nekoliko pada (\ref{real_xs_C0})-(\ref{real_xs_C4}). Zanimiv je tudi upad četrtega harmonika.
\begin{eqnarray}
&\begin{split}
Off_{sin}(\Delta x_s) =1,15\cdot 10^{-3}\Delta x_s^{3}-1,67\cdot 10^{-3}\Delta x_s^{2}+6,08\cdot 10^{-4}\Delta x_s\\-2,68\cdot 10^{-5}
\end{split}\\
&A_{sin}(\Delta x_s) =1,84\cdot 10^{-1}\Delta x_s^{3}-5,17\Delta x_s^{2}+2,24\cdot 10^{-2}\Delta x_s+6,83\cdot 10 \\                    
&\delta_{sin}(\Delta x_s) =-1,02\Delta x_s^{3}+8,18\cdot 10^{-1}\Delta x_s^{2}-1,31\cdot 10^{-1}\Delta x_s+6,23\cdot 10^{-3} \\             
&Off_{cos}(\Delta x_s) =8,50\Delta x_s^{3}-7,49\Delta x_s^{2}+1,60\Delta x_s-4,68\cdot 10^{-3} \\                                       
&A_{cos}(\Delta x_s) =1,27\cdot 10\Delta x_s^{3}-2,50\cdot 10\Delta x_s^{2}+1,16\Delta x_s+6,83\cdot 10 \\                              
&\delta_{cos}(\Delta x_s) =1,12\cdot 10\Delta x_s^{3}-8,16\Delta x_s^{2}+1,34\Delta x_s-3,78\cdot 10^{-2}
\end{eqnarray}
\slikaeps{Potek amplitud posameznega harmonika napake $\varepsilon$ od statične ekscentričnosti v smeri x pri simulacijah z realnim poljem}{./rea/xs_potek}
\begin{eqnarray}
\label{real_xs_C0}
&C_0(\Delta x_s) =5,02\Delta x_s^{3}-3,63\Delta x_s^{2}+5,93\cdot 10^{-1}\Delta x_s-1,47\cdot 10^{-2} \\              
&C_1(\Delta x_s) =3,79\Delta x_s^{3}-2,50\Delta x_s^{2}+5,17\cdot 10^{-1}\Delta x_s-4,10\cdot 10^{-3} \\              
&C_2(\Delta x_s) =1,05\cdot 10\Delta x_s^{3}-1,52\Delta x_s^{2}+4,53\cdot 10^{-1}\Delta x_s-5,39\cdot 10^{-5} \\      
&\begin{split}C_3(\Delta x_s) =-1,41\Delta x_s^{3}+8,22\cdot 10^{-2}\Delta x_s^{2}+3,19\cdot 10^{-1}\Delta x_s\\+7,42\cdot 10^{-3} \end{split}\\
\label{real_xs_C4}
&C_4(\Delta x_s) =1,24\Delta x_s^{3}+2,20\Delta x_s^{2}-1,80\Delta x_s+3,64\cdot 10^{-1} 
\end{eqnarray}

\section{Simulacija statične ekscentričnosti v smeri y-osi}
Pri statični ekscentričnosti v smeri y, je pričakovano glede na prejšnje podpoglavje, znižanje amplitude $B_{sin}$ (slika \ref{./rea/ys_sincos}).  Napaka $\varepsilon$ (slika \ref{./rea/ys_napaka}) je enake oblike kot je bila pri simulacijah statične ekscentričnosti v smeri x (slika \ref{./rea/xs_napaka}). Razvoj napake v Fourierovo vrsto (slika \ref{./rea/xs_fft}) potrdi pričakovanja.
\slikaeps{$B_{sin}$ in $B_{cos}$ pri simulacijah z realnim poljem pri 0,20 mm statične ekscentričnosti v smeri y}{./rea/ys_sincos}
\slikaeps{Napaka $\varepsilon$ pri simulacijah z realnim poljem pri 0,20 mm statične ekscentričnosti v smeri y}{./rea/ys_napaka}
\slikaeps{Amplitude harmonikov napake $\varepsilon$ razvite v Fourierovo vrsto pri simulacijah z realnim poljem pri 0,20 mm statične ekscentričnosti v smeri y}{./rea/ys_fft}
\newpage
\subsection{Sprememba signalov Hallovih sond ter napake v odvisnosti od statične ekscentričnosti v smeri y}
Potek amplitude osnovnega harmonika $B_{sin}$ (slika \ref{./rea/ys_sincos_amp}) se spreminja kot se je spreminjala amplituda osnovnega hamronika $B_{cos}$ pri simulacijah statične ekscentričnosti v smeri x.  Enosmerena komponenta (slika \ref{./rea/ys_sincos_off}) pri $B_{sin}$ se spreminja enako, kot enosmerna komponenta $B_{cos}$ pri statični ekscentričnosti v smeri x (slika \ref{./rea/xs_sincos_off}).  Fazni zamik signala $B_{sin}$ se spreminja, kot se je spreminjal zamik  $B_{cos}$ pri ekscentričnosti v smeri x (slika \ref{./rea/ys_sincos_phase}). Poteki so zapisani s kubičnimi polinomi.
\begin{eqnarray}
&Off_{sin}(\Delta y_s) =8,50\Delta y_s^{3}-7,49\Delta y_s^{2}+1,60\Delta y_s-4,68\cdot 10^{-3} \\                                        
&A_{sin}(\Delta y_s) =1,27\cdot 10\Delta y_s^{3}-2,50\cdot 10\Delta y_s^{2}+1,16\Delta y_s+6,83\cdot 10 \\                               
&\delta_{sin}(\Delta y_s) =1,12\cdot 10\Delta y_s^{3}-8,16\Delta y_s^{2}+1,34\Delta y_s-3,78\cdot 10^{-2} \\                              
&\begin{split}Off_{cos}(\Delta y_s) =-1,15\cdot 10^{-3}\Delta y_s^{3}+1,67\cdot 10^{-3}\Delta y_s^{2}-6,08\cdot 10^{-4}\Delta y_s\\+2,68\cdot 10^{-5}\end{split} \\
&A_{cos}(\Delta y_s) =1,84\cdot 10^{-1}\Delta y_s^{3}-5,17\Delta y_s^{2}+2,24\cdot 10^{-2}\Delta y_s+6,83\cdot 10 \\                     
&\begin{split}\delta_{cos}(\Delta y_s) =-1,02\Delta y_s^{3}+8,18\cdot 10^{-1}\Delta y_s^{2}-1,31\cdot 10^{-1}\Delta y_s\\+6,23\cdot 10^{-3} \end{split}
\end{eqnarray}
\slikaeps{Amplituda osnovnega harmonika  $B_{sin}$ in $B_{cos}$ pri simulacijah z realnim poljem statične ekscentričnosti v smeri y}{./rea/ys_sincos_amp}
\slikaeps{Enosmerna komponenta $B_{sin}$ in $B_{cos}$ pri simulacijah z realnim poljem statične ekscentričnosti v smeri y}{./rea/ys_sincos_off}
\slikaeps{Fazni zamik $B_{sin}$ in $B_{cos}$ pri simulacijah z realnim poljem statične ekscentričnosti v smeri y glede na idealna signala $B_{sin}$ in $B_{cos}$}{./rea/ys_sincos_phase}

Enačbe prikazujejo enake poteke kot poteki pri statični ekscentričnosti v smeri x.
Posledično to vpliva na posamezne harmonike napake.Amplitude harmonikov napake se izrazijo enako kot se je izrazila napaka pri statični ekscentričnosti v smeri x.
\slikaeps{Potek amplitud posameznega harmonika napake $\varepsilon$ od statične ekscentričnosti v smeri y pri simulacijah z realnim poljem}{./rea/ys_potek}
\begin{eqnarray}
&C_0(\Delta y_s) =5,02\Delta y_s^{3}-3,63\Delta y_s^{2}+5,93\cdot 10^{-1}\Delta y_s-1,47\cdot 10^{-2} \\              
&C_1(\Delta y_s) =3,79\Delta y_s^{3}-2,50\Delta y_s^{2}+5,17\cdot 10^{-1}\Delta y_s-4,10\cdot 10^{-3} \\              
&C_2(\Delta y_s) =1,05\cdot 10\Delta y_s^{3}-1,52\Delta y_s^{2}+4,53\cdot 10^{-1}\Delta y_s-5,39\cdot 10^{-5} \\      
&\begin{split}C_3(\Delta y_s) =-1,41\Delta y_s^{3}+8,22\cdot 10^{-2}\Delta y_s^{2}+3,19\cdot 10^{-1}\Delta y_s\\+7,42\cdot 10^{-3} \end{split}\\
&C_4(\Delta y_s) =1,24\Delta y_s^{3}+2,20\Delta y_s^{2}-1,80\Delta y_s+3,64\cdot 10^{-1}     
\end{eqnarray}
\section{Dinamična ekscentričnost v smeri x}
Vpliv dinamične ekscentričnosti v $B_{sin}$ in $B_{cos}$ je pričakovan v obliki enosmerne komponente, vendar se zaradi diferencialnega zajema odštejeti. Na sliki  \ref{./rea/xd_sincos} sta $B_{sin}$ in $B_{cos}$, kjer ni vidnih razlik. V napaki (slika \ref{./rea/xd_napaka})  tudi ni posebnosti, razvoj napake v Fourierovo prikaže majšo spremembo (slika \ref{./rea/xd_fft}). V napaki se pojavi enosmerna komponenta.
\slikaeps{$B_{sin}$ in $B_{cos}$ pri simulacijah z realnim poljem pri 0,24 mm dinamične ekscentričnosti v smeri x}{./rea/xd_sincos}
\slikaeps{Napaka $\varepsilon$ pri simulacijah z realnim poljem pri 0,24 mm dinamične ekscentričnosti v smeri x}{./rea/xd_napaka}
\slikaeps{Amplitude harmonikov napake $\varepsilon$ razvite v Fourierovo vrsto pri simulacijah z realnim poljem pri 0,24 mm dinamične ekscentričnosti v smeri x}{./rea/xd_fft}
\newpage
\subsection{Sprememba signalov Hallovih sond ter napake v odvisnosti od dinamične ekscentričnosti v smeri x}
Spremembe amplitude osnovnega harmonika pri $B_{sin}$ in $B_{cos}$ ni bila pričakovana. Na napako zmanjšanje obeh amplitud ne vpliva. Enosmerni komponenti signalov (slika \ref{./rea/xd_sincos_off}) se zaradi diferencialnega zajema odštejeti. Fazna razlika signalov ostaja konstantna, vendar je opazno lezenje obeh signalov in posledično naraščanje enosmerne komponente napake.
\slikaeps{Amplituda osnovnega harmonika  $B_{sin}$ in $B_{cos}$ pri simulacijah z realnim poljem dinamične ekscentričnosti v smeri x. Poteka amplitud sta enaka in se na sliki prekrivata.}{./rea/xd_sincos_amp}
\slikaeps{Enosmerna komponenta $B_{sin}$ in $B_{cos}$ pri simulacijah z realnim poljem dinamične ekscentričnosti v smeri x. Zaradi diferencialnega zajema sta obe enosmerni komponenti enaki 0.}{./rea/xd_sincos_off}
\slikaeps{Fazni zamik $B_{sin}$ in $B_{cos}$ pri simulacijah z realnim poljem dinamične ekscentričnosti v smeri x glede na idealna signala $B_{sin}$ in $B_{cos}$. Poteka faznega zamika signalov $B_{sin}$ in $B_{cos}$ sta enaka.}{./rea/xd_sincos_phase}
Poteki zapisani s kubičnimi polinomi predstavijo enako spreminjanje signala $B_{sin}$ in $B_{cos}$.
\begin{eqnarray}
&Off_{sin}(\Delta x_d) =0\Delta x_d^{3}+0\Delta x_d^{2}+0\Delta x_d+0 \\
&A_{sin}(\Delta x_d) =2,56\cdot 10^{-2}\Delta x_d^{3}-1,33\cdot 10\Delta x_d^{2}-1,52\Delta x_d+6,83\cdot 10 \\                          
&\delta_{sin}(\Delta x_d) =5,79\Delta x_d^{3}-5,26\Delta x_d^{2}+7,87\cdot 10^{-1}\Delta x_d+6,23\cdot 10^{-3} \\                               
&Off_{cos}(\Delta x_d) =0\Delta x_d^{3}+0\Delta x_d^{2}+0\Delta x_d+0 \\
&A_{cos}(\Delta x_d) =2,56\cdot 10^{-2}\Delta x_d^{3}-1,33\cdot 10\Delta x_d^{2}-1,52\Delta x_d+6,83\cdot 10 \\                          
&\delta_{cos}(\Delta x_d) =5,79\Delta x_d^{3}-5,26\Delta x_d^{2}+7,87\cdot 10^{-1}\Delta x_d-1,57\cdot 10^{-2}
\end{eqnarray}
Potek posameznih harmonikov napake je viden na sliki \ref{./rea/xd_potek}. Iz potekov parametrov signalov $B_{sin}$ in $B_{cos}$, so pričakovane sprembe v enosmerni komponenti. Enosmerna komponenta je posledica spreminjanja začetne lege Hallovih sond.  Poteki so aproksimirani s kubičnimi polinomi.
\slikaeps{Potek amplitud posameznega harmonika napake $\varepsilon$ od dinamične ekscentričnosti v smeri x pri simulacijah z realnim poljem}{./rea/xd_potek}
\begin{eqnarray}
&C_0(\Delta x_d) =5,74\Delta x_d^{3}-5,22\Delta x_d^{2}+7,77\cdot 10^{-1}\Delta x_d-1,49\cdot 10^{-2} \\                               
&\begin{split}C_1(\Delta x_d) =1,51\cdot 10^{-14}\Delta x_d^{3}-1,72\cdot 10^{-14}\Delta x_d^{2}+5,51\cdot 10^{-15}\Delta x_d\\+5,84\cdot 10^{-15}\end{split} \\ 
&\begin{split}C_2(\Delta x_d) =-2,86\cdot 10^{-15}\Delta x_d^{3}+3,37\cdot 10^{-15}\Delta x_d^{2}-3,04\cdot 10^{-16}\Delta x_d\\+2,91\cdot 10^{-15}\end{split} \\
&\begin{split}C_3(\Delta x_d) =1,36\cdot 10^{-14}\Delta x_d^{3}-8,85\cdot 10^{-15}\Delta x_d^{2}+1,03\cdot 10^{-15}\Delta x_d\\+7,33\cdot 10^{-16}\end{split} \\ 
&C_4(\Delta x_d) =1,06\cdot 10\Delta x_d^{3}-6,51\Delta x_d^{2}+3,46\cdot 10^{-1}\Delta x_d+2,96\cdot 10^{-1}
\end{eqnarray}
\section{Dinamična ekscentričnost v smeri y}
V simulacijah z linearno aproksimacijo polja signala $B_{sin}$ in $B_{cos}$ ter posledično tudi napaka, ni bila odvisna od dinamične ekscentričnosti v smeri y. Tu je bila kljub temu opravljena simulacija. Rezultati so podobni dinamični ekscentričnosti v smeri x. Spremembe v $B_{sin}$ in $B_{cos}$ ni opaziti (slika \ref{./rea/yd_sincos}), prav tako ne v napaki (slika \ref{./rea/yd_sincos}). Razvoj v Fourierovo vrsto prikaže spremembo enosmerne komponente.
\slikaeps{$B_{sin}$ in $B_{cos}$ pri simulacijah z realnim poljem pri 0,20 mm dinamične ekscentričnosti v smeri y}{./rea/yd_sincos}
\slikaeps{Napaka $\varepsilon$ pri simulacijah z realnim poljem pri 0,20 mm dinamične ekscentričnosti v smeri y}{./rea/yd_napaka}
\slikaeps{Amplitude harmonikov napake $\varepsilon$ razvite v Fourierovo vrsto pri simulacijah z realnim poljem pri 0,20 mm dinamične ekscentričnosti v smeri y}{./rea/yd_fft}
\newpage
\subsection{Sprememba signalov Hallovih sond ter napake v odvisnosti od dinamične ekscentričnosti v smeri y}
Sprememba amplitude osnovnega harmonika od naraščanja ekscentričnosti pada (slika \ref{./rea/yd_sincos_amp}). Razlika amplitud ostaja nespremenjena. Enosmerni komponenti se odštejeti (slika \ref{./rea/yd_sincos_off}). Opaziti je tudi fazno lezenje obeh signalov (slika \ref{./rea/yd_sincos_phase}).  Poteki so aproksimirani s kubičnimi polinomi.

\begin{eqnarray}
&Off_{sin}(\Delta y_d) =0\Delta y_d^{3}+0\Delta y_d^{2}+0\Delta y_d+0 \\
&A_{sin}(\Delta y_d) =-6,30\Delta y_d^{3}-1,34\Delta y_d^{2}-4,05\cdot 10^{-1}\Delta y_d+6,83\cdot 10 \\                                 
&\delta_{sin}(\Delta y_d) =6,51\Delta y_d^{3}-5,38\Delta y_d^{2}+8,76\cdot 10^{-1}\Delta y_d-2,65\cdot 10^{-2} \\  
&Off_{cos}(\Delta y_d) =0\Delta y_d^{3}+0\Delta y_d^{2}+0\Delta y_d+0 \\
&A_{cos}(\Delta y_d) =-6,30\Delta y_d^{3}-1,34\Delta y_d^{2}-4,05\cdot 10^{-1}\Delta y_d+6,83\cdot 10 \\                                 
&\delta_{cos}(\Delta y_d) =6,51\Delta y_d^{3}-5,38\Delta y_d^{2}+8,76\cdot 10^{-1}\Delta y_d-2,65\cdot 10^{-2} 
\end{eqnarray}

Na sliki \ref{./rea/yd_potek} je prikazana odvisnost amplitud napake ob spreminjanju dinamične ekscentričnosti v smeri y. Napaka se izrazi le v enosmerni komponenti.
\begin{eqnarray}
&C_0(\Delta y_d) =6,53\Delta y_d^{3}-5,39\Delta y_d^{2}+8,75\cdot 10^{-1}\Delta y_d-2,59\cdot 10^{-2} \\                               
&\begin{split}C_1(\Delta y_d) =-3,87\cdot 10^{-14}\Delta y_d^{3}+2,37\cdot 10^{-14}\Delta y_d^{2}-3,01\cdot 10^{-15}\Delta y_d\\+6,30\cdot 10^{-15}\end{split} \\
&\begin{split}C_2(\Delta y_d) =6,08\cdot 10^{-14}\Delta y_d^{3}-4,53\cdot 10^{-14}\Delta y_d^{2}+8,57\cdot 10^{-15}\Delta y_d\\+2,79\cdot 10^{-15}\end{split} \\ 
&\begin{split}C_3(\Delta y_d) =-9,53\cdot 10^{-15}\Delta y_d^{3}+7,01\cdot 10^{-15}\Delta y_d^{2}-1,24\cdot 10^{-15}\Delta y_d\\+7,33\cdot 10^{-16}\end{split} \\
&C_4(\Delta y_d) =7,51\Delta y_d^{3}-1,78\Delta y_d^{2}-9,72\cdot 10^{-1}\Delta y_d+3,24\cdot 10^{-1}        
\end{eqnarray}
\slikaeps{Amplituda osnovnega harmonika  $B_{sin}$ in $B_{cos}$ pri simulacijah z realnim poljem dinamične ekscentričnosti v smeri y. Poteka amplitud sta enaka in se na sliki prekrivata.}{./rea/yd_sincos_amp}
\slikaeps{Enosmerna komponenta $B_{sin}$ in $B_{cos}$ pri simulacijah z realnim poljem dinamične ekscentričnosti v smeri y. Zaradi diferencialnega zajema sta obe enosmerni komponenti enaki 0.}{./rea/yd_sincos_off}
\slikaeps{Fazni zamik $B_{sin}$ in $B_{cos}$ pri simulacijah z realnim poljem dinamične ekscentričnosti v smeri y glede na idealna signala $B_{sin}$ in $B_{cos}$. Poteka faznega zamika signalov $B_{sin}$ in $B_{cos}$ sta enaka.}{./rea/yd_sincos_phase}
\slikaeps{Potek amplitud posameznega harmonika napake $\varepsilon$ od dinamične ekscentričnosti v smeri y pri simulacijah z realnim poljem}{./rea/yd_potek}

V tem poglavju so bile prikazane simulacije z uporabo realnega polja, ki ga merijo Hall-ove sonde. Rezultati podajo manjšo napako kot pri simulacijah z aproksimiranim linearnim magnetnim poljem. Z diferencialnim merjenjem, se odstrani enosmerno komponento v signalih $B_{sin}$ in $B_{cos}$.