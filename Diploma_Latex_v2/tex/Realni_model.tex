\chapter{Realni model magnetnega polja}
S poznavanjem točnejše funkcije polja, je točnejše predvidevanje potekov realnih $B_{sin}$ $B_{cos}$ in napake. Podjetje RLS,  je posredovalo rezultate Z-komponente gostote magnetnega pretoka 1,80 mm nad simuliranim magnetom. Simuliran magnet je imel 4 mm premer in 4 mm višine. definirano je imel remanenco 1050 mT in relativno permeabilnost 1.  Definicijsko območje  je 20x20 mm s korakom 0,02 mm (Slika \ref{Realno_polje}).
\slikaeps{Model Z- komponente gostote magnetnega pretoka uporabljen v simulacijah}{Realno_polje}
Simulacijski model je bil definiran enako kot model pri linearnem polju.
Polje, ki ga pomeri Hallova sonda ob zasuku ni bilo definirano oz. izpeljano analitično kot pri linearni aproksimaciji (\ref{equ:lin_potek_sin}) in (\ref{equ:lin_potek_cos}).  Prvo je bila izračunana relativna lokacija sonde glede na magnet. Nato se je vrednost Z-komponente gostote magnetnega pretoka  aproksimiralo s pretokom, definiranim v geometrijsko najbližji točki definicijskega območja. V tem poglavju so predstavljeni rezultati simulacij ekscentričnosti z uporabo realnega modela magnetnega polja.
\section{Brez napake}
Kljub idealni montaži, Hallovi sondi ne zajameti idealnih signalov. Signala nimata enakih amplitudi osnovnega harmonika, idealnega faznega zamika, vsebujeta tudi enosmerne komponente. Vsebujeta tudi višje harmonike (Slika \ref{./rea/00_sincos}). To se izrazi v napaki (slika \ref{./rea/00_napaka}). V napaki so špicei ( primer $220^{\circ}$ ), ki so posledica simulacije Z-komponente gostote  magnetnega pretoka. Napaka razvita v Fourierovo vrsto prikaže vrednosti amplitud prvih štirih harmonikov napake (slika \ref{./rea/00_fft}). Izrazit je četrti harmonik, ki je pričakovan po podatkovnih listih \cite{AM8192}.
\slikaeps{$B_{sin}$ in $B_{cos}$ pri simulacijah z realnim magnetnim poljem brez ekscentričnosti}{./rea/00_sincos}
\slikaeps{Napaka $\varepsilon$ pri simulacijah z realnim magnetnim poljem brez ekscentričnosti}{./rea/00_napaka}
\slikaeps{Amplitude harmonikov napake $\varepsilon$ razvite v Fourierovo vrsto pri simulacijah z realnim poljem brez ekscentričnosti}{./rea/00_fft}
\newpage
\section{Simulacija statične ekscentričnosti v smeri x-osi}
Po pričakovanjih se bo spremenila amplituda $B_{sin}$ in $B_{cos}$ signala ter zmanjšal njun fazni zamik (izraza (\ref{equ:lin_potek_sin}) in \ref{equ:lin_potek_cos}). Na sliki \ref{./rea/xs_sincos} se opazi višjo amplitudo signala $B_{cos}$, ter fazni zamik $B_{sin}$. Na sliki \ref{./rea/xs_napaka} je prikazana napaka $\varepsilon$. Oblika napake je pričakovana \cite{AM8192}. Razvoj napake v Fourierovo vrsto (slika \ref{./rea/xs_fft}) prikaže enako velikost enosmerne komponente in nižjo amplituda drugega harmonika, kot pri simulacijah z linearnim modelom.
\slikaeps{$B_{sin}$ in $B_{cos}$ pri simulacijah z realnim poljem pri 0,20 mm statične ekscentričnosti v smeri x}{./rea/xs_sincos}
\slikaeps{Napaka $\varepsilon$ pri simulacijah z realnim poljem pri 0,20 mm statične ekscentričnosti v smeri x}{./rea/xs_napaka}
\slikaeps{Amplitude harmonikov napake $\varepsilon$ razvite v Fourierovo vrsto pri simulacijah z realnim poljem pri 0,20mm statične ekscentričnosti v smeri x}{./rea/xs_fft}
\newpage
\subsection{Sprememba signalov Hallovih sond ter napake v odvisnosti od statične ekscentričnosti v smeri x}
Na sliki \ref{./rea/xs_sincos_amp} je prikazana sprememba amplitude prvega harmonika signalov $B_{sin}$ in $B_{cos}$. Pričakovano je hitrejše spreminjanje amplitude signala $B_{cos}$. Amplitude $B_{cos}$ z večjim odmikom narašča počasnje, saj Hall-ova sonda meri Z-komponento gostote magnetnega pretoka v okolici maksimuma.  Na sliki \ref{./rea/xs_sincos_off} je prikazan potek enosmerne komponente $B_{sin}$ in $B_{cos}$. Po rezultatih simulacij ni bilo pričakovano spreminjanje enosmerne komponente. Enosmerna komponenta se spreminja, vendar je njen vpliv praktično zanemarljiv. Slika \ref{./rea/xs_sincos_phase} prikazuje potek faznih signalov. Rezultat je bil pričakovan.
\slikaeps{Amplituda osnovnega harmonika  $B_{sin}$ in $B_{cos}$ pri simulacijah z realnim poljem statične ekscentričnosti v smeri x}{./rea/xs_sincos_amp}
\slikaeps{Enosmerna komponenta $B_{sin}$ in $B_{cos}$ pri simulacijah z realnim poljem statične ekscentričnosti v smeri x}{./rea/xs_sincos_off}
\slikaeps{Fazni zamik $B_{sin}$ in $B_{cos}$ pri simulacijah z realnim poljem statične ekscentričnosti v smeri x glede na idealna signala $B_{sin}$ in $B_{cos}$}{./rea/xs_sincos_phase}
\newpage
Poteke se aproksimira s kubičnimi polinomi. Enačbe potrdijo konstantnost amplitude prvega harmonika in enosmerne komponente signala $B_{sin}$, ter linearno naraščanje faznega zamika.
Na sliki \ref{./rea/xs_potek} so prikazani poteki amplitud posameznega harmonika ob spreminjanja statične ekscentričnosti v smeri x. Enosmerna komponenta in amplituda drugega harmonika naraščata linearno, ostali harmoniki ohranjajo konstantno amplitudo (\ref{real_xs_C0})-(\ref{real_xs_C4}).
\begin{eqnarray}
&A_{sin}(\Delta x_s) = -1,21            \Delta x_s^3+5,40\Delta x_s^2-3,07\cdot 10^{-2}\Delta x_s+34,1                \\      
&Off_{sin}(\Delta x_s) = -0,347\Delta x_s^3+0,388\Delta x_s^2-2,17\cdot 10^{-2}\Delta x_s^1+9,24\cdot 10^{-2}            \\ 
&\delta_{sin}(\Delta x_s) = -5,73            \Delta x_s^3+4,18\cdot 10^{-1}\Delta x_s^2+38,1    \Delta x_s+4,78\cdot 10^{-3}            \\
&A_{cos}(\Delta x_s) = 12,0   \Delta x_s^3-19,4\Delta x_s^2+16,4\Delta x_s+34,1           \\       
&Off_{cos}(\Delta x_s) = 9,67            \Delta x_s^3-7,21\Delta x_s^2+1,25\Delta x_s+8,47\cdot 10^{-2}            \\    
&\delta_{cos}(\Delta x_s) = 12,9    \Delta x_s^3-8,25            \Delta x_s^2+9,90\cdot 10^{-1}\Delta x_s-4,74\cdot 10^{-3}  
\end{eqnarray}
\slikaeps{Potek amplitud posameznega harmonika napake $\varepsilon$ od statične ekscentričnosti v smeri x pri simulacijah z realnim poljem}{./rea/xs_potek}
\begin{eqnarray}
\label{real_xs_C0}
&C_0(\Delta x_s) =6,41\Delta x_s^{3}-6,66\Delta x_s^{2}+19,4\Delta x_s+4,39\cdot 10^{-3} \\     
&C_1(\Delta x_s) =3,66\Delta x_s^{3}-2,92\Delta x_s^{2}+6,43\cdot 10^{-1}\Delta x_s+2,44\cdot 10^{-1} \\
&C_2(\Delta x_s) =-3,77\Delta x_s^{3}-3,17\Delta x_s^{2}+21,8\Delta x_s-9,98\cdot 10^{-3} \\    
&C_3(\Delta x_s) =2,97\Delta x_s^{3}-1,58\Delta x_s^{2}+6,87\cdot 10^{-2}\Delta x_s+1,22\cdot 10^{-1} \\
\label{real_xs_C4}
&C_4(\Delta x_s) =4,66\Delta x_s^{3}+1,12\Delta x_s^{2}-4,23\cdot 10^{-1}\Delta x_s+3,07\cdot 10^{-1} 
\end{eqnarray}

Enosmerna komponenta narašča enako kot pri simulacijah z linearnim modelom. Drugi harmonik narašča nekoliko počasneje.
\section{Simulacija statične ekscentričnosti v smeri y-osi}
Tako kot pri statični ekscentričnosti v smeri x, se tudi na $B_{sin}$ in $B_{cos}$ signalih ob povzročeni ekscentričnosti opazi predvidene spremembe (slika \ref{./rea/ys_sincos}).  Amplituda $B_{sin}$ se je povišala, spremenil se je tudi fazni zamik $B_{cos}$.  Napaka $\varepsilon$ (slika \ref{./rea/ys_napaka}) je enake oblike kot je bila pri simulacijah statične ekscentričnosti v smeri x (slika \ref{./rea/xs_napaka}). Napaka ima le negativno enosmerno komponento. Razvoj napake v Fourierovo vrsto (slika \ref{./rea/xs_fft}) potrdi pričakovanja.
\slikaeps{$B_{sin}$ in $B_{cos}$ pri simulacijah z realnim poljem pri 0,20 mm statične ekscentričnosti v smeri y}{./rea/ys_sincos}
\slikaeps{Napaka $\varepsilon$ pri simulacijah z realnim poljem pri 0,20 mm statične ekscentričnosti v smeri y}{./rea/ys_napaka}
\slikaeps{Amplitude harmonikov napake $\varepsilon$ razvite v Fourierovo vrsto pri simulacijah z realnim poljem pri 0,20 mm statične ekscentričnosti v smeri y}{./rea/ys_fft}
\newpage
\subsection{Sprememba signalov Hallovih sond ter napake v odvisnosti od statične ekscentričnosti v smeri y}
Potek amplitude osnovnega harmonika $B_{sin}$ (slika \ref{./rea/ys_sincos_amp}) se spreminja kot se je spreminjala amplituda osnovnega hamronika $B_{cos}$ pri simulacijah statične ekscentričnosti v smeri x. Sprememba $B_{cos}$ od statične ekscentričnosti v smeri y je proti spremembi $B_{sin}$ zanemarljiva. Enosmerena komponenta (slika \ref{./rea/ys_sincos_off}) pri $B_{sin}$ se spreminja enako, kot enosmerna komponenta $B_{cos}$ pri statični ekscentričnosti v smeri x (slika \ref{./rea/xs_sincos_off}).  Fazni zamik signalov se spreminja po pričakovanjih (slika \ref{./rea/ys_sincos_phase}). Fazni zamik se manjša, pri čemer pada fazni zamik $B_{cos}$ signala.
Poteki zapisani s kubičnimi polinomi.
\begin{eqnarray}
&A_{sin}( \Delta y_s) = 12,0 \Delta y_s^3-19,4\Delta y_s^2+16,4\Delta y_s+34,1            \\       
&Off_{sin} ( \Delta y_s)= 9,67            \Delta y_s^3-7,21\Delta y_s^2+1,25\Delta y_s+8,47\cdot 10^{-2}            \\    
&\delta_{sin}( \Delta y_s) = 12,9 \Delta y_s^3-8,25            \Delta y_s^2+9,90\cdot 10^{-1}\Delta y_s-4,74\cdot 10^{-3}            \\
&A_{cos}( \Delta y_s) = -1,19            \Delta y_s^3+5,39\Delta y_s^2-2,85\cdot 10^{-2}\Delta y_s+34,1              \\      
&Off_{cos} ( \Delta y_s)= -0,348\Delta y_s^3+0,389\Delta y_s^2-0,0223\Delta y_s+9,25\cdot 10^{-2}            \\ 
&\delta_{cos}( \Delta y_s) = 3,69            \Delta y_s^3+1,22\Delta y_s^2-38,4    \Delta y_s+7,68\cdot 10^{-3} 
\end{eqnarray}
\slikaeps{Amplituda osnovnega harmonika  $B_{sin}$ in $B_{cos}$ pri simulacijah z realnim poljem statične ekscentričnosti v smeri y}{./rea/ys_sincos_amp}
\slikaeps{Enosmerna komponenta $B_{sin}$ in $B_{cos}$ pri simulacijah z realnim poljem statične ekscentričnosti v smeri y}{./rea/ys_sincos_off}
\slikaeps{Fazni zamik $B_{sin}$ in $B_{cos}$ pri simulacijah z realnim poljem statične ekscentričnosti v smeri y glede na idealna signala $B_{sin}$ in $B_{cos}$}{./rea/ys_sincos_phase}

Enačbe prikazujejo podobne poteke kot poteki pri statični ekscentričnosti v smeri x. Poteki, ki so veljali za $B_{sin}$ tu veljajo za $B_{cos}$ in obratno. Razlikuje se le pri predzanku faznega zamika $\varphi_{cos}$. 
Posledično to vpliva na posamezne harmonike napake. Po pričakovanju je enosmerna komponenta negativana, drugi harmonik narašča počasneje kot je pri simulacijah z linearnim poljem, kar je pričakovano. Poteki aproksimirani s kubičnimi polinomi so podobni aprksimacijam amplitud posameznih harmonikov napake statične ekscentričnosti v smeri x.
\slikaeps{Potek amplitud posameznega harmonika napake $\varepsilon$ od statične ekscentričnosti v smeri y pri simulacijah z realnim poljem}{./rea/ys_potek}
\begin{eqnarray}
&C_0( \Delta y_s) =5,60\Delta y_s^{3}-9,39\cdot 10^{-1}\Delta y_s^{2}-18,5\Delta y_s-2,28\cdot 10^{-3} \\
&C_1 ( \Delta y_s)=5,15\Delta y_s^{3}-3,14\Delta y_s^{2}+2,37\cdot 10^{-1}\Delta y_s+2,37\cdot 10^{-1} \\        
&C_2 ( \Delta y_s)=4,47\Delta y_s^{3}-9,87\Delta y_s^{2}+23,2\Delta y_s-1,57\cdot 10^{-2} \\             
&C_3( \Delta y_s) =-7,42\Delta y_s^{3}+6,17\Delta y_s^{2}-1,41\Delta y_s+1,44\cdot 10^{-1} \\                    
&C_4 ( \Delta y_s)=-3,31\Delta y_s^{3}+6,83\Delta y_s^{2}-1,44\Delta y_s+3,21\cdot 10^{-1}         
\end{eqnarray}
\section{Dinamična ekscentričnost v smeri x}
Vpliv dinamične ekscentričnosti v $B_{sin}$ in $B_{cos}$ bo viden v enosmerni komponenti. Na sliki  \ref{./rea/xd_sincos} sta $B_{sin}$ in $B_{cos}$ , kjer je vidna enosmerna komponenta v obeh signalih. Posledično se v napaki (slika \ref{./rea/xd_napaka})  izrazi  najizraziteje prvi harmonik. Razvoj napake v Fourierovo vrsto potrdi pričakovanja (slika \ref{./rea/xd_fft}).  
\slikaeps{$B_{sin}$ in $B_{cos}$ pri simulacijah z realnim poljem pri 0,24 mm dinamične ekscentričnosti v smeri x}{./rea/xd_sincos}
\slikaeps{Napaka $\varepsilon$ pri simulacijah z realnim poljem pri 0,24 mm dinamične ekscentričnosti v smeri x}{./rea/xd_napaka}
\slikaeps{Amplitude harmonikov napake $\varepsilon$ razvite v Fourierovo vrsto pri simulacijah z realnim poljem pri 0,24 mm dinamične ekscentričnosti v smeri x}{./rea/xd_fft}
\newpage
\subsection{Sprememba signalov Hallovih sond ter napake v odvisnosti od dinamične ekscentričnosti v smeri x}
Spremembe amplitude osnovnega harmonika pri $B_{sin}$ in $B_{cos}$ ni bila pričakovana. na napako to nebo vplivalo, saj se obe spreminjati enako. Enosmerni komponenti signalov (slika \ref{./rea/xd_sincos_off}) se spreminjati po pričakovanjih linearno. Fazna razlika signalov ostaja konstantna, vendar je opazno lezenje obeh signalov in posledično naraščanje enosmerne komponente napake.
\slikaeps{Amplituda osnovnega harmonika  $B_{sin}$ in $B_{cos}$ pri simulacijah z realnim poljem dinamične ekscentričnosti v smeri x}{./rea/xd_sincos_amp}
\slikaeps{Enosmerna komponenta $B_{sin}$ in $B_{cos}$ pri simulacijah z realnim poljem dinamične ekscentričnosti v smeri x}{./rea/xd_sincos_off}
\slikaeps{Fazni zamik $B_{sin}$ in $B_{cos}$ pri simulacijah z realnim poljem dinamične ekscentričnosti v smeri x glede na idealna signala $B_{sin}$ in $B_{cos}$}{./rea/xd_sincos_phase}
Poteki zapisani s kubičnimi polinomi predstavijo enako spreminjanje signala $B_{sin}$ in $B_{cos}$.
\begin{eqnarray}
&A_{sin}(\Delta x_d)=1,28\cdot10^{-2}\Delta x_d^3-6,67\Delta x_d^2-7,59\cdot10^{-1}\Delta x_d+34,2\\
&Off_{sin}(\Delta x_d)=2,06\Delta x_d^3+5,62\cdot10^{-1}\Delta x_d^2-18,8\Delta x_d+9,92\cdot10^{-2}\\
&\delta_{sin}(\Delta x_d)=5,79\Delta x_d^3-5,26\Delta x_d^2+7,87\cdot10^{-1}\Delta x_d-1,57\cdot10^{-2}\\
&A_{cos}(\Delta x_d)=1,28\cdot10^{-2}\Delta x_d^3-6,67\Delta x_d^2-7,59\cdot10^{-1}\Delta x_d+34,2\\
&Off_{cos}(\Delta x_d)=2,06\Delta x_d^3+5,62\cdot10^{-1}\Delta x_d^2-18,8\Delta x_d+9,92\cdot10^{-2}\\
&\delta_{cos}(\Delta x_d)=5,79\Delta x_d^3-5,26\Delta x_d^2+7,87\cdot10^{-1}\Delta x_d-1,57\cdot10^{-2}
\end{eqnarray}
Potek posameznih harmonikov napake je viden na sliki \ref{./rea/xd_potek}. Po pričakovanjih najhitreje narašča prvi harmonik napake. Ostali harmoniki so zanemarljivi. Poteki so aproksimirani s kubičnimi polinomi.
\slikaeps{Potek amplitud posameznega harmonika napake $\varepsilon$ od dinamične ekscentričnosti v smeri x pri simulacijah z realnim poljem}{./rea/xd_potek}
\begin{eqnarray}
&C_0  (\Delta x_d)=5,43\Delta x_d^{3}-4,87\Delta x_d^{2}+6,65\cdot 10^{-1}\Delta x_d-1,48\cdot 10^{-2} \\            
&C_1  (\Delta x_d)=-10,3\Delta x_d^{3}+12,0\Delta x_d^{2}+46,1\Delta x_d-8,97\cdot 10^{-2} \\
&C_2 (\Delta x_d) =-1,44\Delta x_d^{3}+17,5\Delta x_d^{2}-2,23\cdot 10^{-1}\Delta x_d-4,26\cdot 10^{-4} \\   
&C_3 (\Delta x_d) =1,67\Delta x_d^{3}+7,64\Delta x_d^{2}+3,03\Delta x_d+4,57\cdot 10^{-2} \\                         
&C_4  (\Delta x_d)=4,76\Delta x_d^{3}+4,22\Delta x_d^{2}-1,38\Delta x_d+3,33\cdot 10^{-1}  
\end{eqnarray}
\section{Dinamična ekscentričnost v smeri y}
V simulacijah z linearno aproksimacijo polja signala $B_{sin}$ in $B_{cos}$ ter posledično tudi napaka, ni bila odvisna od dinamične ekscentričnosti v smeri y. Kljub temu je bila opravljena simulacija. Rezultati so razlikujejo od pričakovanj. Spremembe v $B_{sin}$ in $B_{cos}$ ni opaziti (slika \ref{./rea/yd_sincos}), vendar se v napaki  pojavi prvi in tretji harmonik (slika \ref{./rea/yd_sincos}). Razvoj v Fourierovo vrsto potrdi izstopanje omenjenih harmonikov.
\slikaeps{$B_{sin}$ in $B_{cos}$ pri simulacijah z realnim poljem pri 0,20 mm dinamične ekscentričnosti v smeri y}{./rea/yd_sincos}
\slikaeps{Napaka $\varepsilon$ pri simulacijah z realnim poljem pri 0,20 mm dinamične ekscentričnosti v smeri y}{./rea/yd_napaka}
\slikaeps{Amplitude harmonikov napake $\varepsilon$ razvite v Fourierovo vrsto pri simulacijah z realnim poljem pri 0,20 mm dinamične ekscentričnosti v smeri y}{./rea/yd_fft}
\newpage
\subsection{Sprememba signalov Hallovih sond ter napake v odvisnosti od dinamične ekscentričnosti v smeri y}
Sprememba amplitude osnovnega harmonika od naraščanja ekscentričnosti pada (slika \ref{./rea/yd_sincos_amp}). Razlika amplitud ostaja nespremenjena. V obeh signalih se enako spreminjati enosmerni komponenti, ki sta zanemarljivi (slika \ref{./rea/yd_sincos_off}). Opaziti je tudi fazno lezenje obeh signalov, vendar je zanemarljivo (slika \ref{./rea/yd_sincos_phase}).  Poteki so aproksimirani s kubičnimi polinomi.
\slikaeps{Amplituda osnovnega harmonika  $B_{sin}$ in $B_{cos}$ pri simulacijah z realnim poljem dinamične ekscentričnosti v smeri y}{./rea/yd_sincos_amp}
\slikaeps{Enosmerna komponenta $B_{sin}$ in $B_{cos}$ pri simulacijah z realnim poljem dinamične ekscentričnosti v smeri y}{./rea/yd_sincos_off}
\slikaeps{Fazni zamik $B_{sin}$ in $B_{cos}$ pri simulacijah z realnim poljem dinamične ekscentričnosti v smeri y glede na idealna signala $B_{sin}$ in $B_{cos}$}{./rea/yd_sincos_phase}
\begin{eqnarray}
&A_{sin} (\Delta y_d) = -3,15            \Delta y_d^3-6,71\cdot 10^{-1}\Delta y_d^2-2,02\cdot 10^{-1}\Delta y_d+34,2                \\     
&Off_{sin} (\Delta y_d) = -0,895\Delta y_d^3+0,676\Delta y_d^2-0,0463\Delta y_d+0,0895 \\ 
&\delta_{sin} (\Delta y_d) = 6,51            \Delta y_d^3-5,38\cdot 10^{0}\Delta y_d^2+8,76\cdot 10^{-1}\Delta y_d-2,65\cdot 10^{-2}            \\
&A_{cos} (\Delta y_d) = -3,15            \Delta y_d^3-6,71\cdot 10^{-1}\Delta y_d^2-2,02\cdot 10^{-1}\Delta y_d+34,2                \\     
&Off_{cos} (\Delta y_d) = -0,895\Delta y_d^3+0,676\Delta y_d^2-0,0463\Delta y_d+0,0895 \\ 
&\delta_{cos} (\Delta y_d) = 6,51            \Delta y_d^3-5,38\Delta y_d^2+8,76\cdot 10^{-1}\Delta y_d^1-2,65\cdot 10^{-2}
\end{eqnarray}
Na sliki \ref{./rea/yd_potek} je prikazana odvisnost amplitud napake ob spreminjanju dinamične ekscentričnosti v smeri y. Napaka, se po pričakovanjih najbolj izrazi s prvim in tretjim harmonikom. Oblika napake ni posledica spremembe amplitude osnovnega harmonika, enosmerne komponente ali spremembe faznega zamika v $B_{sin}$ in $B_{cos}$. Naraščanje prvega in tretjega harmonika je posledica vpliva drugega harmonika, ki se pojavi v $B_{sin}$ in $B_{cos}$. Drugi harmonik v $B_{sin}$ in $B_{cos}$ se pojavi zaradi magnetnega polja, kar v tem delu ni raziskano.
\slikaeps{Potek amplitud posameznega harmonika napake $\varepsilon$ od dinamične ekscentričnosti v smeri y pri simulacijah z realnim poljem}{./rea/yd_potek}
\begin{eqnarray}
&C_0 (\Delta y_d) =6,68\Delta y_d^{3}-5,51\Delta y_d^{2}+8,88\cdot 10^{-1}\Delta y_d-2,68\cdot 10^{-2} \\                          
&C_1 (\Delta y_d) =-1,13\cdot 10\Delta y_d^{3}+1,05\cdot 10\Delta y_d^{2}+2,52\Delta y_d+1,57\cdot 10^{-1} \\                      
&C_2 (\Delta y_d) =0,113\Delta y_d^{3}+0,191\Delta y_d^{2}-4,44\cdot 10^{-3}\Delta y_d+6,00\cdot 10^{-4} \\
&C_3  (\Delta y_d)=-3,87\Delta y_d^{3}+4,82\Delta y_d^{2}+3,95\Delta y_d+3,55\cdot 10^{-2} \\                                      
&C_4 (\Delta y_d) =6,42\Delta y_d^{3}-1,09\Delta y_d^{2}-1,08\Delta y_d+3,26\cdot 10^{-1}       
\end{eqnarray}

V tem poglavju so bile prikazane simulacije z uporabo realnega polja, ki ga merijo Hall-ove sonde. Rezultati imajo manjšo napako kot pri simulacijah z aproksimiranim linearnim magnetnim poljem. Opaziti je bilo manjši fazni zamik obeh signalov $B_{sin}$ in $B_{cos}$ pri dinamični ekscentričnosti, kar bi bilo smiselno pri meritvah podrobneje opazovati. Na koncu, pri dinamični ekscentričnosti v smeri y je prikazano tudi, da se v zajetem polju pojavijo tudi višji harmoniki, ki še dodatno ustvarijo napako.