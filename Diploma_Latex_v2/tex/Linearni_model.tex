\chapter{Linearni model}
V tem poglavju so prikazani rezultati simulacij z aproksimiranim poljem (\ref{equ:lin_polje}). 
\begin{equation}
\label{equ:lin_polje}
B(x,y)= x
\end{equation}
Prikazan je potek napake pri različnih izmikih, ter potek amplitud posameznih haarmonikov napake v odvisnosti od ekscentričnosti.
Hall-ovi sondi sta postavljeni na krožnico z radijem 2,4 mm \cite{AM8192}.
\section{Brez ekscentričnosti}
Signala $sin$ in $cos$ pomerjena v stanju brez ekscentričnosti imata enaki amplitudi in sta fazno zamaknjena za $90^{\circ}$. Napaka $\varepsilon$, ki se pojavi pri izračunu je le numerična napaka funkcije atan2d (Slika \ref{./LIN/yd_napaka}). Numerična napaka je proti pričakovani napaki zaradi ekscnetričnosti zanemarljiva.
\slikaeps{$sin$ in $cos$ pri simulacijah z linearnim magnetnim poljem brez ekscentričnosti}{./LIN/yd_sincos}
\slikaeps{Napaka $\varepsilon$ pri simulacijah z linearnim magnetnim poljem brez ekscentričnosti}{./LIN/yd_napaka}
%\newpage
\section{Simulacija statične ekscentričnosti v smeri x-osi}
Po pričakovanjih se bo povišala amplituda $sin$ in $cos$ signala ter zmanjšal njun fazni zamik (\ref{equ:Bx_stat})  (\ref{equ:By_stat}). Po pričakovanjih najbolj izstopata enosmerna komponenta (harmonik 0) in drugi harmonik.
\slikaeps{$sin$ in $cos$ pri simulacijah z linearnim poljem pri 0,24 mm statične ekscentričnosti v smeri x}{./LIN/xs_sincos}
%Napaka $\varepsilon$  je prikazana na sliki \ref{./LIN/xs_napaka}.
\slikaeps{Napaka $\varepsilon$ pri simulacijah z linearnim poljem pri 0,24 mm statične ekscentričnosti v smeri x}{./LIN/xs_napaka}
%Napako razvijmo v Fourierovo vrsto in pridobimo amplitude posameznih harmonikov napake(Slika \ref{./LIN/xs_fft}).
\slikaeps{Amplitude harmonikov napake $\varepsilon$ razvite v Fourierovo vrsto pri simulacijah z linearnim poljem pri 0,24 mm statične ekscentričnosti v smeri x}{./LIN/xs_fft}
%\newpage
\subsection{Sprememba $sin$, $cos$ ter napake v odvisnosti od $\Delta x_s$}
Na sliki \ref{./LIN/xs_sincos_amp} je prikazana sprememba amplitude prvega harmonika signalov $sin$ in $cos$. Razvidno iz (\ref{equ:Bx_stat})  (\ref{equ:By_stat}) se linearno narašča amplituda $cos$. Slika \ref{./LIN/xs_sincos_off} prikazuje enosmerni komponenti, ki od statične ekscentričnosti nista odvisni. Slika \ref{./LIN/xs_sincos_phase} prikazuje fazni zamik signalov glede na njuno idealno poravnavo. Po (\ref{equ:By_stat}) je pričakovano spreminjanje faze $sin$.

Spreminjanje amplitude prvega harmonika, enosmerne kompoonente in faznega zamika $sin$ in $cos$ signalov je opisano z (\ref{equ:Bx_stat}) in (\ref{equ:By_stat}). Tu so poteki razviti v Taylorjevo vrsto do tretje stopnje.
 %\ref{./LIN/xs_sincos_amp}. Amplituda $cos$ se spreminja hitreje, kot amplituda $sin$ .
\slikaeps{Amplituda osnovnega harmonika  $sin$ in $cos$ pri simulacijah z linearnim poljem statične ekscentričnosti v smeri x}{./LIN/xs_sincos_amp}
\slikaeps{Enosmerna komponenta $sin$ in $cos$ pri simulacijah z linearnim poljem statične ekscentričnosti v smeri x}{./LIN/xs_sincos_off}
\slikaeps{Fazni zamik $sin$ in $cos$ pri simulacijah z linearnim poljem statične ekscentričnosti v smeri x glede na idealna signala $sin$ in $cos$}{./LIN/xs_sincos_phase}
\begin{eqnarray}
\label{analog_lin_xs}
&A_{sin} = 2,08 \cdot 10^{-1} \Delta x_s^2+2,4\\
&Off_{sin} = 0 \\
&\delta_{sin} =-1,38 \Delta x_s^3+ 23,9 \Delta x_s  \\
&A_{cos} = \Delta x_s+2,4\\
&Off_{cos} = 0 \\
&\delta_{cos} = 0
\end{eqnarray}
\newpage
Spremembi signalov $sin$ in $cos$ se odrazita tudi pri izračunu kota $\varphi$ in napake $\varepsilon$.
Na sliki \ref{./LIN/xs_potek} vidimo odvisnost amplitud posameznega harmonika od spreminjanja statične ekscentričnosti v smeri x. Poteke s slike \ref{./LIN/xs_potek} aproksimiramo s polinomi.
\slikaeps{Potek amplitud posameznega harmonika napake $\varepsilon$ od statične ekscentričnosti v smeri x pri simulacijah z linearnim poljem}{./LIN/xs_potek}

%predstavimo enako kot (\ref{vrsta:xs}).
\begin{eqnarray}
\label{nap_lin_xs}
&C_0 =3,35\cdot 10^{-1}\Delta x_s^{3}-2,48\Delta x_s^{2}+11,9\Delta x_s+1,23\cdot 10^{-5} \\
&C_1 =5,56\cdot 10^{-4}\Delta x_s^{3}-2,00\cdot 10^{-3}\Delta x_s^{2}+4,34\cdot 10^{-3}\Delta x_s+7,67\cdot 10^{-8} \\
&C_2 =4,13\cdot 10^{-1}\Delta x_s^{3}-3,53\Delta x_s^{2}+16,9\Delta x_s-2,31\cdot 10^{-5} \\
&C_3 =-2,17\cdot 10^{-4}\Delta x_s^{3}+2,57\cdot 10^{-4}\Delta x_s^{2}+0,0042\Delta x_s+4,51\cdot 10^{-8} \\
&C_4 =-8,27\cdot 10^{-1}\Delta x_s^{3}+2,42\Delta x_s^{2}+8,08\cdot 10^{-3}\Delta x_s-1,60\cdot 10^{-4}
\end{eqnarray}

Za primerjavo, s dodane tudi enačbe potekov amplitud posameznega harmonika razvitega v Taylorjevo vrsto, katere sledijo iz (\ref{vrsta:xs}):

\begin{eqnarray}
&C_0 =3,45\cdot 10^{-1}\Delta x_s^{3}-2,49\Delta x_s^{2}+11,9\Delta x_s \\
&C_1 = 0\\
&C_2 =3,66\cdot 10^{-1}\Delta x_s^{3}-3,51\Delta x_s^{2}+16,9\Delta x_s\\
&C_3 = 0\\
&C_4 =-1,04\Delta x_s^{3}+2,49\Delta x_s^{2}
\end{eqnarray}

Rezultati so pričakovani. Enosmerna komponenta in amplituda prvega harmonika naračšata linearno, četrti harmonik narašča s kvadratom ekscentričnosti, lihi harmoniki, so zanemarljivi.



\section{Simulacija statične ekscentričnosti v smeri y-osi}

Pričakovani so podobni rezultati kot pri statični ekscentričnosti v smeri x, le da bo tu hitreje naraščala amplituda $sin$ in spreminjal se bo fazni zamik $cos$.
 
\slikaeps{$sin$ in $cos$ pri simulacijah z linearnim poljem pri 0,24 mm statične ekscentričnosti v smeri y}{./LIN/ys_sincos}

Napaka je prikazana na sliki \ref{./LIN/ys_napaka}. Sestavlja jo negativna enosmerna komponenta in izrazit drugi harmonik. Iz napake razvite v vrsto (\ref{./LIN/ys_fft}) je vidna enaka amplituda drugega harmonika, kot pri ekscentričnosti v smeri x. Enosmerna komponenta se razlikuje v predznaku.
\slikaeps{Napaka $\varepsilon$ pri simulacijah z linearnim poljem pri 0,24 mm statične ekscentričnosti v smeri y}{./LIN/ys_napaka}
\slikaeps{Amplitude harmonikov napake $\varepsilon$ razvite v Fourierovo vrsto pri simulacijah z linearnim poljem pri 0,24 mm statične ekscentričnosti v smeri y}{./LIN/ys_fft}
\newpage
\subsection{Sprememba $sin$, $cos$ ter napake od $\Delta y_s$}

Potek hitrejšega spreminjanja amplitude $sin$ je pričakovan. Enosmerna komponenta signalov se prav tako ni spremenila. Fazni zamik signala $cos$ se je zmanjševal, posledično tudi fazna razlika med signaloma. Poteki so opisani s kubičnimi polinomi. Na sliki \ref{./LIN/ys_potek} so prikazani poteki amplitud posameznih harmonikov v odvisnosti od statične ekscentričnosti v smeri y. Poteki so aproksimirani z kubičnimi polinomi. Potek amplitud harmonikov je enak potekom simuliranih s statično ekscentričnostjo v smeri x, razlikuje se enosmerna komponenta z nasprotnim predznakom.
\slikaeps{Amplituda osnovnega harmonika signalov $sin$ in $cos$ pri simulacijah z linearnim poljem statične ekscentričnosti v smeri y}{./LIN/ys_sincos_amp}
\slikaeps{Enosmerna komponenta $sin$ in $cos$ pri simulacijah z linearnim poljem statične ekscentričnosti v smeri y}{./LIN/ys_sincos_off}
\slikaeps{Fazni zamik $sin$ in $cos$ pri simulacijah z linearnim poljem statične ekscentričnosti v smeri y glede na idealna signala $sin$ in $cos$}{./LIN/ys_sincos_phase}
\begin{eqnarray}
\label{analog_lin_ys}
 &A_{sin} = \Delta x_s+2,4\\
 &Off_{sin} = 0 \\
 &\delta_{sin} =0 \\
 &A_{cos} = 2,08 \cdot 10^{-1} \Delta x_s^2+2,4\\
 &Off_{cos} = 0 \\
 &\delta_{cos} = 1,38 \Delta x_s^3- 23,9 \Delta x_s
\end{eqnarray}

\slikaeps{Potek amplitud posameznega harmonika napake $\varepsilon$ od statične ekscentričnosti v smeri y pri simulacijah z linearnim poljem}{./LIN/ys_potek}


\begin{eqnarray}
\label{nap_lin_ys}
&C_0 =-3,35\cdot 10^{-1}\Delta y_s^{3}+2,48\Delta y_s^{2}-11,9\Delta y_s-1,22\cdot 10^{-5}\\                  
&C_1 =1,09\cdot 10^{-4}\Delta y_s^{3}-8,69\cdot 10^{-4}\Delta y_s^{2}+0,00434\Delta y_s+7,62\cdot 10^{-10} \\
&C_2 =4,12\cdot 10^{-1}\Delta y_s^{3}-3,53\Delta y_s^{2}+1,69\cdot 10\Delta y_s-2,31\cdot 10^{-5} \\                   
&C_3 =2,43\cdot 10^{-4}\Delta y_s^{3}-0,00130\Delta y_s^{2}+0,00420\Delta y_s+1,83\cdot 10^{-8} \\ 
&C_4 =-8,26\cdot 10^{-1}\Delta y_s^{3}+2,42\Delta y_s^{2}+6,13\cdot 10^{-3}\Delta y_s-1,60\cdot 10^{-4}            
\end{eqnarray}

\section{Dinamična ekscentričnost v smeri x osi}
%
Dinamična ekscentričnost pričakovano povzroči v $sin$ in $cos$ enosmerno komponento (Slika \ref{./LIN/xd_sincos}).
\slikaeps{$sin$ in $cos$ pri simulacijah z linearnim poljem pri 0,24 mm mm dinamične ekscentričnosti v smeri x}{./LIN/xd_sincos}
 Na sliki \ref{./LIN/xd_napaka} je vidna napaka v obliki prvega harmonika, kar je bilo pričakovati.
\slikaeps{Napaka $\varepsilon$ pri simulacijah z linearnim poljem pri 0,24 mm dinamične ekscentričnosti v smeri x}{./LIN/xd_napaka}
Z razvojem napake v Fourierovo vrsto je nejizrazitejši prvi harmonik, enosmerna komponenta je nič (slika \ref{./LIN/xd_fft}).
\slikaeps{Amplitude harmonikov napake $\varepsilon$ razvite v Fourierovo vrsto pri simulacijah z linearnim poljem pri 0,24 mm dinamične ekscentričnosti v smeri  x}{./LIN/xd_fft}


\newpage
\subsection{Sprememba $sin$, $cos$ ter napake od $\Delta x_d$}
Dinamična ekscentričnost vpliva na enosmerni komponenti $sin$ in $cos$ (slika \ref{./LIN/xd_sincos_off}).
\slikaeps{Amplituda osnovnega harmonika  $sin$ in $cos$ pri simulacijah z linearnim poljem dinamične ekscentričnosti v smeri x}{./LIN/xd_sincos_amp}
\slikaeps{Enosmerna komponenta $sin$ in $cos$ pri simulacijah z linearnim poljem dinamične ekscentričnosti v smeri x}{./LIN/xd_sincos_off}
\slikaeps{Fazni zamik $sin$ in $cos$ pri simulacijah z linearnim poljem dinamične ekscentričnosti v smeri x glede na idealna signala $sin$ in $cos$}{./LIN/xd_sincos_phase}
Z aproksimacijo posameznega parametra $sin$ in $cos$ s kubičnim polinomom sta od dinamične ekscentričnosti odvisni le enosmerni komponenti.
\begin{eqnarray}
\label{analog_lin_xd}
&A_{sin} = 2,4\\
&Off_{sin} = -\Delta x_d \\
&\delta_{sin} =0 \\
&A_{cos} = 2,4\\
&Off_{cos} = -\Delta x_d\\
&\delta_{cos} = 0
\end{eqnarray}
Slika \ref{./LIN/xd_potek} prikazuje odvisnost amplitud napake od spreminjanja dinamične ekscentričnosti v smeri x. V napaki, se po pričakovanjih linearno povečuje prvi harmonik (\ref{vrsta_sincosoff}).
\slikaeps{Potek amplitud posameznega harmonika napake $\varepsilon$ od dinamične ekscentričnosti v smeri x pri simulacijah z linearnim poljem}{./LIN/xd_potek}
Poteki opisani s kubičnimi polinomi.
\begin{eqnarray}
\label{nap_lin_xd}
&C_0 =2,64\cdot 10^{-4}\Delta x_d^{3}+0,00125\Delta x_d^{2}+0,00291\Delta x_d+1,02\cdot 10^{-7} \\
&C_1 =1,58\cdot 10^{-4}\Delta x_d^{3}+2,37\cdot 10^{-3}\Delta x_d^{2}+33,8\Delta x_d+2,28\cdot 10^{-7} \\     
&C_2 =1,06\cdot 10^{-3}\Delta x_d^{3}+9,95\Delta x_d^{2}-1,95\cdot 10^{-3}\Delta x_d+7,96\cdot 10^{-7} \\             
&C_3 =3,91\Delta x_d^{3}-1,41\cdot 10^{-3}\Delta x_d^{2}+9,91\cdot 10^{-4}\Delta x_d+1,06\cdot 10^{-5} \\               
&C_4 =1,73\Delta x_d^{3}-5,52\cdot 10^{-1}\Delta x_d^{2}+6,15\cdot 10^{-2}\Delta x_d-1,36\cdot 10^{-3}            
\end{eqnarray}
Poteki (\ref{vrsta_xd}) razviti v Taylorjevo vrsto, so podali enake rezultate.
\begin{eqnarray}
&C_0 =0\\
&C_1 =33,8\Delta x_d\\     
&C_2 =9,95\Delta x_d^{2}\\             
&C_3 =3,91\Delta x_d^{3}\\         
&C_4 =0            
\end{eqnarray}
Predstavljen je bil potek spreminjanja $sin$ in $cos$ in napake v odvistnosti od ekscentričnosti. Napaka zaradi dinamične ekscentričnosti je bila 0, zato rezultati tudi niso podani. Dinamična ekscentričnost v smeri y nima vpliva na enosmerno komponento, niti na osnovni harmonik $sin$ in $cos$.
%Rezultati se lepo prilegajo, četrti harmonik razvit s Taylorjevo vrsto je nič, saj je šele člen s potenco 4 neničelen. Polinom četrtega reda lahko kljub temu aprosimiramo s kubičnim polinomom po metodi najmanjših kvadratov.

%Dinamična ekscentričnost v smeri y pri simulacijsah s linearnim poljem, ne povzroči napake, zato je ne morem opisati.

%Pogledali smo si, kako se bodo obnašali signali $sin$, $cos$ in napaka glede na ekcentričnost senzorja ali magneta. Poteki so bili pričakovani, vendar semoramo zavedati, da je simulacijski model s linearizacijo polja zelo poenostavljen. V nadaljevanju pričakujem drugačno obnašanje potekov signalov $sin$ in $cos$, napaka se bo izražala enako le harmoniki bodo imeli manjšo amplitudo ob večjih ekscentričnostih.


%
%Dinamična ekscentričnost v smeri y, pri simulacijah z linearnim poljem ni izražala napake, saj se polje ob tej ekscentričnosti zaradi aproksimacije z ravnino, za Hall-ove sonde ni spremenilo.
%
%V tem poglavju sem predstavil numerične rezultate simulacij z magnetnim poljem aproksimiranega z ravnino. Rezultati so potrdili prevladujoče harmomnike, pojavljajo se tudi višji, kar sem upošteval pri napaki izrazeni z neskončno vrsto. V nadaljevanju pričakujem, z boljšim modelom magnetnega polja, manjše napake po amplitudi.
%
%
%%H0	[1,10753993340810e-13;1,24706314026430e-12;-6,88797159092575e-12;1,01065478093830e-11]
%%H1	[3,33808204809378e-14;33,7618618558909;2,34008673890266e-12;-2,18397874745013e-12]
%%H2	[1,02998847934023e-14;-1,41595026512391e-13;9,94718394324374;-7,34898016894629e-14]
%%H3	[3,30877367536122e-15;-2,68649378966597e-13;9,00568019414565e-13;3,90762289998666]
%%
%%\section{YD}
%%ni nic ker ni odvisno od njega