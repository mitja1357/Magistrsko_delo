\chapter{Linearni model}

V prej"snjem poglavju smo magnetno polje magneta aproksimirali z ravnino ter napako izra"cunali z neskončno vrsto. V tem poglavju bom predstavil simulacije opravljene na magnetnem polju aproksimiranega z ravnino, izra"cnan kot $\varphi$ je rezultat numeri"cne funkicje atan2d(y,x) citeatan2d. Predstavil bom napako, jo razstavil na posamezne harmonike, ter prikazal spreminjanje amplitud glede na spremembo ekscentri"cnosti. Ravnina aproksimiranega magnetnega polja je: 
\begin{equation}
\label{equ:lin_polje}
B(x,y)= x
\end{equation}
Hall-ovi sondi sti postavljeni na kro"znico z radijem 2,4 mm \cite{AM8192}.


\section{Brez napake}

Za za"cetek si poglejmo idealno montriran tako senzor kot magnet. Signala $sin$ in $cos$ imata enaki amplitudi in sta fazno zamaknjena za $90^{\circ}$. Napaka $\varepsilon$, ki se pojavi pri izra"cunu je tako le numeri"cna napaka funkcije atan2d (Slika \ref{./LIN/00_napaka}).
\slikaeps{Potek signalov $sin$ in $cos$ brez ekscentričnosti}{./LIN/yd_sincos}
\slikaeps{Napaka $\varepsilon$ pri simulacijah z linearnim magnetnim poljem brez ekscentričnosti}{./LIN/yd_napaka}

Numeri"cno napako lahko na pri"cakovano napako zaradi ekscnetri"cnosti zanemarim.


\newpage
\section{Simulacija stati"cne ekscentri"cnosti v smeri x-osi}

Oglejmo si rezultate simulacij stati"cne ekscentri"cnosti v smeri x. Po pri"cakovanjih se bo povi"sala amplituda $sin$ in $cos$ signala ter zmanj"sal njun fazni zamik (izraza (\ref{equ:Bx_stat}) in \ref{equ:By_stat}).
 
\slikaeps{Signala $sin$ in $cos$ pri simulacijah z linearnim poljem pri 0,24 mm stati"cne ekscentri"cnosti v smeri x}{./LIN/xs_sincos}
\newpage


Napaka $\varepsilon$  je prikazana na sliki \ref{./LIN/xs_napaka}.

\slikaeps{Napaka $\varepsilon$ pri simulacijah z linearnim poljem pri 0,24 mm stati"cne ekscentri"cnosti v smeri x}{./LIN/xs_napaka}

Napako razvijmo v Fourierovo vrsto in pridobimo amplitude posameznih harmonikov napake(Slika \ref{./LIN/xs_fft}).
\slikaeps{Amplitude harmonikov napake $\varepsilon$ razvite v Fourierovo vrsto pri simulacijah z linearnim poljem pri 0,24 mm stati"cne ekscentri"cnosti v smeri x}{./LIN/xs_fft}

Po pri"cakovanjih najbolj izstopata enosmerna komponenta (harmonik 0) in drugi harmonik.
\newpage
\subsection{Sprememba $sin$, $cos$ ter napake od $\Delta x_s$}

Poglejmo tudi kako se spreminjata analogna signala $sin$ in $cos$ ob spreminjanju ekscentričnosti. Na sliki \ref{./LIN/xs_sincos_amp} je prikazana sprememba amplitude prvega harmonika, na sliki \ref{./LIN/xs_sincos_off} enosmerni komponenti in na sliki \ref{./LIN/xs_sincos_phase} fazni zamik signalov glede na njuno idealno poravnavo. Iz slik vidimo kar nam opišeti enačbi (\ref{equ:Bx_stat})(\ref{equ:By_stat}). Spremeniti se amplitudi, pri čemer se amplituda $cos$ signala spreminja hitreje, ter fazni zamik $sin$.

 %\ref{./LIN/xs_sincos_amp}. Amplituda $cos$ se spreminja hitreje, kot amplituda $sin$ .

\slikaeps{Amplituda osnovnega harmonika signalov $sin$ in $cos$ pri simulacijah z linearnim poljem stati"cne ekscentri"cnosti v smeri x}{./LIN/xs_sincos_amp}
\slikaeps{Amplituda osnovnega harmonika signalov $sin$ in $cos$ pri simulacijah z linearnim poljem stati"cne ekscentri"cnosti v smeri x}{./LIN/xs_sincos_off}
\slikaeps{Amplituda osnovnega harmonika signalov $sin$ in $cos$ pri simulacijah z linearnim poljem stati"cne ekscentri"cnosti v smeri x}{./LIN/xs_sincos_phase}

Kako se spreminjajo amplituda prvega harmonika, enosmerna kompoonenta in fazni zamik $sin$ in $cos$ signalov lahko razberemo iz enačb (\ref{equ:Bx_stat}) in (\ref{equ:By_stat}). Tu poteke razvijem v Taylorjevo vrsto do tretje stopnje, kar bom lahko imel za primerjavo.

\begin{eqnarray}
\label{analog_lin_xs}
&A_{sin} = 2,08 \cdot 10^{-1} \Delta x_s^2+2,4\\
&Off_{sin} = 0 \\
&\delta_{sin} =-1,38 \Delta x_s^3+ 23,9 \Delta x_s  \\
&A_{cos} = \Delta x_s+2,4\\
&Off_{cos} = 0 \\
&\delta_{cos} = 0
\end{eqnarray}

\newpage
Spremembi signalov $sin$ in $cos$ se odrazita tudi pri izračunu kota $\varphi$ in napake $\varepsilon$.
Na sliki \ref{./LIN/xs_potek} vidimo odvisnost amplitud posameznega harmonika od spreminjanja statične ekscentričnosti v smeri x.

\slikaeps{Potek amplitud posameznega harmonika napake $\varepsilon$ od stati"cne ekscentri"cnosti v smeri x}{./LIN/xs_potek}

Poteke s slike \ref{./LIN/xs_potek} aproksimiramo s kubičnimi polinomi in dobimo naslednje poteke:

%predstavimo enako kot (\ref{vrsta:xs}).

\begin{eqnarray}
\label{nap_lin_xs}
&C_0 =3,35\cdot 10^{-1}\Delta x_s^{3}-2,48\Delta x_s^{2}+1,19\cdot 10\Delta x_s+1,23\cdot 10^{-5} \\
&C_1 =5,56\cdot 10^{-4}\Delta x_s^{3}-2,00\cdot 10^{-3}\Delta x_s^{2}+4,34\cdot 10^{-3}\Delta x_s+7,67\cdot 10^{-8} \\
&C_2 =4,13\cdot 10^{-1}\Delta x_s^{3}-3,53\Delta x_s^{2}+1,69\cdot 10\Delta x_s-2,31\cdot 10^{-5} \\
&C_3 =-2,17\cdot 10^{-4}\Delta x_s^{3}+2,57\cdot 10^{-4}\Delta x_s^{2}+4,20\cdot 10^{-3}\Delta x_s+4,51\cdot 10^{-8} \\
&C_4 =-8,27\cdot 10^{-1}\Delta x_s^{3}+2,42\Delta x_s^{2}+8,08\cdot 10^{-3}\Delta x_s-1,60\cdot 10^{-4}
\end{eqnarray}

Za primerjavo, dodam tudi enačbe potekov amplitude posameznega harmonika razvitega v Taylorjevo vrsto, katere sledijo iz ( \ref{vrsta:xs}):

\begin{eqnarray}
&C_0 =3,45\cdot 10^{-1}\Delta x_s^{3}-2,49\Delta x_s^{2}+1,19\cdot 10\Delta x_s \\
&C_1 = 0\\
&C_2 =3,66\cdot 10^{-1}\Delta x_s^{3}-3,51\Delta x_s^{2}+1,69\cdot 10\Delta x_s\\
&C_3 = 0\\
&C_4 =-1,04\Delta x_s^{3}+2,49\Delta x_s^{2}
\end{eqnarray}

Rezultati se ujemajo s pričakovanimi. Enosmerna komponenta in amplituda prvega harmonika naračšata linearno, četrti harmonik narašča s kvadratom ekscentričnosti, lihi harmoniki, so zanemarljivi.



%
%\subsection{Sin\_cos}
%\subsection{napaka}
%\subsection{fft\_napake}
%\section{XS}
%\subsection{Sin\_cos}
%\subsection{napaka}
%\subsection{fft\_napake}
%\subsection{visanje\_napake}
%nastavek = 0.3352 xs^3-2.4826 xs^2+11.9361 xs+1.0156e-5
%H0	[1,01555901858575e-05;11,9361321852939;-2,48260332886115;0,335183727157872]
%H1	[2,34507105720499e-14;1,58051177708914e-13;-6,51993471506150e-13;9,46713162781764e-13]
%H2	[-1,86535135849702e-05;16,8819884980667;-3,52835647417836;0,412029175485044]
%H3	[1,30048463315339e-14;-2,42620430114880e-14;3,51682497683473e-13;-6,28770756738832e-13]
%
%


\section{Simulacija stati"cne ekscentri"cnosti v smeri y-osi}

Oglejmo si "se rezultate simulacij stati"cne ekscentri"cnosti v smeri y. Pričakujem podobne rezultate kot pri statični ekscentričnosti v smeri x, le da bo tu hitreje naračšala amplituda $sin$ signala, spreminjal se bo fazni zamik $cos$.
 
\slikaeps{Signala $sin$ in $cos$ pri simulacijah z linearnim poljem pri 0,24 mm stati"cne ekscentri"cnosti v smeri y}{./LIN/ys_sincos}

Napaka je prikazana na sliki \ref{./LIN/ys_napaka}. Sestavlja jo negativna enosmerna komponenta in izrazit drugi harmonik.
\slikaeps{Napaka $\varepsilon$ pri simulacijah z linearnim poljem pri 0,24 mm stati"cne ekscentri"cnosti v smeri y}{./LIN/ys_napaka}
Razvijmo jo v Fourierovo vrsto in pridobimo amplitude posameznih harmonikov napake (Slika \ref{./LIN/ys_fft}).
\slikaeps{Amplitude harmonikov napake $\varepsilon$ razvite v Fourierovo vrsto pri simulacijah z linearnim poljem pri 0,24 mm stati"cne ekscentri"cnosti v smeri y}{./LIN/ys_fft}

Iz fft-ja napake je vidna enaka amplituda drugega harmonika kot pri enaki ekscentričnosti v smeri x. Enosmerna komponenta se razlikuje le v predznaku.
\newpage
\subsection{Sprememba $sin$, $cos$ ter napake od $\Delta y_s$}

Oglejmo si sedaj poteke amplitude, enosmerne komponente in faznega zamika pri statični ekscentričnosti v smeri y. Poteki so podobni kot pri ekscentričnosti v semeri x.

\slikaeps{Amplituda osnovnega harmonika signalov $sin$ in $cos$ pri simulacijah z linearnim poljem stati"cne ekscentri"cnosti v smeri y}{./LIN/ys_sincos_amp}
\slikaeps{Amplituda osnovnega harmonika signalov $sin$ in $cos$ pri simulacijah z linearnim poljem stati"cne ekscentri"cnosti v smeri y}{./LIN/ys_sincos_off}
\slikaeps{Amplituda osnovnega harmonika signalov $sin$ in $cos$ pri simulacijah z linearnim poljem stati"cne ekscentri"cnosti v smeri y}{./LIN/ys_sincos_phase}
Poteke zapišimo še s kubičnimi polinomi.
\begin{eqnarray}
\label{analog_lin_ys}
 &A_{sin} = \Delta x_s+2,4\\
 &Off_{sin} = 0 \\
 &\delta_{sin} =0 \\
 &A_{cos} = 2,08 \cdot 10^{-1} \Delta x_s^2+2,4\\
 &Off_{cos} = 0 \\
 &\delta_{cos} = 1,38 \Delta x_s^3- 23,9 \Delta x_s
\end{eqnarray}

Iz potekov signalov $sin$ in $cos$ lahko predvidimo, da bo potek podoben le enosmerna komponenta bo negativna.

\slikaeps{Potek amplitud posameznega harmonika napake $\varepsilon$ od stati"cne ekscentri"cnosti v smeri y}{./LIN/ys_potek}

Poteke s slike \ref{./LIN/ys_potek}, predstavimo s polinomom tretje stopnje. Aproksimacije pokažejo enake amplitude posameznih harmonikov, le enosmerna komponenta je dobila negativen predznak.

\begin{eqnarray}
\label{nap_lin_ys}
&C_0 =-3,35\cdot 10^{-1}\Delta y_s^{3}+2,48\Delta y_s^{2}-1,19\cdot 10\Delta y_s-1,22\cdot 10^{-5} \\                  
&C_1 =1,09\cdot 10^{-4}\Delta y_s^{3}-8,69\cdot 10^{-4}\Delta y_s^{2}+4,34\cdot 10^{-3}\Delta y_s+7,62\cdot 10^{-10} \\
&C_2 =4,12\cdot 10^{-1}\Delta y_s^{3}-3,53\Delta y_s^{2}+1,69\cdot 10\Delta y_s-2,31\cdot 10^{-5} \\                   
&C_3 =2,43\cdot 10^{-4}\Delta y_s^{3}-1,30\cdot 10^{-3}\Delta y_s^{2}+4,20\cdot 10^{-3}\Delta y_s+1,83\cdot 10^{-8} \\ 
&C_4 =-8,26\cdot 10^{-1}\Delta y_s^{3}+2,42\Delta y_s^{2}+6,13\cdot 10^{-3}\Delta y_s-1,60\cdot 10^{-4}            
\end{eqnarray}
%
%
%
%
%%
%%
%%
%%\section{YS}
%%\subsection{Sin\_cos}
%%\subsection{napaka}
%%\subsection{fft\_napake}
%%\subsection{visanje\_napake}
%%
%%H0	[-1,01555899896284e-05;-11,9361321852922;2,48260332885533;-0,335183727150766]
%%H1	[2,04074110999664e-14;3,79449756434155e-13;-2,10317924735210e-12;2,95771540643260e-12]
%%H2	[-1,86535135760308e-05;16,8819884980664;-3,52835647417692;0,412029175482732]
%%H3	[1,26502083464184e-14;1,05725909044143e-14;-1,64647766137320e-14;1,89710785932321e-14]
%%
%%\section{ZS}
%%ni nic ker je atan(k/k)
%%\section{Xd}
%%\subsection{Sin\_cos}
%%\subsection{napaka}
%%\subsection{fft\_napake}
%%\subsection{visanje\_napake}    
%
%
%
\section{Dinami"cna ekscentri"cnost v smeri x osi}
%
Oglejmo si sedaj rezultate simulacij dinami"cne ekscentri"cnosti. V signalih $sin$ in $cos$ se pojavi enosmerna komponenta (Slika \ref{./LIN/xd_sincos}).
\slikaeps{Signala $sin$ in $cos$ pri simulacijah z linearnim poljem pri 0,24 mm dinami"cne ekscentri"cnosti v smeri x}{./LIN/xd_sincos}

 Na sliki \ref{./LIN/xd_napaka} je vidna napaka v obliki prvega harmonika, kar smo pričakovali.
\slikaeps{Napaka $\varepsilon$ pri simulacijah z linearnim poljem pri 0,24 mm dinami"cne ekscentri"cnosti v smeri y}{./LIN/xd_napaka}

Po fft-ju napake se vidi izrazit prvi harmonik, enosmerne komponente je nič.
\slikaeps{Amplitude harmonikov napake $\varepsilon$ pri simulacijah z linearnim poljem pri 0,24 mm dinami"cne ekscentri"cnosti v smeri x}{./LIN/xd_fft}


\newpage
\subsection{Sprememba $sin$, $cos$ ter napake od $\Delta x_d$}
Oglejmo si, kako se spreminjata signala $sin$ $cos$ po pričakovanjih se bo najbolj izrazito spreminjala enosmerna komponenta.
\slikaeps{Amplituda osnovnega harmonika signalov $sin$ in $cos$ pri simulacijah z linearnim poljem dinamične ekscentri"cnosti v smeri x}{./LIN/xd_sincos_amp}
\slikaeps{Amplituda osnovnega harmonika signalov $sin$ in $cos$ pri simulacijah z linearnim poljem dinamične ekscentri"cnosti v smeri x}{./LIN/xd_sincos_off}
\slikaeps{Amplituda osnovnega harmonika signalov $sin$ in $cos$ pri simulacijah z linearnim poljem dinamične ekscentri"cnosti v smeri x}{./LIN/xd_sincos_phase}
Poteke zapišimo še s kubičnimi polinomi. Amplituda prvega harmonika in fazni zamik so nespremenjeni, spreminja se le enosmerna komponenta.
\begin{eqnarray}
\label{analog_lin_xd}
&A_{sin} = 2,4\\
&Off_{sin} = -\Delta x_d \\
&\delta_{sin} =0 \\
&A_{cos} = 2,4\\
&Off_{cos} = -\Delta x_d\\
&\delta_{cos} = 0
\end{eqnarray}

Na sliki \ref{./LIN/xd_potek} vidimo odvisnost amplitud napake od spreminjanja ekscentri"cnosti. Napaka, se po pričakovanjih najbolj izrazi s prvim harmonikom.

\slikaeps{Potek amplitud posameznega harmonika napake $\varepsilon$ od dinami"cne ekscentri"cnosti v smeri x}{./LIN/xd_potek}

Poteke harmonikov s slike \ref{./LIN/xd_potek} aproksimiramo  s polinomi. 

\begin{eqnarray}
\label{nap_lin_xd}
&C_0 =2,64\cdot 10^{-4}\Delta x_d^{3}+1,25\cdot 10^{-3}\Delta x_d^{2}+2,91\cdot 10^{-3}\Delta x_d+1,02\cdot 10^{-7} \\
&C_1 =1,58\cdot 10^{-4}\Delta x_d^{3}+2,37\cdot 10^{-3}\Delta x_d^{2}+3,38\cdot 10\Delta x_d+2,28\cdot 10^{-7} \\     
&C_2 =1,06\cdot 10^{-3}\Delta x_d^{3}+9,95\Delta x_d^{2}-1,95\cdot 10^{-3}\Delta x_d+7,96\cdot 10^{-7} \\             
&C_3 =3,91\Delta x_d^{3}-1,41\cdot 10^{-3}\Delta x_d^{2}+9,91\cdot 10^{-4}\Delta x_d+1,06\cdot 10^{-5} \\               
&C_4 =1,73\Delta x_d^{3}-5,52\cdot 10^{-1}\Delta x_d^{2}+6,15\cdot 10^{-2}\Delta x_d-1,36\cdot 10^{-3}            
\end{eqnarray}

Tu lahko zapišem enačbe iz (\ref{vrsta_xd}) razvite v Taylorjevo vrsto.


\begin{eqnarray}
&C_0 =0\\
&C_1 =3,38\cdot 10\Delta x_d\\     
&C_2 =9,95\Delta x_d^{2}\\             
&C_3 =3,91\Delta x_d^{3}\\         
&C_4 =0            
\end{eqnarray}

Rezultati se lepo prilegajo, četrti harmonik razvit s Taylorjevo vrsto je nič, saj je šele člen s potenco 4 neničelen. Polinom četrtega reda lahko kljub temu aprosimiramo s kubičnim polinomom po metodi najmanjših kvadratov.

Dinamična ekscentričnost v smeri y pri simulacijsah s linearnim poljem, ne povzroči napake, zato je ne morem opisati.

Pogledali smo si, kako se bodo obnašali signali $sin$, $cos$ in napaka glede na ekcentričnost senzorja ali magneta. Poteki so bili pričakovani, vendar semoramo zavedati, da je simulacijski model s linearizacijo polja zelo poenostavljen. V nadaljevanju pričakujem drugačno obnašanje potekov signalov $sin$ in $cos$, napaka se bo izražala enako le harmoniki bodo imeli manjšo amplitudo ob večjih ekscentričnostih.


%
%Dinami"cna ekscentri"cnost v smeri y, pri simulacijah z linearnim poljem ni izra"zala napake, saj se polje ob tej ekscentri"cnosti zaradi aproksimacije z ravnino, za Hall-ove sonde ni spremenilo.
%
%V tem poglavju sem predstavil numeri"cne rezultate simulacij z magnetnim poljem aproksimiranega z ravnino. Rezultati so potrdili prevladujo"ce harmomnike, pojavljajo se tudi vi"sji, kar sem upošteval pri napaki izrazeni z neskončno vrsto. V nadaljevanju pri"cakujem, z bolj"sim modelom magnetnega polja, manj"se napake po amplitudi.
%
%
%%H0	[1,10753993340810e-13;1,24706314026430e-12;-6,88797159092575e-12;1,01065478093830e-11]
%%H1	[3,33808204809378e-14;33,7618618558909;2,34008673890266e-12;-2,18397874745013e-12]
%%H2	[1,02998847934023e-14;-1,41595026512391e-13;9,94718394324374;-7,34898016894629e-14]
%%H3	[3,30877367536122e-15;-2,68649378966597e-13;9,00568019414565e-13;3,90762289998666]
%%
%%\section{YD}
%%ni nic ker ni odvisno od njega