\chapter{Linearni model}

V prej"snjem poglavju smo magnetno polje magneta aproksimirali z ravnino ter napako izra"cunali z neskončno vrsto. V tem poglavju bom predstavil simulacije opravljene na magnetnem polju aproksimiranega z ravnino, izra"cnan kot $\varphi$ je rezultat numeri"cne funkicje atan2d(y,x) citeatan2d. Predstavil bom napako, jo razstavil na posamezne harmonike, ter prikazal spreminjanje amplitud glede na spremembo ekscentri"cnosti. Ravnina aproksimiranega magnetnega polja je: 
\begin{equation}
\label{equ:lin_polje}
B(x,y)= x
\end{equation}
Hall-ovi sondi sti postavljeni na kro"znico z radijem 2,4 mm.


\section{Brez napake}

Za za"cetek si poglejmo idealno montriran tako senzor kot magnet. Signala $sin$ in $cos$ imata enaki amplitudi in sta fazno zamaknjena za $90^{\circ}$. Napaka $\varepsilon$, ki se pojavi pri izra"cunu je tako le numeri"cna napaka funkcije atan2d (Slika \ref{./LIN/lin_00_napaka}).
\slikaeps{Potek signalov $sin$ in $cos$ ob idealni montaži}{./LIN/lin_00_sincos}
\slikaeps{Napaka $\varepsilon$ pri simulacijah z linearnim magnetnim poljem pri idealni monta"zi}{./LIN/lin_00_napaka}

Numeri"cno napako lahko na pri"cakovano napako zaradi ekscnetri"cnosti zanemarim.


\newpage
\section{Simulacija stati"cne ekscentri"cnosti v smeri x-osi}

Oglejmo si rezultate simulacij stati"cne ekscentri"cnosti v smeri x. Po pri"cakovanjih se bo povi"sala amplituda $sin$ in $cos$ signala ter zmanj"sal njun fazni zamik (izraza (\ref{equ:Bx_stat}) in \ref{equ:By_stat}).
 
\slikaeps{Signala $sin$ in $cos$ pri simulacijah z linearnim poljem pri 0,24 mm stati"cne ekscentri"cnosti v smeri x}{./LIN/lin_xs_sincos}
\newpage
Napaka $\varepsilon$  je prikazana na sliki \ref{./LIN/lin_xs_napaka}.

\slikaeps{Napaka $\varepsilon$ pri simulacijah z linearnim poljem pri 0,24 mm stati"cne ekscentri"cnosti v smeri x}{./LIN/lin_xs_napaka}

Napako razvijmo v Fourierovo vrsto in pridobimo amplitude posameznih harmonikov napake(Slika \ref{./LIN/lin_xs_fft}).
\slikaeps{Amplitude harmonikov napake $\varepsilon$ razvite v Fourierovo vrsto pri simulacijah z linearnim poljem pri 0,24 mm stati"cne ekscentri"cnosti v smeri x}{./LIN/lin_xs_fft}

Po pri"cakovanjih najbolj izstopata enosmerna komponenta (harmonik 0) in drugi harmonik. Na sliki \ref{./LIN/lin_xs_potek} vidimo odvisnost amplitud od spreminjanja ekscentri"cnosti.

\slikaeps{Potek amplitud posameznega harmonika napake $\varepsilon$ od stati"cne ekscentri"cnosti v smeri x}{./LIN/lin_xs_potek}

Poteke s slike \ref{./LIN/lin_xs_potek} predstavimo enako kot (\ref{vrsta:xs}).

\begin{eqnarray}
&C_0 =3,35\cdot 10^{-1}\Delta x_s^{3}-2,48\Delta x_s^{2}+1,19\cdot 10\Delta x_s+1,23\cdot 10^{-5} \\
&C_1 =5,56\cdot 10^{-4}\Delta x_s^{3}-2,00\cdot 10^{-3}\Delta x_s^{2}+4,34\cdot 10^{-3}\Delta x_s+7,67\cdot 10^{-8} \\
&C_2 =4,13\cdot 10^{-1}\Delta x_s^{3}-3,53\Delta x_s^{2}+1,69\cdot 10\Delta x_s-2,31\cdot 10^{-5} \\
&C_3 =-2,17\cdot 10^{-4}\Delta x_s^{3}+2,57\cdot 10^{-4}\Delta x_s^{2}+4,20\cdot 10^{-3}\Delta x_s+4,51\cdot 10^{-8} \\
&C_4 =-8,27\cdot 10^{-1}\Delta x_s^{3}+2,42\Delta x_s^{2}+8,08\cdot 10^{-3}\Delta x_s-1,60\cdot 10^{-4}
\end{eqnarray}




%
%\subsection{Sin\_cos}
%\subsection{napaka}
%\subsection{fft\_napake}
%\section{XS}
%\subsection{Sin\_cos}
%\subsection{napaka}
%\subsection{fft\_napake}
%\subsection{visanje\_napake}
%nastavek = 0.3352 xs^3-2.4826 xs^2+11.9361 xs+1.0156e-5
%H0	[1,01555901858575e-05;11,9361321852939;-2,48260332886115;0,335183727157872]
%H1	[2,34507105720499e-14;1,58051177708914e-13;-6,51993471506150e-13;9,46713162781764e-13]
%H2	[-1,86535135849702e-05;16,8819884980667;-3,52835647417836;0,412029175485044]
%H3	[1,30048463315339e-14;-2,42620430114880e-14;3,51682497683473e-13;-6,28770756738832e-13]
%
%


\section{Simulacija stati"cne ekscentri"cnosti v smeri y-osi}

Oglejmo si "se rezultate simulacij stati"cne ekscentri"cnosti v smeri y. Pričakujem podobne rezultate kot pri statični ekscentričnosti v smeri x, le da bo tu hitreje naračšala amplituda $sin$ signala.
 
\slikaeps{Signala $sin$ in $cos$ pri simulacijah z linearnim poljem pri 0,24 mm stati"cne ekscentri"cnosti v smeri y}{./LIN/lin_ys_sincos}

Napaka je prikazana na sliki \ref{./LIN/lin_ys_napaka}.
\slikaeps{Napaka $\varepsilon$ pri simulacijah z linearnim poljem pri 0,24 mm stati"cne ekscentri"cnosti v smeri y}{./LIN/lin_ys_napaka}

Razvijmo jo v Fourierovo vrsto in pridobimo amplitude posameznih harmonikov napake(Slika \ref{./LIN/lin_ys_fft}).
\slikaeps{Amplitude harmonikov napake $\varepsilon$ razvite v Fourierovo vrsto pri simulacijah z linearnim poljem pri 0,24 mm stati"cne ekscentri"cnosti v smeri y}{./LIN/lin_ys_fft}

Tudi tu najbolj izstopata enosmerna komponenta in drugi harmonik. Za razliko od stat"cne ekscentri"cnosti v smeri x je tu enosmerna komponenta negativna.



Na sliki \ref{./LIN/lin_ys_potek} vidimo odvisnost amplitud od spreminjanja stati"cne ekscentri"cnosti v smeri y.

\slikaeps{Potek amplitud posameznega harmonika napake $\varepsilon$ od stati"cne ekscentri"cnosti v smeri y}{./LIN/lin_ys_potek}

Poteke s slike \ref{./LIN/lin_ys_potek} predstavimo s polinomom tretje stopnje.

\begin{eqnarray}
&C_0 =3,35\cdot 10^{-1}\Delta y_s^{3}-2,48\Delta y_s^{2}+1,19\cdot 10\Delta y_s+1,22\cdot 10^{-5} \\                   
&C_1 =1,09\cdot 10^{-4}\Delta y_s^{3}-8,69\cdot 10^{-4}\Delta y_s^{2}+4,34\cdot 10^{-3}\Delta y_s+7,62\cdot 10^{-10} \\
&C_2 =4,12\cdot 10^{-1}\Delta y_s^{3}-3,53\Delta y_s^{2}+1,69\cdot 10\Delta y_s-2,31\cdot 10^{-5} \\                   
&C_3 =2,43\cdot 10^{-4}\Delta y_s^{3}-1,30\cdot 10^{-3}\Delta y_s^{2}+4,20\cdot 10^{-3}\Delta y_s+1,83\cdot 10^{-8} \\ 
&C_4 =-8,26\cdot 10^{-1}\Delta y_s^{3}+2,42\Delta y_s^{2}+6,13\cdot 10^{-3}\Delta y_s-1,60\cdot 10^{-4} \\               
\end{eqnarray}




%
%
%
%\section{YS}
%\subsection{Sin\_cos}
%\subsection{napaka}
%\subsection{fft\_napake}
%\subsection{visanje\_napake}
%
%H0	[-1,01555899896284e-05;-11,9361321852922;2,48260332885533;-0,335183727150766]
%H1	[2,04074110999664e-14;3,79449756434155e-13;-2,10317924735210e-12;2,95771540643260e-12]
%H2	[-1,86535135760308e-05;16,8819884980664;-3,52835647417692;0,412029175482732]
%H3	[1,26502083464184e-14;1,05725909044143e-14;-1,64647766137320e-14;1,89710785932321e-14]
%
%\section{ZS}
%ni nic ker je atan(k/k)
%\section{Xd}
%\subsection{Sin\_cos}
%\subsection{napaka}
%\subsection{fft\_napake}
%\subsection{visanje\_napake}    



\section{Dinami"cna ekscentri"cnost}

Oglejmo si sedaj rezultate simulacij dinami"cne ekscentri"cnosti. V signalih $sin$ in $cos$ se pojavi enosmerna komponenta (Slika \ref{./LIN/lin_xd_sincos}).
\slikaeps{Signala $sin$ in $cos$ pri simulacijah z linearnim poljem pri 0,24 mm dinami"cne ekscentri"cnosti v smeri x}{./LIN/lin_xd_sincos}

\slikaeps{Napaka $\varepsilon$ pri simulacijah z linearnim poljem pri 0,24 mm dinami"cne ekscentri"cnosti v smeri y}{./LIN/lin_xd_napaka}

V napaki prevladuje prvi harmonik kar je vidno tudi iz razvoja v Fourierovo vrsto (Slika \ref{./LIN/lin_xd_fft})

\slikaeps{Amplitude harmonikov napake $\varepsilon$ razvite v Fourierovo vrsto pri simulacijah z linearnim poljem pri 0,24 mm dinami"cne ekscentri"cnosti v smeri x}{./LIN/lin_xd_fft}



Na sliki \ref{./LIN/lin_xd_potek} vidimo odvisnost amplitud od spreminjanja ekscentri"cnosti.

\slikaeps{Potek amplitud posameznega harmonika napake $\varepsilon$ od dinami"cne ekscentri"cnosti v smeri x}{./LIN/lin_xd_potek}

Poteke harmonikov s slike \ref{./LIN/lin_xd_potek} aproksimiramo  s polinomi. 

\begin{eqnarray}
&C_0 =2,64\cdot 10^{-4}\Delta x_d^{3}+1,25\cdot 10^{-3}\Delta x_d^{2}+2,91\cdot 10^{-3}\Delta x_d+1,02\cdot 10^{-7} \\
&C_1 =1,58\cdot 10^{-4}\Delta x_d^{3}+2,37\cdot 10^{-3}\Delta x_d^{2}+3,38\cdot 10\Delta x_d+2,28\cdot 10^{-7} \\     
&C_2 =1,06\cdot 10^{-3}\Delta x_d^{3}+9,95\Delta x_d^{2}-1,95\cdot 10^{-3}\Delta x_d+7,96\cdot 10^{-7} \\             
&C_3 =3,91\Delta x_d^{3}-1,41\cdot 10^{-3}\Delta x_d^{2}+9,91\cdot 10^{-4}\Delta x_d+1,06\cdot 10^{-5} \\             
&C_4 =1,73\Delta x_d^{3}-5,52\cdot 10^{-1}\Delta x_d^{2}+6,15\cdot 10^{-2}\Delta x_d-1,36\cdot 10^{-3} \\           
\end{eqnarray}

Dinami"cna ekscentri"cnost v smeri y, pri simulacijah z linearnim poljem ni izra"zala napake, saj se polje ob tej ekscentri"cnosti zaradi aproksimacije z ravnino, za Hall-ove sonde ni spremenilo.

V tem poglavju sem predstavil numeri"cne rezultate simulacij z magnetnim poljem aproksimiranega z ravnino. Rezultati so potrdili prevladujo"ce harmomnike, pojavljajo se tudi vi"sji, kar sem upošteval pri napaki izrazeni z neskončno vrsto. V nadaljevanju pri"cakujem, z bolj"sim modelom magnetnega polja, manj"se napake po amplitudi.


%H0	[1,10753993340810e-13;1,24706314026430e-12;-6,88797159092575e-12;1,01065478093830e-11]
%H1	[3,33808204809378e-14;33,7618618558909;2,34008673890266e-12;-2,18397874745013e-12]
%H2	[1,02998847934023e-14;-1,41595026512391e-13;9,94718394324374;-7,34898016894629e-14]
%H3	[3,30877367536122e-15;-2,68649378966597e-13;9,00568019414565e-13;3,90762289998666]
%
%\section{YD}
%ni nic ker ni odvisno od njega