\chapter{Sklep}
\section{Merjenje kota in vpliv na napako}
Aplikacije, ki za delovanje potrebujejo informacijo o zasuku, lahko zasuk merijo na različne načine. Senzor RM44 za merjenje kota uporablja Hallove sonde. Senzor je robusten in zato primeren tudi za aplikacije v bolj obremenjenih okoljih. Izhod senzorja kljub robustonsti lahko vsebuje napako. Napaka je lahko posledica nepravilne montaže. S poznavanjem vplivov na napako senzorja zaradi nepravilne montaže, se napako lahko predvidi in odstrani.
\section{Nepravilna montaža}
Merilni sistem za merjenje kota je sestavljen iz dveh delov, magnetnega aktuatorja in senzorja za merjenje magnetnega polja. Nepravilno je lahko montiran aktuator ali senzor.
Nepravilno montriran senzor, se izrazi kot statična eksecentričnost. Hallova sonda v senzorju zajame magnetno polje s spremenjeno amplitudo in s spremenjenim faznim zamikom.
 Nepravilno montiran aktuator se izrazi kot dinamična ekscentričnost. Hallova sonda v senzorju pomeri dodatno enosmerno komponento magnetnega polja. Z diferencialnim odčitavanjem polja se  enosmerno komponento odstrani.
\section{Oblika napake}
Zaradi nepravilnega zajema magnetnega polja, podatek o kotu vsebuje napako. Napaka zaradi dinamične ekscentričnosti se izrazi kot sinusni signal. Z diferenicialnim odčitavanjem napaka ne vsebuje prvega harmonika temveč le enosmerno komponeto. Napaka zaradi statične ekscentričnosti se izrazi s sinusnim signalom dvojne frekvence in enosmerno komponento.
\section{Izvedba meritev}
Delovnje senzorja nam je nepoznano (black-box). Senzor vrne izmerjena signala $B_{sin}$ in $B_{cos}$. Na napravi izdelani v LRTME na fakulteti, so bile opravljene meritve. Meritve so potrdile pričlakovanja.  Senzor ni bil postavljen v pravilno izhodiščno lego, zato se poteki statične ekscentričnosti v smeri x in y razlikujejo. Pravilno izhodiščno lego bi našel  z uporabo razvojne plošče RMK2. RMK2 vsebuje enka čip AM256, uporabniku pa so na voljo vsi pini čipa. Čip ima analogni pin Error, s katerim se lahko preveri poravnavo med osjo vrtenja, magnetom in čipom.
\section{Komentar rezultatov}
Zgradba in podrobnejše delovanje senzorja je poslovna skrinost, zato sem predvideval, da bo napaka pri meritvah manjša kot so pokazali rezultati simulacij. Meritve so pokazale, višje amplitude napake, kot so bile posimulirane.  Simulacijski model je bil sestavljen iz štirih Hallovih sond, neposredno iz zajetega polja je bil izračunan kot. Model je bil postavljen zelo idealno. Veliko stranskih vplivov je bilo zanemarjenih. 
Podobni senzorji \cite{iCMHM}, omogočajo popravljanje enosmernih komponent in razmerija amplitud osnovega harmonika signalov  $B_{sin}$ in $B_{cos}$. S kalibracijo senzorja bi bila napaka manjša.

Pri projektu sem se osredotočil le na parametre enosmerne komponente, amplitude in faze osnovnega harmonika  $B_{sin}$ in $B_{cos}$ Signala vsebujeta tudi višje harmonike, kateri tu niso bili obravnavani. V prihodnosti se lahko nameni pozornost tudi višjim harmonikom.



