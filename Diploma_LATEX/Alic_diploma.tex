% -*- TeX:SI -*-
% slovene sub-mode for spell check
% ----------------------------------------------------------------------
%  Predloga za obliko in navodila za pisanje diplomskih nalog v LaTex-u

%  Univerza v Ljubljani, Fakulteta za elektrotehniko

%  zbral in uredil Roman Kamnik, junij 2013

% ----------------------------------------------------------------------

\documentclass[a4paper,twoside,openright,12pt]{book}
\usepackage[latin2]{inputenc}  %Kodna stran za Windows okolje, za linux je kodna stran latin2
\usepackage[slovene]{babel}    % pravila za slovensko deljenje besed
\usepackage[pdftex]{UNI-LJ-FE-Diploma} %Stil za diplome na Fakulteti za elektrotehniko (za pdfTeX v MkiTex)
%\usepackage[pctex]{UNI-LJ-FE-Diploma} %Stil za diplome na Fakulteti za elektrotehniko  (za pcTex)

%*************************** PRILAGODITVE *****************************
% mapa s slikami
\potgrafike{./Slike/}
%prilagoditev levega roba sodih strani. �e se pri dvostranskem tisku robovi ne umemajo se lahko pove�a ali pomanj�a
\zamaknirobsodihstrani{0mm}

%*************************** NASLOVNA STRAN *****************************
\naslov{Vpliv stat�ne in dinami�ne ekscentri�nosti na napako senzorja RM44, u�inkovitost kalibracije in robustnost kalibracije na harmonske oscilacije mehanske hitrosti}
\avtor{Mitja Ali�} \univerza{Univerza v Ljubljani}
\fakulteta{Fakulteta za elektrotehniko}
\delo{Magistrsko delo}
%\delo{Diplomsko delo visoko�olskega strokovnega �tudija}
\date{Ljubljana, 2017}
\mentor{doc. dr. Mitja Nemec}
%\somentor{prof. dr. Ime Priimek}
\begin{document}

%------------------------ ZA�ETNI DEL -----------------------------------
\frontmatter
%------------------------------------------------------------------------


%************************ NASLOVNA STRAN ********************************
\maketitle


%*************************** ZAHVALA ************************************
\zahvala V zahvali se kandidati zahvali mentorju in poimensko tudi
vsem sodelavcem in prijateljem, ki so pomagali in prispevali pri
delu v laboratoriju, na ra�unalniku, v delavnici, pri tehni�ni
izdelavi dela in drugje.


%*************************** VSEBINA *************************************
\tableofcontents

%*************************** SEZNAM SLIK in TABEL  ***********************
\seznamslik
\seznamtabel

%***************************  SEZNAM UPORABLJENIH SIMBOLOV  **************

\seznamsimbolov

V pri�ujo�em zaklju�nem delu so uporabljeni naslednje veli�ine in
simboli:

\begin{table}[h]
\centering
%\begin{footnotesize}
\begin{tabular}{l l l l}
 \hline \multicolumn{2}{c}{\bf{Veli�ina / oznaka}} & \multicolumn{2}{c}{\bf{Enota}}  \\
 \hline
Ime & Simbol & Ime & Simbol \\
 \hline
 �as & $t$  & sekunda & s \\
 frekvenca & $f$  & Hertz & Hz \\
 tlak & $p$  & Pascal & Pa \\
 sila vzgona & $\textbf{\textit{f}}_\text{vz}$  & Newton & N \\
 gostota & $\rho$  & - & kg/m$^3$ \\
 masa telesa  & $m_\text{t}$  & kilogram & kg \\
 vhodna napestost & $U_\text{vh}$ & volt  & V \\
 Jacobijeva matrika & $\mathbf{J}$  & - & - \\
  \hline
\end{tabular}
%\end{footnotesize}
  \caption{Veli�ine in simboli}
  \label{prebojne_trdnosti}
\end{table}



%------------------------ GLAVNI DEL ------------------------------------
\mainmatter
%-------------------------------------------------------------------------


%********************* POVZETEK V SLOVEN��INI ****************************
\povzetek

V pri�ujo�em delu so predstavljena navodila za izdelavo 

\kljucnebesede beseda1, beseda2, beseda3


%*************************** POVZETEK V ANGLE��INI ***********************
\abstract

The thesis addresses ...

\keywords word1, word2, word3


%***************************** UVOD **************************************
\chapter{Uvod} \label{uvod}

Uvod v zaklju�no delo ima namen, da uvede bralca v tematiko
zaklju�nega dela. V njem kandidat raz�leni zahteve in cilje
zaklju�nega dela, po literaturi povzame znane 
re"sitve in oceni
njihov pomen za zaklju�no delo. Sklicevanje na literaturo se v
besedilu ozna�i s "stevilko v oglatem oklepaju, ki jo ima ta v
seznamu uporabljenih virov, in po potrebi navede strani, npr.
\cite{miklavvcivc2010objavljanje} ali \cite[stran 520 -
534]{juvznivc1992diplomska}.

%*********************** OSREDNJA POGLAVJA ********************************
\chapter{Izbira teme zaklju�nega dela} \label{izbira_teme}

\chapter{Princip delovanja senzorja RM44}



\chapter{Dolo"canje kota}

\chapter{Vpliv ekscentri"cnosti}

\section{Vpliv stati"cne ekscentri"cnosti}

\section{Vpliv dinami"cne ekscentri"cnosti}

\section{Rezultati simulacij}

\chapter{Zaklju�ek} \label{zakljucek}





\end{document}
