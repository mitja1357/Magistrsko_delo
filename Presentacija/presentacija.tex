\documentclass{beamer}
\usepackage{pgfpages}
\usepackage{tikz}
\usepackage[utf8]{inputenc}  %Kodna stran za Windows okolje, za linux je kodna stran latin2
\usepackage[Slovene]{babel}
\usepackage{xkeyval}
\usepackage{graphicx}
\usepackage{epstopdf}
\usepackage{xmpincl}
\usepackage{tikz}
\usepackage{breqn}
\usetikzlibrary{calc}
\usepackage[pdftex]{UNI-LJ-FE-Diploma}
\setbeameroption{show notes on second screen}
\hypersetup{pdfpagemode=FullScreen}
\title{Vpliv statične in dinamične
ekscentričnosti magnetnega
senzorja RM44 na napako v
signalu kota}
\author{Mitja Alič}
\institute{Univerza v Ljubljani}
\date{December 21, 2018}

\begin{document}

\begin{frame}
\titlepage
\end{frame}

\begin{frame}
\frametitle{RM44}
\begin{itemize}
\item Robots have the potential to solve many problems
\item Moving from controlled to natural environments is difficult
\item We need methods for them to learn and adapt to new situations
\end{itemize}
\note{Senzor RM44 je 13 bitni enkoder, primeren za merjenje zasuka rotirajočega pogona [5]. Enkoder se nahaja v robustem ohišju, zato je primeren za delovanje
v težkem industrijskem okolju. Oblika izhodnega podatka, je prilagodljiva na
sistem aplikacije v kateri bo uporabljen [6]. Izhod senzorja je lahko analogen v
obliki sinusnega in kosinusnega signala ali linearno spreminjajče se napetosti med
potencialoma GND in V DD v odvisnosti od kota zasuka. Izhod je lahko tudi v
obliki inkrementalnih signalov A in B s katerih se lahko določi smer in relativni
zasuk vrtenja ter signal Ri kateri določa referenčno točko. Izhod je možen tudi
preko SSI vodila. Senzor ima možnost nastavitev resolucije od 5 do 13 bitov
na obrat [7] [5].}
\end{frame}

\begin{frame}

\frametitle{RM44}
\begin{figure}[!ht]
	\centering
	\begin{tikzpicture}
	%\CaCS{3}{0}{0};
	\magnet {0} {0} {0}{ }{1};
	\hall {2.3}{1} {0};
	\draw [decorate,decoration={brace,amplitude=5pt,mirror},xshift= 0pt,yshift=0pt]
	(2.55,0) -- (2.55,1) node [black,midway,xshift= 0.4cm] 
	{\footnotesize $y_0$};
	\draw [decorate,decoration={brace,amplitude=5pt,mirror},xshift= 0pt,yshift=0pt]
	(0,-0.1) -- (2.3,-0.1) node [black,midway,yshift= -0.4cm] 
	{\footnotesize $x_0$};
	\draw[dotted](0,0)--(2.3,0)--(2.3,1) node[yshift = 0.5cm, xshift = 0.4 cm] {$H(x_0,y_0)$};
	\node at (0,0.2) {$S_r(0, 0)=S_m(0, 0)$};
	\end{tikzpicture}
	\caption{Definicija koordinatnega sistema z magnetom in Hall-ovo sondo}
	\label{fig:def_kks}
\end{figure}

\end{frame}
\end{document}